\newpage
\section{Method} \label{Method}

\subsection{Research questions} \label{research_questions}

\begin{comment}
Onderzoeksvraag:
Hoe bruikbaar is Ampersand voor het ontwerpen van register-systemen door middel van het analyseren van wet- en regelgeving in de volksgezondheid en in het bijzonder de Wet-BIG
\end{comment}
Within the CIBG we manage the \acrshort{zorro} that needs to be replaced.
The \acrshort{zorro} is an implementation of \acrshort{big}.
However, because \acrshort{zorro} is end-of-live, it needs to be redesigned and rebuilt.
This is a unique opportunity to test the usefulness of Ampersand for the design in practice.
With the help of Ampersand, for a real-life situation, we make a design that can be put into practice.
This law will be analysed and converted into the relational algebra as used within Ampersand.

The above leads to the following research question:
\newline
\textbf{How useful is Ampersand for designing registry systems by analysing public health legislation and regulations, in particular the \acrshort{big}.}

When investigating the research question, the following sub-questions will contribute to the answer to the research question.
\newline Related questions:
\begin{enumerate}
\item[RQ1]- What knowledge, in the role of software engineer, is needed to use Ampersand.
\item[RQ2]- What are the Concepts, Relationships and Rules in the \acrshort{big}.
\item[RQ3]- How are the laws and regulations set up so that they can be used in a useful way for the Ampersand method.
\item[RQ4]- What are the strengths and weaknesses (SWOT) in using Ampersand for registry systems for a government organization.
\end{enumerate}

When investigating RQ1, we will determine what knowledge is needed to start as a software engineer with Ampersand and to be able to perform the conceptual analysis.
Is it then sufficient to have a brief knowledge of relational algebra and of Ampersand, or is it necessary to have in-depth knowledge of this domain.
The result of RQ2 should follow from the conceptual analysis of the \acrshort{big}.
This is input for making the prototype and the functional design.
Another aspect of the usefulness of the Ampersand method concerns the source of the data.
This data source, the laws and regulations of the \acrshort{big}, should be structured in such a way that concepts, relationships and rules can be derived from it.
In order to determine the usability of the Ampersand method, it is also necessary to be accepted by an organization in addition to the implementation of the method.
The sub-question RQ4 is therefore about the strengths and weaknesses of the Ampersand method.



\begin{comment}
\item[RQ1]- What knowledge is necessary for using Ampersand.
\item[RQ2]- What are the concepts and relationships in the new law big.
\item[RQ3]- Are the results useful for the CIBG organization.
\item[RQ4]- Is the description of the legislation and regulations, set up in such a way that this legislation can be used for the Ampersand method.
\end{comment}


The research question focuses on the usefulness of Ampersand for designing registration systems.
The design of a real life situation of a register system.
The CIBG has asked to take the \acrshort{big} as a case.
The system that supports the \acrshort{big} needs to be replaced.
We focus on a case.
Due to years of involvement in this system, it is not possible to look at the system completely objectively.
So this case is authentic.
This is why we opt for action research as a research approach.

By solving a real world problem in the form of an empirical research method.
A prerequisite for action research is having a problem owner who is willing to cooperate and tackle the problem.
It is an authentic problem and there are authentic knowledge results for those involved.
An additional challenge of this method is the relative unfamiliarity with the method.
Within the field of software engineering, this is a more commonly used method, but not always recognized as such.
We opt for the method of an action research.
The action research\citep{Easterbrook} method was chosen precisely to contribute to Ampersand.

The chosen research method is based on the research results of \cite{Easterbrook}.
\cite{Easterbrook} distinguish between different types of empirical research. 
The empirical research is always presented from someone his own perspective. 
The different philosophical points of view determine the view on the research approach. 
As mentioned, the approach is in the form of a case, in which a confirming case is the starting point. 
This form is used to test existing theories. 
The chosen case is a single case. 
The theory must be shown to hold. 
The unit of analysis concerns the professions in individual health care. 
A weakness of this research method is the creation of bias and more open interpretation.
To avoid these weaknesses, \cite{Easterbrook} recommends using an explicit framework for case selection and collection.
Using only one case prevents this from leading to bias.
Our philosophical position is based on constructivism.
The main concern of this point is to understand how the \acrfull{big} transformation can and will be used in a real government organization.

The mentioned unit of analysis, namely the \acrfull{big} is the basis of the data collection technique.
Because we are talking about an existing law here and there is a lot of knowledge about it within the CIBG organization.
The tool to be used is triangulation~\citep{carter_use_2014, farquhar_triangulation_2020, runeson_guidelines_2008}.
Triangulation offers the possibility to view the source from multiple perspectives.
The perspective of the law itself.
In addition, the engineers have the necessary knowledge and information about the application of the law and from the business perspective.
This is a way of assuring the validity of research through the use of a variety of methods to collect data on the same topic, which involves different types of samples as well as methods of data collection\footnote{\url{https://en.wikipedia.org/wiki/Triangulation_(social_science)}}.




