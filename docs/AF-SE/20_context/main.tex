\section{Context} \label{context}
\begin{wrapfigure}{r}{.5\textwidth} 
    \includegraphics[scale=0.3]
        {Contented_Definitie_Ampersand_Wikipedia-1024x698.png}
    \caption{www.contented.nl/wat-weet-jij-van-het-en-teken-de-ampersand}
    \label{fig:Ampersand definition}
\end{wrapfigure}
The research focuses on the question of the usability of Ampersand(\&).
The ampersand sign~\footnote{\url{https://www.contented.nl/wat-weet-je-van-het-en-teken-de-ampersand}} and the Ampersand method both emphasize the meaning "and self-contained".
On the website of 
Ampersand~\footnote{\url{https://ampersandtarski.gitbook.io/documentation/why-ampersand/business-rules-in-ampersand}} 
we find an interpretation of the statement "standalone (op zichzelfstaand)".

There are more case studies conducted in the past about the usefulness of Ampersand.
Like the graduation study of \citeNonPub{koopman_tim_adequate_2014} and the example study of \citeNonPub{baecke_elleke_argument_2018} about adapting Ampersand in legal environment.
Also in the field of legislation.
This case study aims to demonstrate that this is also possible for legislation coming from the Ministry of Health, Welfare and Sport.

For knowledge of Ampersand, the books by ~\citeNonPub{wedemeijer_lex_joosten_stef_woude_jaap_van_der_rule-based_nodate}
and \citeNonPub{wedemeijer_l_joosten_smm_michels_garkenbout_jlc_werkboek_ontwerpen_met_bedrijfsregelspdf_nodate} from the Open University are available.

\newpage
\section{Results} \label{Results}

\begin{comment}
\textbf{How useful is Ampersand for designing registry systems by analysing public health legislation and regulations, in particular the \acrshort{big}.}

When investigating the research question, the following sub-questions will contribute to the answer to the research question.
\newline Related questions:
\begin{enumerate}
\item[RQ1]- What knowledge, in the role of software engineer, is needed to use Ampersand.
\item[RQ2]- What are the Concepts, Relationships and Rules in the \acrshort{big}.
\item[RQ3]- How are the laws and regulations set up so that they can be used in a useful way for the Ampersand method.
\item[RQ4]- What are the strengths and weaknesses (SWOT) in using Ampersand for registry systems for a government organization.
\end{enumerate}

Hoe nuttig is Ampersand voor het ontwerpen van registratiesystemen door analyse van wet- en regelgeving op het gebied van volksgezondheid, in het bijzonder de Wet-BIG.

Bij het onderzoeken van de onderzoeksvraag zullen de volgende deelvragen bijdragen aan het beantwoorden van de onderzoeksvraag. Gerelateerde vragen:
RQ1 - Welke kennis, in de rol van software engineer, is nodig om Ampersand te gebruiken.
RQ2 - Wat zijn de concepten, relaties en regels in de Wet-BIG.
RQ3 - Hoe zijn de wet- en regelgeving opgezet zodat ze kunnen worden gebruikt in een bruikbare manier voor de Ampersand-methode.
RQ4 - Wat zijn de sterke en zwakke punten (SWOT) bij het gebruik van Ampersand voor registratiesystemen voor een overheidsorganisatie.

\end{comment}

-notities
\newline
------------
\newline
[RQ1]- What knowledge, in the role of software engineer, is needed to use Ampersand.
\begin{enumerate}
    \item rq1- Indelen in Ampersand (patterns) heeft consequenties voor het prototype. VS ontbeert een refactoring optie bij verplaatsen
    \item rq1- Ampersand heeft geen annotatie mogelijkheid -> vergt een separate actie/document om bij te houden wat er geweest is
    \item rq1- aparte excel gemaakt om de multiplicteiten van de relaties uit te schrijven en te ontdekken
    \item rq1- automatische rules zijn beschreven, maar om te implementeren is ook hier veel geduld en proberen nodig. 
    \item rq1- de browser houdt data vast en er moet regelmatig een cache worden geleegd om het nieuwe werkend te krijgen
    \item rq1- docker is ook nog een ding om te leren
    \item rq1- elke relatie is onderdeel van een record structuur
    \item rq1- er is geen swagger gemaakt voor de api; 
    \item rq1- het toevoegen van stukjes php script moet mogelijk zijn, maar is niet duidelijk hoe
    \item rq1- html href en target blank werkt niet
    \item rq1- implementatie in docker met RAP werkt wel, maar niet met includes om dat er steeds een nieuwe directory wordt aangemaakt; pas waneer er echt lokaal wordt gedraaid dan werkt het ook met includes
    \item rq1- in het begin niet duidelijk wanneer C of c in de INTERFACE toe te passen; mogelijk staat dit wel in de handleiding, maar moet je toch proefondervindelijk ontdekken.
    \item rq1- inrichting van A in lokale omgeving is specifiek en niet evident; hulp is hier nodig
    \item rq1- kiss
    \item rq1- middels include statements wordt de volgorde van document bepaald, maar niet overal zijn includes nodig
    \item rq1- notatie wijze van concepten en relaties en rules zijn deels vastgelegd. Enkel de eerste positie is hoofd of kleine letter; geen advies over overige schrijfwijze.
    \item rq1- om een rule toe te passen is veel geduld en oefening nodig; vrij steile leercurve; 
    \item rq1- op internet geen voorbeeld te vinden, enkel in de repo van A zelf; en dat is moeizaam zoeken
    \item rq1- output in latex 
    \item rq1- represent definieert een type van een concept, maar datetime geeft problemen bij de interface
    \item rq1- TOT wordt meestal ondervangen door een tot-rule -> blijkt dat een TOT tot gevolg heeft dat iets kan worden gesaved wanneer ingevuld, terwijl een tot-rule een save kan plaatsvinden terwijl de melding open blijft staan
    \item rq1- VS heeft ook geen generieken zoek optie over de adl's heen
    \item rq1- wet lezen is een vak 
    \item rq1- XML download van wetBig lijkt een logische stap, echter is deze zo complex opgebouwd dat dit geen zin heeft (idem voor json)
\end{enumerate}
-------
\newline
[RQ2]- What are the Concepts, Relationships and Rules in the \acrshort{big}.
\begin{enumerate}
    \item rq2- inhoudelijk -> wetbig omvat ook tuchtrecht, dat is een andere tak van sport
    \item rq2- wordt enkel gebruik gemaakt van UNI, TOT, INJ en SUR
    \item rq2/rq3- in de wetbig wordt specialismes genoemd, maar geen lijst oid
\end{enumerate}

--------
\newline
[RQ3]- How are the laws and regulations set up so that they can be used in a useful way for the Ampersand method.
\begin{enumerate}
    \item rq3- bij interface kan je een FOR toevoegen voor autorisatie
    \item rq3- door het lezen van de wet wordt er een opbouw duidelijk -> persoon; inschrijving; registratie->beheer; tucht->maatregelen; 
    \item rq3- er zijn delen van de wet die niet meer geldig zijn, deze worden niet meegenomen
    \item rq3- er zijn meer wetten bij betrokken dan enkel de wet BIG
    \item rq3- het toevoegen van de juiste beschrijving bij een concept en relatie is nog niet zo eenvoudig; gemakkelijk om af te dwalen en een eigen interpretatie toe te voegen. Ontbreekt een directe toets.
\end{enumerate}
--------
\newline
[RQ4]- What are the strengths and weaknesses (SWOT) in using Ampersand for registry systems for a government organization.
\begin{enumerate}
    \item rq4- A kan niet rekenen; Maar aangezien A statisch is, kunnen proces gegevens op andere manier worden bewaakt.
    \item rq4- api koppeling werkt goed, echter komen hele meldingen terug; zouden eigenlijk codes moeten krijgen 
    \item rq4- inbedden in architectuur 
    \item rq4- interface levert veel meldingen en deze blijven ook staan
    \item rq4- postman gebruikt voor API koppeling met A. 
    \item rq4- ui model; wettenkern met gedeelde concepten en proces deel; wettenkern is specifiek wet; gedeelde concepten zijn ook onderdeel van de wet maar komen ook elders voor;
    \item rq4- wat gebeurd er als A geimplementeerd is en er toch wijzigingen in de structuur plaatsvinden (normaal voor software)
\end{enumerate}


\newpage
\section{Results} \label{Results}

\begin{comment}
\textbf{How useful is Ampersand for designing registry systems by analysing public health legislation and regulations, in particular the \acrshort{big}.}

When investigating the research question, the following sub-questions will contribute to the answer to the research question.
\newline Related questions:
\begin{enumerate}
\item[RQ1]- What knowledge, in the role of software engineer, is needed to use Ampersand.
\item[RQ2]- What are the Concepts, Relationships and Rules in the \acrshort{big}.
\item[RQ3]- How are the laws and regulations set up so that they can be used in a useful way for the Ampersand method.
\item[RQ4]- What are the strengths and weaknesses (SWOT) in using Ampersand for registry systems for a government organization.
\end{enumerate}

Hoe nuttig is Ampersand voor het ontwerpen van registratiesystemen door analyse van wet- en regelgeving op het gebied van volksgezondheid, in het bijzonder de Wet-BIG.

Bij het onderzoeken van de onderzoeksvraag zullen de volgende deelvragen bijdragen aan het beantwoorden van de onderzoeksvraag. Gerelateerde vragen:
RQ1 - Welke kennis, in de rol van software engineer, is nodig om Ampersand te gebruiken.
RQ2 - Wat zijn de concepten, relaties en regels in de Wet-BIG.
RQ3 - Hoe zijn de wet- en regelgeving opgezet zodat ze kunnen worden gebruikt in een bruikbare manier voor de Ampersand-methode.
RQ4 - Wat zijn de sterke en zwakke punten (SWOT) bij het gebruik van Ampersand voor registratiesystemen voor een overheidsorganisatie.

\end{comment}

-notities
\newline
------------
\newline
[RQ1]- What knowledge, in the role of software engineer, is needed to use Ampersand.
\begin{enumerate}
    \item rq1- Indelen in Ampersand (patterns) heeft consequenties voor het prototype. VS ontbeert een refactoring optie bij verplaatsen
    \item rq1- Ampersand heeft geen annotatie mogelijkheid -> vergt een separate actie/document om bij te houden wat er geweest is
    \item rq1- aparte excel gemaakt om de multiplicteiten van de relaties uit te schrijven en te ontdekken
    \item rq1- automatische rules zijn beschreven, maar om te implementeren is ook hier veel geduld en proberen nodig. 
    \item rq1- de browser houdt data vast en er moet regelmatig een cache worden geleegd om het nieuwe werkend te krijgen
    \item rq1- docker is ook nog een ding om te leren
    \item rq1- elke relatie is onderdeel van een record structuur
    \item rq1- er is geen swagger gemaakt voor de api; 
    \item rq1- het toevoegen van stukjes php script moet mogelijk zijn, maar is niet duidelijk hoe
    \item rq1- html href en target blank werkt niet
    \item rq1- implementatie in docker met RAP werkt wel, maar niet met includes om dat er steeds een nieuwe directory wordt aangemaakt; pas waneer er echt lokaal wordt gedraaid dan werkt het ook met includes
    \item rq1- in het begin niet duidelijk wanneer C of c in de INTERFACE toe te passen; mogelijk staat dit wel in de handleiding, maar moet je toch proefondervindelijk ontdekken.
    \item rq1- inrichting van A in lokale omgeving is specifiek en niet evident; hulp is hier nodig
    \item rq1- kiss
    \item rq1- middels include statements wordt de volgorde van document bepaald, maar niet overal zijn includes nodig
    \item rq1- notatie wijze van concepten en relaties en rules zijn deels vastgelegd. Enkel de eerste positie is hoofd of kleine letter; geen advies over overige schrijfwijze.
    \item rq1- om een rule toe te passen is veel geduld en oefening nodig; vrij steile leercurve; 
    \item rq1- op internet geen voorbeeld te vinden, enkel in de repo van A zelf; en dat is moeizaam zoeken
    \item rq1- output in latex 
    \item rq1- represent definieert een type van een concept, maar datetime geeft problemen bij de interface
    \item rq1- TOT wordt meestal ondervangen door een tot-rule -> blijkt dat een TOT tot gevolg heeft dat iets kan worden gesaved wanneer ingevuld, terwijl een tot-rule een save kan plaatsvinden terwijl de melding open blijft staan
    \item rq1- VS heeft ook geen generieken zoek optie over de adl's heen
    \item rq1- wet lezen is een vak 
    \item rq1- XML download van wetBig lijkt een logische stap, echter is deze zo complex opgebouwd dat dit geen zin heeft (idem voor json)
\end{enumerate}
-------
\newline
[RQ2]- What are the Concepts, Relationships and Rules in the \acrshort{big}.
\begin{enumerate}
    \item rq2- inhoudelijk -> wetbig omvat ook tuchtrecht, dat is een andere tak van sport
    \item rq2- wordt enkel gebruik gemaakt van UNI, TOT, INJ en SUR
    \item rq2/rq3- in de wetbig wordt specialismes genoemd, maar geen lijst oid
\end{enumerate}

--------
\newline
[RQ3]- How are the laws and regulations set up so that they can be used in a useful way for the Ampersand method.
\begin{enumerate}
    \item rq3- bij interface kan je een FOR toevoegen voor autorisatie
    \item rq3- door het lezen van de wet wordt er een opbouw duidelijk -> persoon; inschrijving; registratie->beheer; tucht->maatregelen; 
    \item rq3- er zijn delen van de wet die niet meer geldig zijn, deze worden niet meegenomen
    \item rq3- er zijn meer wetten bij betrokken dan enkel de wet BIG
    \item rq3- het toevoegen van de juiste beschrijving bij een concept en relatie is nog niet zo eenvoudig; gemakkelijk om af te dwalen en een eigen interpretatie toe te voegen. Ontbreekt een directe toets.
\end{enumerate}
--------
\newline
[RQ4]- What are the strengths and weaknesses (SWOT) in using Ampersand for registry systems for a government organization.
\begin{enumerate}
    \item rq4- A kan niet rekenen; Maar aangezien A statisch is, kunnen proces gegevens op andere manier worden bewaakt.
    \item rq4- api koppeling werkt goed, echter komen hele meldingen terug; zouden eigenlijk codes moeten krijgen 
    \item rq4- inbedden in architectuur 
    \item rq4- interface levert veel meldingen en deze blijven ook staan
    \item rq4- postman gebruikt voor API koppeling met A. 
    \item rq4- ui model; wettenkern met gedeelde concepten en proces deel; wettenkern is specifiek wet; gedeelde concepten zijn ook onderdeel van de wet maar komen ook elders voor;
    \item rq4- wat gebeurd er als A geimplementeerd is en er toch wijzigingen in de structuur plaatsvinden (normaal voor software)
\end{enumerate}


\newpage
\section{Results} \label{Results}

\begin{comment}
\textbf{How useful is Ampersand for designing registry systems by analysing public health legislation and regulations, in particular the \acrshort{big}.}

When investigating the research question, the following sub-questions will contribute to the answer to the research question.
\newline Related questions:
\begin{enumerate}
\item[RQ1]- What knowledge, in the role of software engineer, is needed to use Ampersand.
\item[RQ2]- What are the Concepts, Relationships and Rules in the \acrshort{big}.
\item[RQ3]- How are the laws and regulations set up so that they can be used in a useful way for the Ampersand method.
\item[RQ4]- What are the strengths and weaknesses (SWOT) in using Ampersand for registry systems for a government organization.
\end{enumerate}

Hoe nuttig is Ampersand voor het ontwerpen van registratiesystemen door analyse van wet- en regelgeving op het gebied van volksgezondheid, in het bijzonder de Wet-BIG.

Bij het onderzoeken van de onderzoeksvraag zullen de volgende deelvragen bijdragen aan het beantwoorden van de onderzoeksvraag. Gerelateerde vragen:
RQ1 - Welke kennis, in de rol van software engineer, is nodig om Ampersand te gebruiken.
RQ2 - Wat zijn de concepten, relaties en regels in de Wet-BIG.
RQ3 - Hoe zijn de wet- en regelgeving opgezet zodat ze kunnen worden gebruikt in een bruikbare manier voor de Ampersand-methode.
RQ4 - Wat zijn de sterke en zwakke punten (SWOT) bij het gebruik van Ampersand voor registratiesystemen voor een overheidsorganisatie.

\end{comment}

-notities
\newline
------------
\newline
[RQ1]- What knowledge, in the role of software engineer, is needed to use Ampersand.
\begin{enumerate}
    \item rq1- Indelen in Ampersand (patterns) heeft consequenties voor het prototype. VS ontbeert een refactoring optie bij verplaatsen
    \item rq1- Ampersand heeft geen annotatie mogelijkheid -> vergt een separate actie/document om bij te houden wat er geweest is
    \item rq1- aparte excel gemaakt om de multiplicteiten van de relaties uit te schrijven en te ontdekken
    \item rq1- automatische rules zijn beschreven, maar om te implementeren is ook hier veel geduld en proberen nodig. 
    \item rq1- de browser houdt data vast en er moet regelmatig een cache worden geleegd om het nieuwe werkend te krijgen
    \item rq1- docker is ook nog een ding om te leren
    \item rq1- elke relatie is onderdeel van een record structuur
    \item rq1- er is geen swagger gemaakt voor de api; 
    \item rq1- het toevoegen van stukjes php script moet mogelijk zijn, maar is niet duidelijk hoe
    \item rq1- html href en target blank werkt niet
    \item rq1- implementatie in docker met RAP werkt wel, maar niet met includes om dat er steeds een nieuwe directory wordt aangemaakt; pas waneer er echt lokaal wordt gedraaid dan werkt het ook met includes
    \item rq1- in het begin niet duidelijk wanneer C of c in de INTERFACE toe te passen; mogelijk staat dit wel in de handleiding, maar moet je toch proefondervindelijk ontdekken.
    \item rq1- inrichting van A in lokale omgeving is specifiek en niet evident; hulp is hier nodig
    \item rq1- kiss
    \item rq1- middels include statements wordt de volgorde van document bepaald, maar niet overal zijn includes nodig
    \item rq1- notatie wijze van concepten en relaties en rules zijn deels vastgelegd. Enkel de eerste positie is hoofd of kleine letter; geen advies over overige schrijfwijze.
    \item rq1- om een rule toe te passen is veel geduld en oefening nodig; vrij steile leercurve; 
    \item rq1- op internet geen voorbeeld te vinden, enkel in de repo van A zelf; en dat is moeizaam zoeken
    \item rq1- output in latex 
    \item rq1- represent definieert een type van een concept, maar datetime geeft problemen bij de interface
    \item rq1- TOT wordt meestal ondervangen door een tot-rule -> blijkt dat een TOT tot gevolg heeft dat iets kan worden gesaved wanneer ingevuld, terwijl een tot-rule een save kan plaatsvinden terwijl de melding open blijft staan
    \item rq1- VS heeft ook geen generieken zoek optie over de adl's heen
    \item rq1- wet lezen is een vak 
    \item rq1- XML download van wetBig lijkt een logische stap, echter is deze zo complex opgebouwd dat dit geen zin heeft (idem voor json)
\end{enumerate}
-------
\newline
[RQ2]- What are the Concepts, Relationships and Rules in the \acrshort{big}.
\begin{enumerate}
    \item rq2- inhoudelijk -> wetbig omvat ook tuchtrecht, dat is een andere tak van sport
    \item rq2- wordt enkel gebruik gemaakt van UNI, TOT, INJ en SUR
    \item rq2/rq3- in de wetbig wordt specialismes genoemd, maar geen lijst oid
\end{enumerate}

--------
\newline
[RQ3]- How are the laws and regulations set up so that they can be used in a useful way for the Ampersand method.
\begin{enumerate}
    \item rq3- bij interface kan je een FOR toevoegen voor autorisatie
    \item rq3- door het lezen van de wet wordt er een opbouw duidelijk -> persoon; inschrijving; registratie->beheer; tucht->maatregelen; 
    \item rq3- er zijn delen van de wet die niet meer geldig zijn, deze worden niet meegenomen
    \item rq3- er zijn meer wetten bij betrokken dan enkel de wet BIG
    \item rq3- het toevoegen van de juiste beschrijving bij een concept en relatie is nog niet zo eenvoudig; gemakkelijk om af te dwalen en een eigen interpretatie toe te voegen. Ontbreekt een directe toets.
\end{enumerate}
--------
\newline
[RQ4]- What are the strengths and weaknesses (SWOT) in using Ampersand for registry systems for a government organization.
\begin{enumerate}
    \item rq4- A kan niet rekenen; Maar aangezien A statisch is, kunnen proces gegevens op andere manier worden bewaakt.
    \item rq4- api koppeling werkt goed, echter komen hele meldingen terug; zouden eigenlijk codes moeten krijgen 
    \item rq4- inbedden in architectuur 
    \item rq4- interface levert veel meldingen en deze blijven ook staan
    \item rq4- postman gebruikt voor API koppeling met A. 
    \item rq4- ui model; wettenkern met gedeelde concepten en proces deel; wettenkern is specifiek wet; gedeelde concepten zijn ook onderdeel van de wet maar komen ook elders voor;
    \item rq4- wat gebeurd er als A geimplementeerd is en er toch wijzigingen in de structuur plaatsvinden (normaal voor software)
\end{enumerate}





