\subsection{Wet-BIG} \label{section:big}
During the graduation project, research was conducted into the suitability of Ampersand for designing registers for the government.
These registers are always based on legislation and regulations.
The research focuses on a specific law, namely the \acrshort{big}.
The \acrshort{cibg} is the executing party for the \acrshort{big}.

The first health care law was enacted in 1865.
This law, together with eleven other laws, forms the basis of the \acrfull{big}~\footnote{\url{https://nl.wikipedia.org/wiki/Wet_op_de_beroepen_in_de_individuele_gezondheidszorg}}.
The \acrshort{big}~\citepNonPub{van_wet_2018} replaces the ban on medical action by unauthorized persons by granting responsibilities to healthcare providers.
The professions regulated in Article 3 of the \acrshort{big}, which include doctors, nurses and physiotherapists, have a compulsory registration with periodic re-registration, a statutory disciplinary law and a protected professional title.
The former paramedical professions that are now regulated in Article 34 of the \acrshort{big} have no registration obligation and no legally regulated disciplinary law.
These Article 34 professions are only certified.

Then we have the \acrshort{big}.
It describes the following:
\blockquote{de tot dusverre geldende wettelijke regeling op het gebied van de uitoefening van de geneeskunst, inhoudende een het gehele gebied der geneeskunst bestrijkend verbod van beroepsuitoefening zonder hiertoe wettelijk verleende bevoegdheid, te vervangen door een regeling welke een ruimer gebied van individuele gezondheidszorg bestrijkt en waarbij slechts het verrichten van bij de wet
aangewezen categorieën van handelingen wordt voorbehouden aan categorieën van daartoe overeenkomstig de wet gekwalificeerden, terwijl het voeren van wettelijk bescherm-de beroepstitels uitsluitend toekomt aan degenen die in de voor de desbetreffende beroepen overeenkomstig de wet ingestelde registers ingeschreven staan en ten aanzien van andere beroepen op het gebied van de individuele gezondheidszorg voorzien wordt in de mogelijkheid tot het regelen van de opleiding tot die beroepen; 
\newline
voor onderscheidene categorieën van overeenkomstig de wet gekwa-lificeerden een aan de gebleken behoeften aangepaste regeling van tuchtrechtspraak in het leven te roepen; }
This law consists of 148 articles, of which a number are still pending.
The articles are also regularly updated.
More recently, in July 2020 there were still amendments to the law.

On the  website  {\url{wetten.overheid.nl}}~\footnote{\url{https://wetten.overheid.nl/BWBR0006251/2020-07-01}} are the Dutch laws including the \acrshort{big}.
Within this website, the content is kept up to date.
All changes to the law are traceable.
On this website a user can find any law in any given period of time.


This law does not stand alone.
Appendix \ref{list:ass-laws-regulations} contains an overview of the laws and regulations that relate to the \acrshort{big}.

When we talk about the new \acrshort{big}, we can say that it is not there yet.
Proposals have been written about a new law~\citepNonPub{bussemaker_jet_2019}, but it has not yet been enacted into law, and it is also a descriptive text and not an article-by-article summary.




\subsubsection{Law tax analysis} \label{law_analysis}
In the Netherlands, the tax authorities have also devised a method for analyzing laws.
The tax authorities have developed a method~\citepNonPub{ausems_wetsanalyse_2021} that is intended to analyse tax laws and other laws.
This is performed in these 6 steps:
\begin{enumerate}
    \item Determining the work area.
    \item Making the structure visible in legislation.
    \item Defining the meaning of legislation.
    \item Validate the analysis results.
    \item Identify missing execution policy.
    \item Setting up the knowledge model.
\end{enumerate}
Emphasis is placed on the cooperation between the implementer, ICT and policy.
By going through the method step by step, one arrives at a shared language.
This shared language includes the definition of concepts by the collaborating parties.
An important part of the approach is dividing the law into small pieces and always refer to these pieces of law in the implementation.
As a result, the method meets the requirement of the justification of government decisions.
The decisions are traceable, explainable, and it is possible to account for them.
What is not clear from the webinar~\citeNonPub{belastingdienst_webinar_2021} is how these steps were converted into an implementation.
The book~\citeNonPub{ausems_wetsanalyse_2021} indicates that the legal analysis method does not contain a development tool, but that the Tax and Customs Administration has developed an instrument based on the legal model, which is not freely available.