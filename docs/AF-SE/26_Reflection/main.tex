\newpage
\section{Reflection} \label{reflection}

The conclusions have been drawn and recommendations have been made on the basis of these.
Now is the time to reflect on the quality of the research and the validity of the conclusions.
My research has yielded a lot of observations made from my perspective.
From my IT perspective I have looked at the Ampersand tool and method.
At the start of the research, I was very focused on how Ampersand works, so I started looking at the use of Ampersand and that's what the first observations are about.
My observations therefore have the character of a testing Ampersand.
Anything missing or differs from what I would expect is noted.

At a later stage, when the first pieces of the prototype were running, I focused on the \acrlong{ca}.
There I was discovering how this had to be built and how I had to deal with the scripts to get a logical story from it.
Building on the \acrlong{ca} naturally also affects the prototype.

In order to make \acrlong{ca} the law had to be analyzed as well.
I got a lot less far with this than I would have liked.
That's because I had to get through the first two phases if I wanted to be able to perform this phase.
And then it turns out that the law is difficult to understand.
The danger I noticed was that I suffered from bias.
Because of the approach in the form of \acrshort{ar} I was in the middle of the research.
Partly due to the text that is difficult to comprehend and the knowledge of \acrshort{zorro}, the legal text was not always looked closely enough and sometimes a shape was given to it based on our own knowledge.
We failed to correct this in all cases in the \acrlong{ca}.

After building part of the prototype and also having a version of the \acrlong{ca}, I started talking with some colleagues.
The prototype was not always well understood, but the \acrlong{ca} was perceived as recognizable.
The prototype was in the different form than people were used to and the terminology is not based on the \acrshort{big} at all.
These are sometimes different terms than are used within the register systems.

The research focused on the usefulness of Ampersand for use as a design tool for registry systems.
I have sufficiently demonstrated that the usefulness of Ampersand as a method and tool is fine.
Whether it is also useful for the organization depends on the will to use it.