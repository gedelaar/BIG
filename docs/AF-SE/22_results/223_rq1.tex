\begin{comment}
\textbf{How useful is Ampersand for designing registry systems by analysing public health legislation and regulations, in particular the \acrshort{big}.}

When investigating the research question, the following sub-questions will contribute to the answer to the research question.
\newline Related questions:
\begin{enumerate}
\item[RQ1]- What knowledge, in the role of software engineer, is needed to use Ampersand.
\item[RQ2]- What are the Concepts, Relationships and Rules in the \acrshort{big}.
\item[RQ3]- How are the laws and regulations set up so that they can be used in a useful way for the Ampersand method.
\item[RQ4]- What are the strengths and weaknesses (SWOT) in using Ampersand for registry systems for a government organization.
\end{enumerate}

Hoe nuttig is Ampersand voor het ontwerpen van registratiesystemen door analyse van wet- en regelgeving op het gebied van volksgezondheid, in het bijzonder de Wet-BIG.

Bij het onderzoeken van de onderzoeksvraag zullen de volgende deelvragen bijdragen aan het beantwoorden van de onderzoeksvraag. Gerelateerde vragen:
RQ1 - Welke kennis, in de rol van software engineer, is nodig om Ampersand te gebruiken.
RQ2 - Wat zijn de concepten, relaties en regels in de Wet-BIG.
RQ3 - Hoe zijn de wet- en regelgeving opgezet zodat ze kunnen worden gebruikt in een bruikbare manier voor de Ampersand-methode.
RQ4 - Wat zijn de sterke en zwakke punten (SWOT) bij het gebruik van Ampersand voor registratiesystemen voor een overheidsorganisatie.

\end{comment}



\subsection{Use of Ampersand}\label{use_of_ampersand}
When searching for the question "\acrlong{rq1}" it was decided to split this question even further and to label the items found.
A layout has been made in advance.
These items of this classification always provide a different perspective on the sub-question.
\begin{comment}
When asking questions about the knowledge required in the role of software engineers to be able to use Ampersand, the focus is not only on the Ampersand method itself.
We must also ask ourselves whether, in addition to Ampersand, knowledge of the underlying theory on which Ampersand is based is also required.
Is it necessary to know relationship algebra.
The environment in which a prototype runs also plays a role.
This includes browser settings and the use of containers (Docker) in which Ampersand runs.
During development we are dealing with an \acrfull{ide}.
The use of such a \acrshort{ide} for determining the usability of Ampersand is relevant.
Internet support is now indispensable when creating the scripts.
Scripting for Ampersand also requires support for web search capabilities.
But can this be found for a relatively unknown method like Ampersand?
Finally, we look at output produced by method.
Is it true that how the method is used also determines what comes out?
This applies to both the prototype and the generated documentation.
Subsequently, the approach to this process, the use of the method, among other things, determines the result.
These aspects will be discussed in the following paragraphs.
The sub-questions can be further subdivided into small parts.
Starting from the question "\acrlong{rq1}", the question then follows which observations relate to this knowledge.
Is it knowledge about only Ampersand or is knowledge also required of relation algebra.
By working with Ampersand, we are also confronted with the environment in which we develop it.
Ampersand stands alone as a language, but in use there are development tools like \acrlong{vsc} in this case.
To run it, you need a Docker container.
That is at least the implementation as the \acrlong{ou} provides.
During the development phase, examples are searched on the internet.
And these are generally easy to find, is that also the case at Ampersand?
Ampersand also provides documentation and a prototype.
Here too, knowledge of Ampersand is required.
Finally, you also have to ask yourself whether the way Ampersand is used, the approach to the project, also requires knowledge of Ampersand.
There are a number of angles that play around Ampersand's knowledge.
\end{comment}

\subsubsection{Observation summary}
A list of occurrences is included at the end of this section.

The Ampersand item has a wide palette of comments.
This is about the setup of the tool, looking for examples to be able to work with it.
Considerable experience is required to work with it.
A remark that comes up a lot with various items is about overview.
Overview in terms of usage and how to track progress.

\subsubsection{Observations }
The observations in order of multiplicity of occurrence within this sub question.
\newcounter{subsubsubsection}[subsubsection]
%Concept:
%- notatie 2
%- shared concepts 2
%- werkwijze bij het ontdekken van een concept
%- bepaalde concepten als datum en tijd zijn complex in gebruik
%- waar worden bepaalde concepten gebruikt (in script) en in relaties
%- dezelfde concepten zijn vaker te definieren, onbewust zou dit redefines kunnen worden 2
%- wanneer is iets een concept -> immutable
%v
\paragraph{Concept}
The term Concept is an important concept within Ampersand.
The Concept is described in detail on the website of Ampersand~\footnote{\url{https://ampersandtarski.gitbook.io/documentation/the-language-ampersand/syntactical-conventions/the-concept-statement}}.
However, there are observations on the Concept that are described on this site.
These are observations related to the notation of a Concept.
The fact that a Concept is immutable is also described there.
An example of the use of the concept of date is also described.

%v
The comments regarding items not mentioned on the website relate to the use of the Concepts.
It concerns the use and reuse of concepts within the scripts.
The definitions of concepts go beyond the patterns.
It is possible to use the same Concepts in multiple places.
Redefinition of a Concept is possible with impunity.
This redefinition may lead to the same Concept with a different definition.

\paragraph{Interface}
% CRUD met hoofd en kleine letters
% de concepts in de interface moeten van het type object zijn
% op de crud U/l wordt niet gecontroleerd wijzigingen hebben niet altijd inpact
% max 1 TOT in de interface, anders wordt de data niet opgeslagen
% waarneming interface behoort tot design en niet enkel prototype
% views in HTML niet werkend te krijgen
% linkto in de interface; dropdown, signatuur
% browser issue omdat de cache vasthoud
% target blank wordt verwijderd.
The interface has some implementation challenges.
The website of the Ampersand refers to the operation of the Interface from several places.
Within the Ampersand notation it falls into the Service category.
The \acrlong{crud} usage leads to confusion.
The use of upper and lower case letters is not validated.
And in some situations it has no visible effect either.

When setting up the Interface, the requirement is that the Input Concept is of type object.
\begin{lstlisting}
INTERFACE RegistratieArts FOR MEDEWERKER : I[Registratie] cRud
\end{lstlisting}

The use of the Interface within the browser has as a point of interest that the browser has a cache that stores outdated data.
It is necessary to clear the cache after changes in the Interface.

The multiplicity of total \footnote{For any a in A there must be at least one b in B in the population of r} must be only one in the Interface.
With multiple occurrences in one Interface, the data can no longer be stored.

A LINKTO signature occurring in multiple Interfaces causes unwanted behavior in the Interface.
When clicking on the link, a choice is offered of all links with the relevant signature.

The use of Views within the Interface is well described in the documentation, but not clear enough.
It is not always working.
The examples do not contribute to this either.

The Interface is part of the design.
It was expected that this would only be part of the prototype, but is of course part of the whole.

\paragraph{Ampersand}
% the documentation gives an indication of how to configure ampersand and make it work locally. Using XAMP.  All this is not going to work.     Not clear why.     It did work in a Docker environment.
% Little to be found about Ampersand except in its own repos
% The need for overview is there as the script grows.
% it comes down to wanting to maintain an overview.     That is what I want with the annotation.
% Quick fix of a bug in Ampersand by the development team.
% There should always be someone with experience in the background or in the collaboration.
% Practice a lot and keep using it.
Implementing Ampersand locally does not work.
Despite description about the possible use of XAMPP.
It is possible, with some help, to make this work via Docker.

Not much information is available on the internet about Ampersand outside of the Ampersand website itself.
In addition, stackoverflow is a resource and information can be found on GitHub\footnote{\url{https://github.com/AmpersandTarski}}.

When processing the \acrshort{big} in Ampersand it is difficult to keep an overview.
Overview from different perspectives.
Pick up where you left off in the text.
And which parts are defined as concept and relationship.
The latter to avoid duplicates.


\paragraph{Rule}


\subsubsection{Knowledge of Ampersand}
\begin{comment}
- wat zijn de exacte waarnemingen geweest!
plaats hier de afgehandelde items.


\end{comment}









