\begin{comment}
\textbf{How useful is Ampersand for designing registry systems by analysing public health legislation and regulations, in particular the \acrshort{big}.}

When investigating the research question, the following sub-questions will contribute to the answer to the research question.
\newline Related questions:
\begin{enumerate}
\item[RQ1]- What knowledge, in the role of software engineer, is needed to use Ampersand.
\item[RQ2]- What are the Concepts, Relationships and Rules in the \acrshort{big}.
\item[RQ3]- How are the laws and regulations set up so that they can be used in a useful way for the Ampersand method.
\item[RQ4]- What are the strengths and weaknesses (SWOT) in using Ampersand for registry systems for a government organization.
\end{enumerate}

Hoe nuttig is Ampersand voor het ontwerpen van registratiesystemen door analyse van wet- en regelgeving op het gebied van volksgezondheid, in het bijzonder de Wet-BIG.

Bij het onderzoeken van de onderzoeksvraag zullen de volgende deelvragen bijdragen aan het beantwoorden van de onderzoeksvraag. Gerelateerde vragen:
RQ1 - Welke kennis, in de rol van software engineer, is nodig om Ampersand te gebruiken.
RQ2 - Wat zijn de concepten, relaties en regels in de Wet-BIG.
RQ3 - Hoe zijn de wet- en regelgeving opgezet zodat ze kunnen worden gebruikt in een bruikbare manier voor de Ampersand-methode.
RQ4 - Wat zijn de sterke en zwakke punten (SWOT) bij het gebruik van Ampersand voor registratiesystemen voor een overheidsorganisatie.

\end{comment}

\subsection{Use of Ampersand}\label{use_of_ampersand}
In the search for the question "\acrlong{rq1}" we found the following perspectives.
When asking questions about the knowledge needed in the role of software engineers to be able to use Ampersand, the focus is not only on the Ampersand method itself.
We should also ask ourselves whether, in addition to Ampersand, knowledge is also required of the underlying theory on which Ampersand is based.
Is it necessary to know relation algebra.
The environment in which a prototype runs also plays a role.
This includes browser settings and the use of containers (Docker) in which Ampersand runs.
During development we are dealing with an \acrfull{ide}.
The use of such a \acrshort{ide} for determining the usability of Ampersand is relevant.
Internet support is now indispensable when creating the scripts.
Creating scripts for Ampersand also requires support for internet search capabilities.
But can this be found for a relatively unknown method as Ampersand?
Finally, we look at output produced by method.
Is it the case that the way in which the method is used also determines what comes out?
This applies to both the prototype and the generated documentation.
Then the approach to this process, the use of the method, among other things, determines the result.
These aspects will be discussed in the following paragraphs.

\subsubsection{Knowledge of Ampersand}
\begin{comment}
- wat zijn de exacte waarnemingen geweest!
plaats hier de afgehandelde items.
\end{comment}
\begin{enumerate}
    \item rq1-4 automatische \textbf{rules} zijn beschreven, maar om te implementeren is ook hier veel geduld en proberen nodig. 
    \newline\textbf{obs}: niet eenvoudig om te maken
    
    \item rq1-7:10-11: elke \textbf{relatie} is onderdeel van een record structuur
    \newline\textbf{obs}: goed om te zien hoe de database structuur wordt vastgesteld. 
    Dit staat vast wel ergens vermeld hoe dit gebeurt.
    Maar via reversed enginering kan dit vastgesteld worden.
    geeft vooral inzicht en maakt het tastbaarder.
    
    \item rq1-8:14-11: er is geen swagger gemaakt voor de \textbf{api}; 
    \newline\textbf{obs}: bij gebruik willen maken van een externe invoer, dan zijn api beschrijvingen zeer relevant. 
    Deze worden niet automatisch gegenereert. 
    Dat zou wel wenselijk zijn
    
    \item rq1-9 het toevoegen van stukjes \textbf{php} script moet mogelijk zijn, maar is niet duidelijk hoe
    \newline\textbf{obs}: onbreekt nog informatie hoe dit te doen
    
    \item rq1-11 implementatie in \textbf{docker} met RAP werkt wel, maar niet met includes om dat er steeds een nieuwe directory wordt aangemaakt; pas waneer er echt lokaal wordt gedraaid dan werkt het ook met includes
    \newline\textbf{obs}: voor eenvoudige toepassing is RAP prima geschikt, echter om een systeem te ontwerpen is echt een development omgeving nodig.
    
    \item rq1-12 in het begin niet duidelijk wanneer C of c in de \textbf{INTERFACE} toe te passen; mogelijk staat dit wel in de handleiding, maar moet je toch proefondervindelijk ontdekken.
    \newline\textbf{obs}: Stoeien met het toestaand van CRUD onderdelen. 
    Dit moet in het begin van het traject proefondervindelijk worden gedaan.
    
    \item rq1-14:10-11: kiss
    \newline\textbf{obs}:  houdt de oplossing zo simpel mogelijk. 
    Geldt natuurlijk niet alleen voor Ampersand, maar ook voor andere talen en methodes.
    
    \item rq1-16 notatie wijze van \textbf{concepten} en \textbf{relaties} en rules zijn deels vastgelegd. Enkel de eerste positie is hoofd of kleine letter; geen advies over overige schrijfwijze.
    \newline\textbf{obs}: Je wordt niet gedwongen in een bepaalde structuur te werken. 
    Is dat een voordeel of  nadeeel?

    \item rq1-40:10-10/27-11: \textbf{Interface} concepten moeten van het type object zijn (represent). 
    waarom kunnen hier geen alpha of integer zijn.
    \newline\textbf{obs}: Interface wilde niet op de juiste manier werken. 
    DIt werd veroorzaakt doordat de interface concept niet van het type object was.
    
    \item rq1-50:30-10: het \textbf{represent} statement laat de interface anders reageren. Bij gebruik hiervan verdwijnt de "+"
    \newline\textbf{obs}: vreemd gedrag, nog even nazoeken waardoor dit nu gebeurde
    
    \item rq1-52:2-11: als een onderdeel niet werkt, of niet werkt zoals je wenst, dan weglaten uit het ontwerp
    \newline\textbf{obs}: is meer een afspraak dan een observatie
    
    \item rq1-53:2-11: crud en CRUD in de \textbf{interface} werken niet altijd zoals het moet. Geen volledige validatie op het gebruik. Dus een aan / uit is niet overal zinvol. 
    \newline rq1-37:3-10: CRUD/crud opties dient ook enige bestudering voordat deze juist toegepast kunnen worden.
    \newline\textbf{obs}: geen (volledige) validatie op het gebruik en mogen gebruiken van crud. 
    het is mogelijk om variaties toe te passen die geen impact hebben.
    
    \item rq1-55:2-11: bij de RULE  is het noodzakelijk om een ROLE met een MAINTAINS toe te voegen, anders werkt het niet
    \newline\textbf{obs}: in het begin is dit niet van zelf sprekend. 
    Pas bij het bestuderen van voorbeelden wordt dit duidelijk
    
    \item rq1-56:7-11: ampersand en state gaat niet zo goed samen, vb timestamp (wat bedoel ik hiermee?)
    \newline\textbf{obs}: ????
    
    \item rq1-58:8-11: per \textbf{interface} max 1 \textbf{TOT} opnemen, anders krijg je data niet opgeslagen.
    \newline\textbf{obs}: binnen een interface meerdere TOT - relaties opgenomen. 
    Gevolg was dat er binnen het prototype geen data meer kon worden toegevoegd. 
    Door de rules.
    
    \item rq1-59:9-11: er blijven wel heel veel meldingen openstaan wanneer niet aan alles voorwaarden wordt voldaan.
    \newline\textbf{obs}: wanneer er gemakzuchtig met de invoer wordt omgegaan, dan komen er steeds meer melding te staan.
    De meldingen worden overigens wel per type samengevoegd.
    Het werkbare scherm wordt steeds kleiner.
    
    \item rq1-63:10-11: ampersand is \textbf{flexibel} bij uitbreiding concepten en relaties. bv een adres opdelen in kleine stukjes is snel mogelijk. Feitelijke adres opmaak staat niet in de wet, maar ergens wordt blijkbaar verwezen naar BRP implementatie.
    \newline\textbf{obs}: Ampersand werkt heel flexibel. Defineer een Concept en relatie en het is gerealiseerd. 
    tweede observatie is dat in het geval van het adres het niet direct duidelijk hoe dit eruit moet zien. 
    Maar daar zijn andere wetten en bronnen voor.
    Is wel even zoeken en aannames doen.
    
    \item rq1-64:10-11: het overnemen van de \textbf{POPULATION} in import xls, lijkt gedrag van de interface te veranderen. 
    \newline\textbf{obs}: nogmaals toetsen !!
    
    \item rq1-65:10-11: DATETIME (\textbf{represent}) veld kan niet worden geconverteerd naar \textbf{excel}. het compile proces loopt hierop vast.
    \newline\textbf{obs}: zie item
    
    \item rq1-66:10-11: XLSX files indeling mede op basis van \textbf{multipliciteit}; 1-n relatie levert een eigen tabblad/sheet op 
    \newline\textbf{obs}: is eigenlijk weerspiegeling van de database structuur waar je op deze wijze inzicht in krijgt.
    
    \item rq1-67:11-11: een toevoeging van datum/tijd lukte toch wel, maar leverde  een melding op. Proces liep wel door, mbv Stef opgelost. Dit kan ik niet meer reproduceren.
    \newline\textbf{obs}: ??
    
    \item rq1-71:14-11/16-11: de \textbf{interface} behoort ook tot het ontwerp en niet enkel tot het prototype. Door verandering in de CRUD wordt het gedrag van de API anders.
    \newline\textbf{obs}: zie item
    
    \item rq1-79:20-11: eenmaal het \textbf{concept} Datum definieren en dan kan het overal binnen de context gebruikt worden, dus in het script.
    Vraag is dan hoe om te gaan met gedeelde Concepten? en het beheer daarop.
    \newline\textbf{obs}: binnen de context wordt een concept hergebruikt.
    Maar wanneer we een concept willen delen met andere contexten. HOe zou dit moeten werken
    
    \item rq1-83:27-11: experiment met HTML-view binnen het \textbf{interface} lukt niet. 
    Documentatie hiervan geeft geen uitsluitsel. 
    ook de voorbeelden zijn niet voldoende 
    \newline\textbf{obs}: krijg dit onderdeel niet aan de praat
    
    \item rq1-86:30-11: \textbf{classify} is een specialisatie van een concept. hier geen ervaring mee opgedaan.
    \newline\textbf{obs}: ??
    
    \item rq1-98:30-12: bij gebruik van \textbf{LINKTO} in de \textbf{INTERFACE} als laatste element in de interface, en de signatuur komt vaker voor dan verschijnt een dropdown naar alle subinterfaces (van gelijke signatuur)
    \newline\textbf{obs}: vreemd/onverwacht gedrag van de LINKTO
    
\end{enumerate}

Working with Ampersand requires an arduous learning curve.
The resources from which to draw are limited because the method is not widely used.
You can use the websites~\footnotemark{} on which Ampersand is offered.
\footnotetext{
	\url{https://ampersandtarski.gitbook.io/documentation/}
	\newline  
	\url{https://github.com/AmpersandTarski}
    \newline
    \url{https://stackoverflow.com}} 
Furthermore, there is not much to be found on the internet about the Ampersand method.
The examples should mainly be taken from the \url{https://github.com/AmpersandTarski/ampersand-models}.
By studying these and searching for a corresponding situation.

Before starting to create the scripts, it is recommended that you at least read the course documentation~\citepNonPub{wedemeijer_l_joosten_smm_michels_garkenbout_jlc_werkboek_ontwerpen_met_bedrijfsregelspdf_nodate} globally.



\subsubsection{Knowledge of Relation Algebra}
\begin{comment}
plaats hier de afgehandelde items.
\end{comment}

\begin{enumerate}
    \item rq1-17 om een \textbf{rule} toe te passen is veel geduld en oefening nodig; vrij steile leercurve; 
    \newline\textbf{obs}: heeft een rule wel direct te maken met relatie Algebra.

    \item rq2-5:2-10: expliciet maken van de \textbf{multipliciteit}.\newline
    UNI P->0-1 H  most\newline
    TOT P->1-* H  least\newline
    INJ H->1 P    één\newline
    SUR H->1-* P  minimaal 1
    \newline\textbf{obs}: het is fijn om dit uitgeschreven te hebben omdat je niet constant scherp hebt hoe dit werkt
    
    \item rq2-11:19-10: een \textbf{relatie} die univalent is, is een functie. Een een functie daar mag maar één ding uitkomen. Beschrijving van UNI is dan ook P ->0-1 H at most (zie 2-5)
    \newline\textbf{obs}: achtergrond info over de werking van relatie algebra
\end{enumerate}

\subsubsection{Environment}
\begin{comment}
plaats hier de afgehandelde items.
\end{comment}
\begin{enumerate}
    \item rq1-5:30-10: de browser houdt data van de \textbf{interface} vast en er moet regelmatig een cache worden geleegd om het nieuwe werkend te krijgen
    \newline\textbf{obs}: het oogt alsof de wijzigingen die in het script gemaakt worden geen invloed hebben op werking. 
    Blijkt dat dit veroorzaakt wordt doordat de cache van de browser niet geleegd wordt.
    Gelukkig zijn dat browser extensies voor om dit alsnog uit te voeren.
    
    \item rq1-13:17-10: inrichting van Ampersand in \textbf{lokale omgeving} is specifiek en niet evident; hulp is hier nodig; pogingen gedaan om zaken in localhost werkend te krijgen. ondanks berichten op de ampersand site levert dit niets op.
    \newline\textbf{obs}: de documentation geeft in een indicatie hoe ampersand locaal in te richten en te laten werken.
    Met oa gebruik maken van xamp. 
    Dit lijkt allemaal niet te werken. Niet duidelijk waarom.
    Krijg het het uiteindelijk met hulp wel werkend in de Docker omgeving.
    
    \item rq1-22 \textbf{\acrshort{vsc}} heeft ook geen generieke zoek optie over de adl's heen
    \newline\textbf{obs}: onhandig bij het zoeken naar gebruik van concepten en relaties of het refactoren.
    Om hergebruik te bevoorderen is vindbaarheid noodzakelijk
    Nu moeten er tools buiten \acrshort{vsc} gebruikt worden, binnen het gebruikte OS, om te zoeken binnen files
    
    \item rq1-32:14-9: \textbf{\acrfull{vsc}} heeft een Ampersand extensie. Deze blijft een enkele keer hangen.
    \newline\textbf{obs}: zal vast wel aan mijn systeem liggen, maar vervelend is het wel
    
    \item rq1-33:9-1: \acrshort{vsc} ondersteunt slecht de \textbf{latex} omgeving. heel vaak blijft mijn pc hierop hangen.  
    \newline rq1-87:3-12: in VS kan ook latex geschreven worden. maar hierdoor hangt mijn pc. is blijkbaar toch een andere versie, want de import wil niet direct lukken. werkt niet echt fijn en resultaat is mager. 
    \newline\textbf{obs}: \acrlong{vsc} ondersteunt middels add-ons ook de TEX omgeving. 
    Maar deze add-on hangt mijn systeem compleet op .
    Ik krijg een 100\% cpu load voor lange tijd. Onwerkbaar.
    
    \item rq1-81:20-11: compilatie fout door een \textbf{include} die niet meer bestond. Observatie hier is dat er een adl van naam gewijzigd of verplaatst of verwijderd is. De tool \acrlong{vsc} ondersteunt niet een refactoring slag bij genoemde wijzigingen. 
    \newline\textbf{obs}: zie item

    \item rq1-88:5-12: \textbf{obsidian} als nieuwe tool geprobeerd. maar ook hier krijg ik niet direct overzicht en is digitaal. Blijkbaar is een logboek voor mij toch handiger
    \newline\textbf{obs}: tijdens het schrijven van de logs ook een nieuwe tool uitgeprobeerd. 
    Of deze werkt niet lekker voor me of ik moet meer geduld oefenen.

\end{enumerate}

\subsubsection{Ampersand on the internet}
\begin{comment}
plaats hier de afgehandelde items.
\end{comment}
\begin{enumerate}
    \item rq1-18 op internet geen voorbeeld te vinden, enkel in de \textbf{repo} van A zelf; en dat is moeizaam zoeken    
    \newline\textbf{obs}: weinig  te vinden over Ampersand behalve in de eigen repo's
    
\end{enumerate}

\subsubsection{Docker}
\begin{comment}
plaats hier de afgehandelde items.
\end{comment}
\begin{enumerate}
    \item rq1-6:21-10/30-10: \textbf{docker} is ook nog een ding om te leren. zou ook nog een introductie cursus moeten zijn om docker gebruik voor Ampersand snel te doorgronden. zonde van de tijd om dit zelf te moeten opzoeken.
    \newline\textbf{obs}: Docker kennis (beperkt) is noodzakelijk
    
\end{enumerate}

\subsubsection{Prototype}
\begin{comment}
plaats hier de afgehandelde items.
\end{comment}
\begin{enumerate}
    \item rq1-1 Indelen in Ampersand (\textbf{patterns}) heeft consequenties voor het \textbf{prototype}. relatie met rq1-22
    \newline\textbf{obs}: welke dan ?? nog even over nadenken
    
    \item rq1-36:3-10: hoe zit het met testscenario's van het prototype
    \newline\textbf{obs}: ogenschijnlijk ontberen wat testtools. mogelijk moet er gebruik worden gemaakt van generieke tools zoals Selenium ed.

\end{enumerate}

\subsubsection{Documentation}
\begin{comment}
plaats hier de afgehandelde items.
\end{comment}
\begin{enumerate}
    \item rq1-10 html href en target blank werkt niet binnen de \textbf{interface}   
    \newline rq1-77:20-11: in de html modus wordt de <a href="x" target=\_blank> niet ondersteunt. 
    De target wordt in de complatie verwijderd.    
    \newline\textbf{obs}: de verwachting was dat de target \_blank een nieuwe tab zou openen in de html tekst, maar dat gebeurt niet.
    
    \item rq1-15:4-10 middels \textbf{include} statements wordt de volgorde van document bepaald, maar niet overal zijn includes nodig
    \newline\textbf{obs}: includes zijn niet enkel om de scripts volledig te laten draaien, maar ook om de documentatie te sturen.
    
    \item rq1-30:12-9:hoe gaan we de meaning en def van \textbf{concepten} vastleggen. 
    is dit conform een vast stramien. 
    is het dan nog leesbaar of moet het vrijer worden gedocumenteerd.
    \newline\textbf{obs}: het vastleggen van de meaning en definitie is erg vrij.
    
    \item rq1-41:19-10: de website van de wettenbank bevat een persistente hyperlink, deze kan gebruikt worden in de \textbf{documentatie} als referentie.
    \newline\textbf{obs}: er kunnen referenties naar persistente links worden opgenomen. 
    Maar is de uitvoer dan nog wel prettig leesbaar
    
    \item rq1-42:19-10: aanwennen om direct bij de opname van een \textbf{concept} en \textbf{relatie} de beschrijving toe te voegen. Anders weet je het later echt niet meer. Dit gebeurt dus ook gewoon.
    \newline\textbf{obs}: zie item
    
    \item rq1-43:23-10: de volgorde van de gegevens in het \textbf{Conceptueel ontwerp} is wat vreemd. Nog even checken, maar eerst wordt de definitie getoond en dan pas de naam van de relatie en daaronder weer de meaning
    \newline\textbf{obs}: zie item
    
    \item rq1-44:23-10: voor het \textbf{conceptuele ontwerp}, houdt rekening met enters in de teksten. Deze komen direct terug in de documenten en leveren dan vreemde opmaak op.
    \newline\textbf{obs}: afbreek enters in de \acrshort{ide} leveren ook extra newlines op in de uitvoer. de opmaak gaat hierdoor fout.
    
    \item rq1-45:24-10: overzicht     binnen een \textbf{script} is lastig te behouden en te verkrijgen.
    \newline\textbf{obs}: zie item
    
    \item rq1-54:2-11: \textbf{documentatie} kan op verschillende wijze worden geschreven. Dit kan mbv mark down, html en latex
    \newline\textbf{obs}: deze had ik ook uit documentatie kunnen halen

    \item rq1-75:20-11: nog wat geexperimenteerd met het \textbf{prototype}. 
    bij het beschrijven van de purpose van de context is het even puzzelen hoe deze tekst goed meegegeven kan worden. 
    een <h1> levert op in H4 een extra hoofstuk op en wordt H4 dan H5. En h5 is dan een niets zeggend stukje geworden. NIet handig dus.
    een <h2> en <h3> werkt het wel goed.
    \newline\textbf{obs}: ingrijpen in structuur kan onverwachte gevolgen hebben.
    
    \item rq1-76:20-11: de "disclaimer" komt niet terug in de \textbf{conceptuele analyse} dus een kopje extra in de inleiding levert een lege kop op in de analyse.
    \newline\textbf{obs}: zie item
    
    \item rq1-78:20-11: \textbf{documenttie} gegenereert in HTML in firefox geladen en er zijn geen png's zichtbaar. Chrome doet het wel goed.
    \newline\textbf{obs}: zie item
    
    \item rq1-99:6-1: bij het genereren van een \textbf{conceptuele analyse} krijgt het doc de naam van het eerste concept
    \newline\textbf{obs}: dit is vreemd gedrag.     
    
    \item rq1-2 Ampersand heeft geen annotatie mogelijkheid -> vergt een separate actie/document om bij te houden wat er geweest is
    \newline\textbf{obs}: komt weer neer op het houden van \textbf{overzicht}, wat zou ik met de annotatie willen=
    
\end{enumerate}

\textbf{Conceptual analysis}
    

\textbf{Other documentation}

\subsubsection{Development environment}
\begin{comment}
plaats hier de afgehandelde items.
\end{comment}
\begin{enumerate}
    \item rq1-36:29-9: bezig geweest om onder w10 te laten draaien. het \textbf{prototype} onder \textbf{localhost} te draaien. niet gelukt; opstarten van de service was niet te doen. Wel docker draaiend gekregen. Zat nog wel een foutje in de installatie documentatie. Blijkt dat het niet de installatie directory RapInstall te zijn, maar RAP te zijn.
    \newline\textbf{obs}: zie item
    
    \item rq1-47:27-10: Het constateren van de fout. Door deze te plaatsen in github-issues bij de \textbf{repo} komt er binnen een dag een reactie en wordt het opgelost. In dit geval was het een fout in Ampersand die snel opgelost werd met een een nieuwe versie.
    \newline\textbf{obs}: zie item
        
    \item rq1-48:27-10: het \textbf{concept} current date is wel heel ingewikkeld opgelost. Maar werkt uiteindelijk wel. Current time lijkt nog niet ontwikkeld. Al lijken de voorbeeld scripts toch iets anders te zeggen.
    \newline\textbf{obs}: zie item
    
    \item rq1-49:30-10: het isoleren van een \textbf{subsysteem} tbv het testen wil niet werken. Dit heeft te maken met het inrichten van docker en mogelijk onkunde van mijn kant.
    \newline\textbf{obs}: zie item    
    
\end{enumerate}

\subsubsection{Reading law}
\begin{comment}
Hoort dit niet bij rq3?
plaats hier de afgehandelde items.
\end{comment}
\begin{enumerate}
    \item rq1-23:24-10: \textbf{wet} lezen is een vak 
    \newline\textbf{obs}: zie item
    
    \item rq1-25:12-9: eerst \textbf{overzicht} maken van de alle wetten en regelingen
    \newline\textbf{obs}: scoping belangrijk
    
    \item rq1-26:12-9: ook de wetten en de regelgevingen kunnen nog weer \textbf{verwijzigen} hebben naar andere wetten en regelgevingen. Omdat ze daarop gebaseerd kunnen zijn of deze uitbreiden.
    \newline\textbf{obs}: dat is waar.
    
    \item rq1-27:12-9: er zijn ook wetten en regels die niet opgenomen zijn in deze specifieke wet, maar geldig zijn vanuit een hoger liggende wet (implicite \textbf{verwijzingen}). 
    Denk hierbij aan de grondwet. 
    In geval van big zou dit bv de archiefwet kunnen zijn of de termijnenwet en strafrecht.
    \newline\textbf{obs}: hoe komt je tot een volledig overzicht van wetten en regelgeving
    
    \item rq1-28:12-9: regelgeving over een afgesloten en verleden, wordt niet meer meegenomen. 
    Deze is niet meer geldig. 
    Doel is niet om historie op te bouwen maar de huidige wet te ondersteunen.
    \newline\textbf{obs}: geen observatie maar meer scoping
    
    \item rq1-29:12-9: Niet alle wet- en regelgeving mbv BIG vinden we terug onder de \textbf{zoekterm} big.
    \newline\textbf{obs}: je moet er attend op zijn dat er meer is dan enkel BIG
    
    \item rq1-42:21-10: het is makkelijk om van de wetsteksten af te wijken. 
    Juist omdat deze zo lastig te lezen zijn. 
    En enigszins \textbf{kennis van de wet }cq het proces maakt dat eigen interpretatie snel gemaakt is. 
    Action research maakt ook dat je snel in deze valkuil stapt.
    \newline\textbf{obs}: zie item
    
    \item rq1-61:9-11: er zou een check op geb datum moeten komen(\textbf{rule}), dat iemand minimaal 18 moet zijn. 
    Klinkt logisch, maar is een afgeleide regel. 
    dus niet opnemen. 
    Deze zit al impliciet in de opleidingseis. 
    De duur van de opleiding maakt al dat iemand minimaal 18 jaar is voor dat de opleiding is afgerond.
    \newline\textbf{obs}: zie item
    
    \item rq1-95:29-12: de opmaak van een \textbf{concept} bignummer  is niet de wet opgenomen
    \newline\textbf{obs}: zouden er eisen zijn aan het bignummer nu is het volgens mij 8 cijfers, maar zou dit ook zo moeten zijn. 
    Een guid is misschien vanuit gebruikersvriendleijkheid niet zo handig
\end{enumerate}

\subsubsection{Approach}
\begin{comment}
plaats hier de afgehandelde items.
\end{comment}
\begin{enumerate}
    \item rq1-3 aparte excel gemaakt om de \textbf{multiplicteiten} van de relaties uit te schrijven en te ontdekken
    \newline\textbf{obs}: om een werkwijze te ontdekken is dit een goede manier. 
    Wel gemerkt dat het lastig is (vanuit beheersperspectief) om het excel in sync te houden met de scripts

    \item rq1-24 12-9/14-9 \textbf{XML} download van wetBig lijkt een logische stap, echter is deze zo complex opgebouwd dat dit geen zin heeft (idem voor json) => beide structuren zijn niet prettig leesbaar. Gedachte die hier op komt is waarom SDU niet direct annotaties toevoegt op de concepten en relaties. Beide structuren zijn gedownload om hier de annotaties aan toe te voegen. zodat je later via een programma de annotatie (xml/json annotaties) eruit kan halen tesamen met de definities en meaning en hiermee een script of een deel van een script mee kan genereren.
    \newline\textbf{obs}: zie item
    
    \item rq1-31:14-9: naast \textbf{xml} en \textbf{json} is ook \textbf{rtf en pdf} een optie. In rtf (doc) kan er via "opmerkingen" items in de marges worden toegevoegd. BIj een pdf kunnen annotaties en kleurmarkering deze functie krijgen.
    \newline\textbf{obs}: tbv het overzicht en om bij te houden waar je geweest bent
    
    \item rq1-33:14-9: \textbf{Patterns} binnen Ampersand zijn belangrijk. Dit zijn in feite de \textbf{subsystemen} van het informatie systeem. Vraag is even of dit op voorhand ingedeeld zou moeten worden of dat het zich vanzelf opbouwt.
    \newline\textbf{obs}: zie item
    
    \item rq1-34:14-9: schrijfwijze van een \textbf{pattern} is met een hoofdletten. En de pattern wordt afgesloten met een endpattern. Er zijn meerdere patterns mogelijk binnen één script.
    \newline\textbf{obs}: staat vast in de documentatie, maar daar lees je overheen en moet je even overkomen
    
    \item rq1-35:14-9: de wet is in het nederlands opgesteld en dat maakt dat de \textbf{conceptuele analyse} ook in het nederlands te doen.
    \newline\textbf{obs}: zie item
    
    \item rq1-38:3-10: moeten de \textbf{subsystemen} op voorhand worden ingedeeld. Dit wordt gestuurd middels \textbf{patterns}. 
    \newline\textbf{obs}: niet per definitie, maar misschien moet je wel nadenken over de indeling. 
    Maar op welk moment?
    
    \item rq1-39:3-10: er moet niet vergeten worden om naast de toevoeg en mutatie regels ook verwijder regels te maken. \textbf{Lifecycle} aanpak.
    \newline\textbf{obs}: oja effect.
    
    \item rq1-46:24-10: in de scripts is geen relatie tussen de \textbf{relaties} en \textbf{concepten}. Het \textbf{overzicht} is lastig te verkrijgen. Veel zoeken en de IDE helpt niet heel erg mee.
    \newline\textbf{obs}: zie item
    
    \item rq1-51:2-11: het bespreken van het \textbf{conceptuele analyse} moet thema voor thema worden gedaan.
    \newline\textbf{obs}: thema is gelijk aan pattern
    
    \item rq1-57:7-11: ampersand gebruiken voor \textbf{validaties} en \textbf{relaties}. 
    zaken als volgende big-nummer of now() en vandaag() is beter op te lossen in een dev-taal.
    \newline\textbf{obs}: zie item
    
    \item rt1-60:9-11: het proces is niet ongeoefend \textbf{uit te voeren}, er moet altijd wel iemand met ervaring op de achtergrond of in de samenwerking zijn.
    \newline\textbf{obs}: zie item
    
    \item rq1-62:9-11: JK stelt dat \textbf{regelanalyse} onderdeel zou moeten uitmaken van het eindresultaat. 
    navragen wat die regelanalyse eigenlijk is. ben ik blijkbaar even kwijt
    \newline\textbf{obs}: ??
    
    \item rq1-62:10-11: er zou een koppeling moeten zijn tussen de \textbf{wettenkern} en de \textbf{registerkern}
    \newline\textbf{obs}: ampersand analyse moet passen in de architectuur en de manier van werken
    
    \item rq1-67:11-11: als er een automatische \textbf{rule} is, moet er dan toch een validatie rule op?
    \newline\textbf{obs}: wel als er kans is dat de automatische rule (per ongeluk) wordt verwijderd of veranderd.
    Meer intrinsieke controle middel
    
    \item rq1-68:14-11: door het neerzetten van het \textbf{prototype} en de kennis van de omgeving is het mogelijk om een schets te maken van de architectuur waar het in zou moeten passen.
    \newline\textbf{obs}: zie item
    
    \item rq1-69:14-11: postman geinstalleerd en werkt met het \textbf{prototype}.
    \newline\textbf{obs}: nvt
    
    \item rq1-70:14-11: postman werkt mbv api/v1/resource, bv GET localhost/api/v1/resource/Persoon/P001/Persoon; 
    hiermee wordt een bestaand persoon opgehaald. 
    Dus kan er van buiten Ampersand gebruik gemaakt worden van de validatie structuur van ampersand middels \textbf{API}.
    \newline\textbf{obs}: dus ampersand is meer open dan het op het eerste gezicht lijkt
    
    \item rq1-72:14-11: naast de GET(ophalen), werkt ook de POST(toevoegen) en PUT (muteren)
    \newline\textbf{obs}: uitwerking van de \textbf{API} mogelijkheden
    
    \item rq1-73:14-11: ampersand is te gebruiken vanuit andere applicaties middels \textbf{API}'s, maar de return waardes zijn naast de gevraagde infomatie ook meldingen (en geen meldingcodes). Deze codes zouden in de meldingen opgenomen kunnen worden, maar blijven "ongestructureerde" data.
    \newline\textbf{obs}: hier mis je structuur. 
    Dus Ampersand is niet bedoelt om zo te gebruiken. 
    Zie ook opmerking over swagger(rq1-8).
    
    \item rq1-74:16-11: koppeling tussen front-end en back-end (ampersand-\textbf{api}) kan wijziging in de back-end detecteren. Als Ampersand wijzigt dan moet de front end waarschijnlijk mee wijzigen. anders doet ie het niet meer.
    \newline\textbf{obs}: zie item
    
    \item rq1-80:20-11: let op een consequente benoeming van \textbf{Concepten}. 
    Een eenmaal gedefineerd concept zou zomaar (door gebrek aan het overizcht) opnieuw gedefinieerd kunnen worden. 
    Met net een ander formaat of definitie. 
    Elke analyse tool heeft dat natuurlijk, maar je zou dit opgelost willen zien.
    \newline\textbf{obs}: zie item
    
    \item rq1-82:20-11: \textbf{includes} lijken niet altijd nodig bij compilatie. niet helemaal duidelijk wanneer dit nu wel of niet nodig is. 
    andere functie van includes is het opmaken van de analyse.
    \newline\textbf{obs}: includes werking nog even testen
    
    \item rq1-84:30-11: \textbf{Concepten} zijn onveranderlijk. bv persoon is concept, arts niet. 
    moet intrinsieke eigenschap zijn, die niet wijzigbaar is.
    \newline\textbf{obs}: belangrijk concept van Ampersand. 
    
    \item rq1-85:30-11: nieuwe structuur waarbij de \textbf{registers} onafhankelijk van elkaar kunnen opereren, met enkel de generieke elementen als multiple use zaken.
    \newline\textbf{obs}: dit werkt niet ivm multicontext issue
    
    \item rq1-89:7-12: zaken benoemd als common \textbf{concepts}, EA bevat waarschijnlijk wel tekeningen en beschrijvingen van de huidige situatie.
    \newline\textbf{obs}: hoe om te gaan met common concepts
    
    \item rq1-90:14-12: verzameling model van regelgeving; dan \textbf{includes} -> klein houden en overzichtelijk; tbv herbruikbaarheid van de adl.
    per feature een module;
    \newline\textbf{obs}: zie item
    
    \item rq1-91:14-12: \textbf{concepten} en \textbf{relaties} kunnen meerdere malen gedefinieerd worden binnen de eigen patterns; zodat de patterns op zichzelf kunnen staan.
    \newline\textbf{obs}: gevaarlijk want hiermee kunnen dezelfde concepten verschillende definities bevatten
    
    \item rq1-92:14-12: in principe proberen om per \textbf{register} een eigen container te maken. Multicontext problematiek. 
    hierdoor lukt het niet om deze containers te isoleren. 
    \newline\textbf{obs}: zie item
    
    \item rq1-93:19-12/21-12: implementatie keuze voor seperate \textbf{registers} heeft impact op het geheel.
    hoe om te gaan met gedeelde modules
    hoe om te gaan met gedeelde data (zoals persoon)
    of moet de keuze gemaakt worden om enkel de concepten en relaties te delen en niet implemetatie
    oplossing zou kunnen zijn om elk register van een eigen db te voorzien en van een gedeelde db voor bv personen
    gebruik van ports is dan ook een issue. Kan iets voor geregeld worden in de .env.
    uitwerking van de eigen containers lijkt niet te werken, steeds wordt de db structuur overschreven door het nieuwe register
    \newline\textbf{obs}: zie item
    
    \item rq1-94:29-12: er valt niets te generaliseren; eigen containers werken niet
    \newline\textbf{obs}:  zie rq1-93
    
    \item rq1-96:30-12: \textbf{vaardigheid} in de scripting ben je snel kwijt wanneer je dit niet frequent doet
    \newline\textbf{obs}: eigen waarneming
    
    \item rq1-97:30-12: door gepuzzel met ampersand wordt snel vergeten om correct te \textbf{documenteren}. vaak ben je blij dat iets werkt.
    \newline\textbf{obs}: zeker een gevaar.
    
\end{enumerate}



