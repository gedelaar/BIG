\begin{comment}
\textbf{How useful is Ampersand for designing registry systems by analysing public health legislation and regulations, in particular the \acrshort{big}.}

When investigating the research question, the following sub-questions will contribute to the answer to the research question.
\newline Related questions:
\begin{enumerate}
\item[RQ1]- What knowledge, in the role of software engineer, is needed to use Ampersand.
\item[RQ2]- What are the Concepts, Relationships and Rules in the \acrshort{big}.
\item[RQ3]- How are the laws and regulations set up so that they can be used in a useful way for the Ampersand method.
\item[RQ4]- What are the strengths and weaknesses (SWOT) in using Ampersand for registry systems for a government organization.
\end{enumerate}

Hoe nuttig is Ampersand voor het ontwerpen van registratiesystemen door analyse van wet- en regelgeving op het gebied van volksgezondheid, in het bijzonder de Wet-BIG.

Bij het onderzoeken van de onderzoeksvraag zullen de volgende deelvragen bijdragen aan het beantwoorden van de onderzoeksvraag. Gerelateerde vragen:
RQ1 - Welke kennis, in de rol van software engineer, is nodig om Ampersand te gebruiken.
RQ2 - Wat zijn de concepten, relaties en regels in de Wet-BIG.
RQ3 - Hoe zijn de wet- en regelgeving opgezet zodat ze kunnen worden gebruikt in een bruikbare manier voor de Ampersand-methode.
RQ4 - Wat zijn de sterke en zwakke punten (SWOT) bij het gebruik van Ampersand voor registratiesystemen voor een overheidsorganisatie.

\end{comment}

\subsection{Use of Ampersand}\label{use_of_ampersand}
In the search for the question "\acrlong{rq1}" we found the following perspectives.
When asking questions about the knowledge needed in the role of software engineers to be able to use Ampersand, the focus is not only on the Ampersand method itself.
We should also ask ourselves whether, in addition to Ampersand, knowledge is also required of the underlying theory on which Ampersand is based.
Is it necessary to know relation algebra.
The environment in which a prototype runs also plays a role.
This includes browser settings and the use of containers (Docker) in which Ampersand runs.
During development we are dealing with an \acrfull{ide}.
The use of such a \acrshort{ide} for determining the usability of Ampersand is relevant.
Internet support is now indispensable when creating the scripts.
Creating scripts for Ampersand also requires support for internet search capabilities.
But can this be found for a relatively unknown method as Ampersand?
Finally, we look at output produced by method.
Is it the case that the way in which the method is used also determines what comes out?
This applies to both the prototype and the generated documentation.
Then the approach to this process, the use of the method, among other things, determines the result.
These aspects will be discussed in the following paragraphs.

\subsubsection{Knowledge of Ampersand}
\begin{comment}
- wat zijn de exacte waarnemingen geweest!
plaats hier de afgehandelde items.
\end{comment}
\begin{enumerate}
    \item rq1-4 automatische rules zijn beschreven, maar om te implementeren is ook hier veel geduld en proberen nodig. 
    \item rq1-7:10-11: elke relatie is onderdeel van een record structuur
    \item rq1-8:14-11: er is geen swagger gemaakt voor de api; 
    \item rq1-9 het toevoegen van stukjes php script moet mogelijk zijn, maar is niet duidelijk hoe
    \item rq1-11 implementatie in docker met RAP werkt wel, maar niet met includes om dat er steeds een nieuwe directory wordt aangemaakt; pas waneer er echt lokaal wordt gedraaid dan werkt het ook met includes
    \item rq1-12 in het begin niet duidelijk wanneer C of c in de INTERFACE toe te passen; mogelijk staat dit wel in de handleiding, maar moet je toch proefondervindelijk ontdekken.
    \item rq1-14:10-11: kiss
    \item rq1-16 notatie wijze van concepten en relaties en rules zijn deels vastgelegd. Enkel de eerste positie is hoofd of kleine letter; geen advies over overige schrijfwijze.
    \item rq1-17 om een rule toe te passen is veel geduld en oefening nodig; vrij steile leercurve; 
    \item rq1-19 output in latex 
    \item rq1-21:7-11: TOT wordt meestal ondervangen door een tot-rule -> blijkt dat een TOT tot gevolg heeft dat iets kan worden gesaved wanneer ingevuld, terwijl een tot-rule een save kan plaatsvinden terwijl de melding open blijft staan
    \item rq1-40:10-10/27-11: Interface concepten moeten van het type object zijn (represent). waarom kunnen hier geen alpha of integer zijn.
    represent definieert een type van een concept, maar datetime geeft problemen bij de interface
    \item rq1-50:30-10: het represent statement laat de interface anders reageren. Bij gebruik hiervan verdwijnt de "+"
    \item rq1-52:2-11: als een onderdeel niet werkt, of niet werkt zoals je wenst, dan weglaten uit het ontwerp
    \item rq1-53:2-11: crud en CRUD werken niet altijd zoals het moet. Geen volledige validatie op het gebruik. Dus een aan / uit is niet overal zinvol. 
    \item rq1-55:2-11: bij de RULE  is het noodzakelijk om een ROLE met een MAINTAINS toe te voegen, anders werkt het niet
    \item rq1-56:7-11: ampersand en state gaat niet zo goed samen, vb timestamp (wat bedoel ik hiermee?)
    \item rq1-58:8-11: per interface max 1 TOT opnemen, anders krijg je data niet opgeslagen.
    \item rq1-59:9-11: er blijven wel heel veel meldingen openstaan wanneer niet aan alles voorwaarden wordt voldaan.
    \item rq1-63:10-11: ampersand is flexibel bij uitbreiding concepten en relaties. bv een adres opdelen in kleine stukjes is snel mogelijk. Feitelijke adres opmaak staat niet in de wet, maar ergens wordt blijkbaar verwezen naar BRP implementatie.
    \item rq1-64:10-11: het overnemen van de POPULATION in import xls, lijkt gedrag van de interface te veranderen. 
    \item rq1-65:10-11: DATETIME (represent) veld kan niet worden geconverteerd naar excel. het compile proces loopt hierop vast.
    \item rq1-66:10-11: XLSX files indeling mede op basis van multipliciteit; 1-n relatie levert een eigen tabblad/sheet op 
    \item rq1-67:11-11: een toevoeging van datum/tijd lukte toch wel, maar leverde  een melding op. Proces liep wel door, mbv Stef opgelost. Dit kan ik niet meer reproduceren.
    \item rq1-71:14-11/16-11: de interface behoort ook tot het ontwerp en niet enkel tot het prototype. Door verandering in de CRUD wordt het gedrag van de API anders.
    \item rq1-79:20-11: een maal het concept Datum definieren en dan kan het overal binnen de context gebruikt worden, dus in het script.
    Vraag is dan hoe om te gaan met gedeelde Concepten? en het beheer daarop.
    \item rq1-83:27-11: experiment met HTML-view lukt niet. 
    Documentatie hiervan geeft geen uitsluitsel. 
    ook de voorbeelden zijn niet voldoende 
    \item rq1-86:30-11: classify is een specialisatie van een concept. hier geen ervaring mee opgedaan.
    \item rq1-98:30-12: bij gebruik van LINKTO in de INTERFACE als laatste element in de interface, en de signatuur komt vaker voor dan verschijnt een dropdown naar alle subinterfaces (van gelijke signatuur)
\end{enumerate}

Working with Ampersand requires an arduous learning curve.
The resources from which to draw are limited because the method is not widely used.
You can use the websites~\footnotemark{} on which Ampersand is offered.
\footnotetext{
	\url{https://ampersandtarski.gitbook.io/documentation/}
	\newline  
	\url{https://github.com/AmpersandTarski}
    \newline
    \url{https://stackoverflow.com}} 
Furthermore, there is not much to be found on the internet about the Ampersand method.
The examples should mainly be taken from the \url{https://github.com/AmpersandTarski/ampersand-models}.
By studying these and searching for a corresponding situation.

Before starting to create the scripts, it is recommended that you at least read the course documentation~\citepNonPub{wedemeijer_l_joosten_smm_michels_garkenbout_jlc_werkboek_ontwerpen_met_bedrijfsregelspdf_nodate} globally.



\subsubsection{Knowledge of Relational Algebra}
\begin{comment}
plaats hier de afgehandelde items.
\end{comment}

\begin{enumerate}
    \item rq1-17 om een rule toe te passen is veel geduld en oefening nodig; vrij steile leercurve; 
\end{enumerate}

\subsubsection{Environment}
\begin{comment}
plaats hier de afgehandelde items.
\end{comment}
\begin{enumerate}
    \item rq1-5:30-10: de browser houdt data vast en er moet regelmatig een cache worden geleegd om het nieuwe werkend te krijgen
    \item rq1-13:17-10: inrichting van A in lokale omgeving is specifiek en niet evident; hulp is hier nodig; pogingen gedaan om zaken in localhost werkend te krijgen. ondanks berichten op de ampersand site levert dit niets op.
    \item rq1-22 VS heeft ook geen generieke zoek optie over de adl's heen
    \item rq1-32:14-9: Visual Studio Code(VSC) heeft een Ampersand extensie. Deze blijft een enkele keer hangen.
    \item rq1-33:9-1: VSC ondersteunt slecht de latex omgeving. heel vaak blijft mijn pc hierop hangen.  
    \item rq1-81:20-11: compilatie fout door een include die niet meer bestond. Observatie hier is dat er een adl van naam gewijzigd of verplaatst of verwijderd is. De tool (VSC) ondersteunt niet een refactoring slag bij genoemde wijzigingen. 
    \item rq1-87:3-12: in VS kan ook latex geschreven worden. maar hierdoor hangt mijn pc. is blijkbaar toch een andere versie, want de import wil niet direct lukken. werkt niet echt fijn en resultaat is mager. 
    \item rq1-88:5-12: obsidian als nieuwe tool geprobeerd. maar ook hier krijg ik niet direct overzicht en is digitaal. Blijkbaar is een logboek voor mij toch handiger
    \item rq3-1 bij interface kan je een FOR toevoegen voor autorisatie
    \newline\textbf{obs}:

\end{enumerate}

\subsubsection{Ampersand on the internet}
\begin{comment}
plaats hier de afgehandelde items.
\end{comment}
\begin{enumerate}
    \item rq1-18 op internet geen voorbeeld te vinden, enkel in de repo van A zelf; en dat is moeizaam zoeken    
\end{enumerate}

\subsubsection{Docker}
\begin{comment}
plaats hier de afgehandelde items.
\end{comment}
\begin{enumerate}
    \item rq1-6:21-10/30-10: docker is ook nog een ding om te leren. zou ook nog een introductie cursus moeten zijn om docker gebruik voor Ampersand snel te doorgronden. zonde van de tijd om dit zelf te moeten opzoeken.

\end{enumerate}

\subsubsection{Prototype}
\begin{comment}
plaats hier de afgehandelde items.
\end{comment}
\begin{enumerate}
    \item rq1-1 Indelen in Ampersand (patterns) heeft consequenties voor het prototype. VS ontbeert een refactoring optie bij verplaatsen
    \item rq1-36:3-10: hoe zit het met testscenario's van het prototype
    \item rq1-37:3-10: CRUD/crud opties dient ook enige bestudering voordat deze juist toegepast kunnen worden.
\end{enumerate}

\subsubsection{Documentation}
\begin{comment}
plaats hier de afgehandelde items.
\end{comment}
\begin{enumerate}
    \item rq1-10 html href en target blank werkt niet    
    \item rq1-15:4-10 middels include statements wordt de volgorde van document bepaald, maar niet overal zijn includes nodig
    \item rq1-19 output in latex 
    \item rq1-30:12-9:hoe gaan we de meaning en def van concepten vastleggen. is dit conform een vast stramien. is het dan nog leesbaar of moet het vrijer worden gedocumenteerd.
    \item rq1-41:19-10: de website van de wettenbank bevat een persistente hyperlink, deze kan gebruikt worden in de documentatie als referentie.
    \item rq1-42:19-10: aanwennen om direct bij de opname van een concept en relatie de beschrijving toe te voegen. Anders weet je het later echt niet meer. Dit gebeurt dus ook gewoon.
    \item rq1-43:23-10: de volgorde van de gegevens in het Con. ontw. is wat vreemd. Nog even checken, maar eerst wordt de definitie getoond en dan pas de naam van de relatie en daaronder weer de meaning
    \item rq1-44:23-10: voor het conceptuele ontwerp, houdt rekening met enters in de teksten. Deze komen direct terug in de documenten en leveren dan vreemde opmaak op.
    \item rq1-45:24-10: overzicht binnen een script is lastig te behouden en te verkrijgen.
    \item rq1-54:2-11: documentatie kan op verschillende wijze worden geschreven. Dit kan mbv mark down, html en latex
    \item rq1-75:20-11: nog wat geexperimenteerd met het prototype. 
    bij het beschrijven van de purpose van de context is het even puzzelen hoe deze tekst goed meegegeven kan worden. 
    een <h1> levert op in H4 een extra hoofstuk op en wordt H4 dan H5. En h5 is dan een niets zeggend stukje geworden. NIet handig dus.
    een <h2> en <h3> werkt het wel goed.
    \item rq1-76:20-11: de "disclaimer" komt niet terug in de conceptuele analyse dus een kopje extra in de inleiding levert een lege kop op in de analyse.
    \item rq1-77:20-11: in de html modus wordt de <a href="x" target=\_blank> niet ondersteunt. De target wordt in de complatie verwijderd.
    \item rq1-78:20-11: doc gegenereert in HTML in firefox geladen en er zijn geen png's zichtbaar. Chrome doet het wel goed.
    \item rq1-99:6-1: bij het genereren van een conceptuele analyse krijgt het doc de naam van het eerste concept
\end{enumerate}

\textbf{Conceptual analysis}
    

\textbf{Other documentation}

\subsubsection{Development environment}
\begin{comment}
plaats hier de afgehandelde items.
\end{comment}
\begin{enumerate}
    \item rq1-36:29-9: bezig geweest om onder w10 te laten draaien. het prototype onder localhost te draaien. niet gelukt; opstarten van de service was niet te doen. Wel docker draaiend gekregen. Zat nog wel een foutje in de installatie documentatie. Blijkt dat het niet de installatie directory RapInstall te zijn, maar RAP te zijn.
    \item rq1-47:27-10: Het constateren van de fout. Door deze te plaatsen in github-issues komt er binnen een dag een reactie en wordt het opgelost. In dit geval was het een fout in Ampersand die snel opgelost werd met een een nieuwe versie.
    \item rq1-48:27-10: current date is wel heel ingewikkeld opgelost. Maar werkt uiteindelijk wel. Current time lijkt nog niet ontwikkeld. Al lijken de voorbeeld scripts toch iets anders te zeggen.
    \item rq1-49:30-10: het isoleren van een subsysteem tbv het testen wil niet werken. Dit heeft te maken met het inrichten van docker en mogelijk onkunde van mijn kant.
    
\end{enumerate}

\subsubsection{Reading law}
\begin{comment}
Hoort dit niet bij rq3?
plaats hier de afgehandelde items.
\end{comment}
\begin{enumerate}
    \item rq1-23:24-10: wet lezen is een vak 
    \item rq1-25:12-9: eerst overzicht maken van de alle wetten en regelingen
    \item rq1-26:12-9: ook de wetten en de regelgevingen kunnen nog weer verwijzigen hebben naar andere wetten en regelgevingen. Omdat ze daarop gebaseerd kunnen zijn of deze uitbreiden.
    \item rq1-27:12-9: er zijn ook wetten en regels die niet opgenomen zijn in deze specifieke wet, maar geldig zijn vanuit een hoger liggende wet. 
    Denk hierbij aan de grondwet. 
    In geval van big zou dit bv de archiefwet kunnen zijn of de termijnenwet en strafrecht.
    \item rq1-28:12-9: regelgeving over een afgesloten en verleden, wordt niet meer meegenomen. 
    Deze is niet meer geldig. 
    Doel is niet om historie op te bouwen maar de huidige wet te ondersteunen.
    \item rq1-29:12-9: Niet alle wet- en regelgeving mbv BIG vinden we terug onder de zoekterm big.
    \item rq1-42:21-10: het is makkelijk om van de wetsteksten af te wijken. Juist omdat deze zo lastig te lezen zijn. En enigszins kennis van de wet cq het proces maakt dat eigen interpretatie snel gemaakt is. Action research maakt ook dat je snel in deze valkuil stapt.
    \item rq1-61:9-11: er zou een check op geb datum moeten komen, dat iemand minimaal 18 moet zijn. Klinkt logisch, maar is een afgeleide regel. dus niet opnemen. Deze zit al impliciet in de opleidingseis. De duur van de opleiding maakt al dat iemand minimaal 18 jaar is voor dat de opleiding is afgerond.
    \item rq1-95:29-12: de opmaak van een bignummer is niet de wet opgenomen
\end{enumerate}

\subsubsection{Approach}
\begin{comment}
plaats hier de afgehandelde items.
\end{comment}
\begin{enumerate}
    \item rq1-2 Ampersand heeft geen annotatie mogelijkheid -> vergt een separate actie/document om bij te houden wat er geweest is
    \item rq1-3 aparte excel gemaakt om de multiplicteiten van de relaties uit te schrijven en te ontdekken
    \item rq1-24 12-9/14-9 XML download van wetBig lijkt een logische stap, echter is deze zo complex opgebouwd dat dit geen zin heeft (idem voor json) => beide structuren zijn niet prettig leesbaar. Gedachte die hier op komt is waarom SDU niet direct annotaties toevoegt op de concepten en relaties. Beide structuren zijn gedownload om hier de annotaties aan toe te voegen. zodat je later via een programma de annotatie (xml/json annotaties) eruit kan halen tesamen met de definities en meaning en hiermee een script of een deel van een script mee kan genereren.
    \item rq1-31:14-9: naast xml en json is ook rtf en pdf een optie. In rtf (doc) kan er via "opmerkingen" items in de marges worden toegevoegd. BIj een pdf kunnen annotaties en kleurmarkering deze functie krijgen.
    \item rq1-33:14-9: Patterns binnen Ampersand zijn belangrijk. Dit zijn in feite de subsystemen van het informatie systeem. Vraag is even of dit op voorhand ingedeeld zou moeten worden of dat het zich vanzelf opbouwt.
    \item rq1-34:14-9: schrijfwijze van een pattern is met een hoofdletten. En de pattern wordt afgesloten met een endpattern. Er zijn meerdere patterns mogelijk binnen één script.
    \item rq1-35:14-9: de wet is in het nederlands opgesteld en dat maakt dat de conceptuele analyse ook in het nederlands te doen.
    \item rq1-38:3-10: moeten de subsystemen op voorhand worden ingedeeld. Dit wordt gestuurd middels patterns. 
    \item rq1-39:3-10: er moet niet vergeten worden om naast de toevoeg en mutatie regels ook verwijder regels te maken. Lifecycle aanpak.
    \item rq1-46:24-10: in de scripts is geen relatie tussen de relaties en concepten. Het overzicht is lastig te verkrijgen. Veel zoeken en de IDE helpt niet heel erg mee.
    \item rq1-51:2-11: het bespreken van het conceptuele analyse moet thema voor thema worden gedaan.
    \item rq1-57:7-11: ampersand gebruiken voor validaties en relaties. zaken als volgende big-nummer of now() en vandaag() is beter op te lossen in een dev-taal.
    \item rt1-60:9-11: het proces is niet ongeoefend uit te voeren, er moet altijd wel iemand met ervaring op de achtergrond of in de samenwerking zijn.
    \item rq1-62:9-11: JK stelt dat regelanalyse onderdeel zou moeten uitmaken van het eindresultaat. navragen wat die regelanalyse eigenlijk is. ben ik blijkbaar even kwijt
    \item rq1-62:10-11: er zou een koppeling moeten zijn tussen de wettenkern en de registerkern
    \item rq1-67:11-11: als er een automatische rule is, moet er dan toch een validatie rule op?
    \item rq1-68:14-11: door het neerzetten van het prototype en de kennis van de omgeving is het mogelijk om een schets te maken van de architectuur waar het in zou moeten passen.
    \item rq1-69:14-11: postman geinstalleerd en werkt met het prototype.
    \item rq1-70:14-11: postman werkt mbv api/v1/resource, bv GET localhost/api/v1/resource/Persoon/P001/Persoon; 
    hiermee wordt een bestaand persoon opgehaald. 
    Dus kan er van buiten Ampersand gebruik gemaakt worden van de validatie structuur van ampersand.
    \item rq1-72:14-11: naast de GET(ophalen), werkt ook de POST(toevoegen) en PUT (muteren)
    \item rq1-73:14-11: ampersand is te gebruiken vanuit andere applicaties middels API's, maar de return waardes zijn naast de gevraagde infomatie ook meldingen (en geen meldingcodes). Deze codes zouden in de meldingen opgenomen kunnen worden, maar blijven "ongestructureerde" data.
    \item rq1-74:16-11: koppeling tussen front-end en back-end (ampersand) kan wijziging in de back-end detecteren. Als Ampersand wijzigt dan moet de front end waarschijnlijk mee wijzigen. anders doet ie het niet meer.
    \item rq1-80:20-11: let op een consequente benoeming van Concepten. Een eenmaal gedefineerd concept zou zomaar (door gebrek aan het overizcht) opnieuw gedefinieerd kunnen worden. Met net een ander formaat of definitie. Elke analyse tool heeft dat natuurlijk, maar je zou dit opgelost willen zien.
    \item rq1-82:20-11: includes lijken niet altijd nodig bij compilatie. niet helemaal duidelijk wanneer dit nu wel of niet nodig is. 
    andere functie van includes is het opmaken van de analyse.
    \item rq1-84:30-11: Concepten zijn onveranderlijk. bv persoon is concept, arts niet. 
    moet intrinsieke eigenschap zijn, die niet wijzigbaar is.
    \item rq1-85:30-11: nieuwe structuur waarbij de registers onafhankelijk van elkaar kunnen opereren, met enkel de generieke elementen als multiple use zaken.
    \item rq1-89:7-12: zaken benoemd als common concepts, EA bevat waarschijnlijk wel tekeningen en beschrijvingen van de huidige situatie.
    \item rq1-90:14-12: verzameling model van regelgeving; dan includes -> klein houden en overzichtelijk; tbv herbruikbaarheid van de adl.
    per feature een module;
    \item rq1-91:14-12: concepten en relaties kunnen meerdere malen gedefinieerd worden binnen de eigen patterns; zodat de patterns op zichzelf kunnen staan.
    \item rq1-92:14-12: in principe proberen om per register een eigen container te maken. Multicontext problematiek. 
    hierdoor lukt het niet om deze containers te isoleren. 
    \item rq1-93:19-12/21-12: implementatie keuze voor seperate registers heeft impact op het geheel.
    hoe om te gaan met gedeelde modules
    hoe om te gaan met gedeelde data (zoals persoon)
    of moet de keuze gemaakt worden om enkel de concepten en relaties te delen en niet implemetatie
    oplossing zou kunnen zijn om elk register van een eigen db te voorzien en van een gedeelde db voor bv personen
    gebruik van ports is dan ook een issue. Kan iets voor geregeld worden in de .env.
    uitwerking van de eigen containers lijkt niet te werken, steeds wordt de db structuur overschreven door het nieuwe register
    \item rq1-94:29-12: er valt niets te generaliseren; eigen containers werken niet
    \item rq1-96:30-12: vaardigheid in de scripting ben je snel kwijt wanneer je dit niet frequent doet
    \item rq1-97:30-12: door gepuzzel met ampersand wordt snel vergeten om correct te documenteren. vaak ben je blij dat iets werkt.
\end{enumerate}

