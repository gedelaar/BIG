\subsection{Ampersand as tool} \label{Ampersand as tool}
\def\cat{6}
To view Ampersand as a tool, we look more closely at the technical functioning of Ampersand.

\begin{table}[H]
    \begin{tabularx}{\linewidth}{|X|X|}
        \hline
        Category        & CAT1-\cat \\\hline
        Category Title  & \acrshort{cat\cat} \\\hline
        Definition      & \acrlong{cat\cat} \\\hline
        Anchor examples &          
        \begin{itemize}
        \setlength{\itemindent}{-2em}
            \item rq1-49:30-10: Isolating a {pattern} or subsystem for testing does not work.
            \item rq1-53:2-11: The crud (\acrlong{crud}) and \acrshort{crud} in the interface don't always work as it should be.
        \end{itemize}
        \\\hline

        Coding rules    & Technical operation of Ampersand \\\hline
    \end{tabularx}
    \caption{Category \acrshort{cat\cat}}
    \label{tab:Ampersand as tool}
\end{table}

When the script is ready for compilation, removing includes appears to cause a compilation error.
On the other hand, it also doesn't seem necessary to include everything via includes.
And appear to be includes especially for structuring the Conceptual analysis.
This is a side effect of the lack of proper refactoring options.

Despite the fact that Ampersand is very flexible and easily expandable, it lacks an option to break up a model and share parts.
In the situation where there is a common model and several individual models, it is not possible to separate these individual models.
When attempting to do that, we get no further than the common part in combination with one of the individual models.
The case is that we wanted to realize an implementation per registry.
The common part is recording the person and the registration.
This part is used for all registers, but there may be differences in the requirements for each register.
The moment a second register is set up, the common part is also overwritten and all data, including database structure, from the first register is gone.

An important part of the prototype is the use of the Interface~\footnote{\url{https://ampersandtarski.gitbook.io/documentation/the-language-ampersand/services/layout-of-user-interfaces}}.
The \acrshort{crud} interface defines whether \acrshort{crud} items are active at the place in the interface.
Uppercase is active and lowercase is not active.
In not all cases this is also done by the Ampersand Interface.
The Respresent~\footnote{\url{https://ampersandtarski.gitbook.io/documentation/the-language-ampersand/atoms}} definition of a Concept also has an effect on the Interface.

Ampersand does not support all functions, but does offer the option to build it.
The documentation does not tell you exactly how to do this.
But in the examples and during the research it appears that it is possible.
This is then done in PHP and can then be called within a Rule.
Powerful, but somewhat clumsy.