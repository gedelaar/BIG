\subsection{Ampersand as tool} \label{Ampersand as tool}
\def\cat{6}
To view Ampersand as a tool, we look more closely at the technical functioning of Ampersand.

\begin{table}[H]
    \caption{Category \acrshort{cat\cat}}
    \begin{tabularx}{\linewidth}{|X|X|}
        \hline
        Category        & CAT1-\cat \\\hline
        Category Title  & \acrshort{cat\cat} \\\hline
        Definition      & \acrlong{cat\cat} \\\hline
        Anchor examples &          
        \begin{itemize}
        \setlength{\itemindent}{-2em}
            \item \nameref{obs:rq1-49:30-10}
            \item \nameref{obs:rq1-53:2-11}
        \end{itemize}
        \\\hline

        Coding rules    & Technical operation of Ampersand \\\hline
    \end{tabularx}
    \label{tab:Ampersand as tool}
\end{table}
\sbbs{1}{Includes}
\POstart{%1
    \POS{obs:rq1-15:4-10}     %\oref{obs:rq1-15:4-10}    x
    \POS{obs:rq1-81:20-11}     %\oref{obs:rq1-81:20-11}    x
    \POS{obs:rq1-82:20-11}     %\oref{obs:rq1-82:20-11}    x
    \POS{obs:rq1-90:14-12}     %\oref{obs:rq1-90:14-12}    x
}

When the script is ready for compilation, removing Includes appears to cause a compilation error (\oref{obs:rq1-81:20-11}).
On the other hand, it also doesn't seem necessary to include everything via includes (\oref{obs:rq1-82:20-11}) and appears to be includes especially for structuring the Conceptual analysis (\oref{obs:rq1-15:4-10}).
Keep the includes small (\oref{obs:rq1-90:14-12}).
\sbbs{2}{Common objects}
\POstart{%2
    \POS{obs:rq1-49:30-10}     %\oref{obs:rq1-49:30-10}    x
    \POS{obs:rq1-63:10-11}     %\oref{obs:rq1-63:10-11}    x
    \POS{obs:rq1-85:30-11}     %\oref{obs:rq1-85:30-11}    x
    \POS{obs:rq1-89:7-12}     %\oref{obs:rq1-89:7-12}    x
    \POS{obs:rq1-92:14-12}     %\oref{obs:rq1-92:14-12}    x
}

Ampersand is flexible and easily expandable (\oref{obs:rq1-63:10-11}), but it is not possible to split a model (\oref{obs:rq1-49:30-10}).
In the situation where there is a common model and several individual models, it is not possible to separate these individual models from each other by the common part.
The case is that we wanted to realize an implementation for each registry.
The common part is the recording of the person and the registration (\oref{obs:rq1-85:30-11}, \oref{obs:rq1-89:7-12}).
This component is used for all registries, but there may be differences in the requirements for each registry.
The moment a second registry is set up, the common part is also overwritten and all data, including database structure, is gone from the first defined registry (\oref{obs:rq1-92:14-12}).
\sbbs{3}{Crud}
\POstart{%3
    \POS{obs:rq1-12}            %\oref{obs:rq1-12}    x
    \POS{obs:rq1-50:30-10}      %\oref{obs:rq1-50:30-10}    x
    \POS{obs:rq4-4}             %\oref{obs:rq4-4}    x
    \POS{obs:rq1-40:10-10}      %\oref{obs:rq1-40:10-10}    x
    \POS{obs:rq1-53:2-11}       %\oref{obs:rq1-53:2-11}    x
    \POS{obs:rq1-83:27-11}      %\oref{obs:rq1-83:27-11}    x
}

An important part of the prototype is the use of the Interface~\footnote{\url{https://ampersandtarski.gitbook.io/documentation/the-language-ampersand/services/layout-of-user-interfaces}}.
The \acrshort{crud} interface defines whether \acrshort{crud} items are active at the place in the interface (\oref{obs:rq1-12}).
Uppercase is active and lowercase is not active (\oref{obs:rq1-12}).
In not all cases the \acrshort{crud} works correctly.
Validation on used is missing in some cases (\oref{obs:rq1-53:2-11}).
After performing the validations on the data and in case of input errors, the interface delivers many messages and these remain active across the screens (\oref{obs:rq4-4}).
The Respresent~\footnote{\url{https://ampersandtarski.gitbook.io/documentation/the-language-ampersand/atoms}} definition (\oref{obs:rq1-50:30-10}) of a Concept also has an effect on the Interface (\oref{obs:rq1-40:10-10}).
Experiment with HTML in the interface was not successful, due to the lack of examples (\oref{obs:rq1-83:27-11}).
\sbbs{4}{PhP}
\POstart{%4
    \POS{obs:rq1-9}     %\oref{obs:rq1-9}    x
    \POS{obs:rq1-48:27-10}     %\oref{obs:rq1-48:27-10}    x
    \POS{obs:rq2-16:19-10}     %\oref{obs:rq2-16:19-10}    x
}

Ampersand supports creating new functions via php.
The documentation doesn't tell you exactly how to do this (\oref{obs:rq1-9}).
But in the examples and during the research it appears that it is possible (\oref{obs:rq1-48:27-10}, \oref{obs:rq2-16:19-10}).
This is done in PHP and can then be called within a line using "ExecEngine" (see~\ref{lst:persoon}).
\sbbs{5}{Model maintance}
\POstart{%5
    \PIS{int:I-1.5}     %\iref{int:I-1.5}    x
    \PIS{int:I-1.7}     %\iref{int:I-1.7}    x
    \PIS{int:I-2.4}     %\iref{int:I-2.4}    x
    \PIS{int:I-2.5}     %\iref{int:I-2.5}    x
    \POS{obs:rq4-7}     %\oref{obs:rq4-7}    x
}

Ampersand makes it possible to set up an information system without using a programming language.
In case of \acrshort{cibg} it is therefore not necessary to use C\#, the language used there (\iref{int:I-1.5}).
When performing the analysis, we always create a new model (\iref{int:I-1.7}).
There is no question of maintaining a model, because a new model is always being developed (\iref{int:I-2.5}).
When we go from model version one to model version two, we now have no tools available to facilitate the data conversion (\oref{obs:rq4-7}).
The data from Ampersand prototype is stored in a MariaDB database (\iref{int:I-2.4}).
\sbbs{6}{Excluded}
\POstart{%6
    \PIS{int:I-4.4}     %\iref{int:I-4.4}    o
    \POS{obs:rq1-10}     %\oref{obs:rq1-10}    o
    \POS{obs:rq1-65:10-11}     %\oref{obs:rq1-65:10-11}    o
    \POS{obs:rq1-66:10-11}     %\oref{obs:rq1-66:10-11}    o
    \POS{obs:rq1-91:14-12}     %\oref{obs:rq1-91:14-12}    o
    \POS{obs:rq1-98:30-12}     %\oref{obs:rq1-98:30-12}    o
    \POS{obs:rq2-9:7-10}     %\oref{obs:rq2-9:7-10}    o
}

The following observations have not been explicitly included.
