\newpage
\section{Results} \label{Results}

- wet lezen is een vak

- er zijn meer wetten bij betrokken dan enkel de wet BIG

- er zijn delen van de wet die niet meer geldig zijn, deze worden niet meegenomen

- inhoudelijk -> wetbig omvat ook tuchtrecht, dat is een andere tak van sport

- Ampersand heeft geen annotatie mogelijkheid -> vergt een separate actie/document om bij te houden wat er geweest is

- XML download van wetBig lijkt een logische stap, echter is deze zo complex opgebouwd dat dit geen zin heeft (idem voor json)

- A - indelen in Ampersand (patterns) heeft consequenties voor het prototype. VS ontbeert een refactoring optie bij verplaatsen

- VS heeft ook geen generieken zoek optie over de adl's heen

- middels include statements wordt de volgorde van document bepaald, maar niet overal zijn includes nodig

- door het lezen van de wet wordt er een opbouw duidelijk -> persoon; inschrijving; registratie->beheer; tucht->maatregelen; 

- inrichting van A in lokale omgeving is specifiek en niet evident; hulp is hier nodig

- notatie wijze van concepten en relaties en rules zijn deels vastgelegd. Enkel de eerste positie is hoofd of kleine letter; geen advies over overige schrijfwijze.

- aparte excel gemaakt om de multiplicteiten van de relaties uit te schrijven en te ontdekken

- wordt enkel gebruik gemaakt van UNI, TOT, INJ en SUR

- TOT wordt meestal ondervangen door een tot-rule -> blijkt dat een TOT tot gevolg heeft dat iets kan worden gesaved wanneer ingevuld, terwijl een tot-rule een save kan plaatsvinden terwijl de melding open blijft staan

- om een rule toe te passen is veel geduld en oefening nodig; vrij steile leercurve; 

- op internet geen voorbeeld te vinden, enkel in de repo van A zelf; en dat is moeizaam zoeken

- in het begin niet duidelijk wanneer C of c in de INTERFACE toe te passen; mogelijk staat dit wel in de handleiding, maar moet je toch proefondervindelijk ontdekken.

- automatische rules zijn beschreven, maar om te implementeren is ook hier veel geduld en proberen nodig. 

- represent definieert een type van een concept, maar datetime geeft problemen bij de interface

- implementatie in docker met RAP werkt wel, maar niet met includes om dat er steeds een nieuwe directory wordt aangemaakt; pas waneer er echt lokaal wordt gedraaid dan werkt het ook met includes

- in de wetbig wordt specialismes genoemd, maar geen lijst oid

- het toevoegen van de juiste beschrijving bij een concept en relatie is nog niet zo eenvoudig; gemakkelijk om af te dwalen en een eigen interpretatie toe te voegen. Ontbreekt een directe toets.

- bij interface kan je een FOR toevoegen voor autorisatie

- A kan niet rekenen; Maar aangezien A statisch is, kunnen proces gegevens op andere manier worden bewaakt.

- docker is ook nog een ding om te leren

- het toevoegen van stukjes php script moet mogelijk zijn, maar is niet duidelijk hoe

- de browser houdt data vast en er moet regelmatig een cache worden geleegd om het nieuwe werkend te krijgen

- interface lever veel meldingen en deze blijven ook staan

- inbedden in architectuur 

- elke relatie is onderdeel van een record structuur

- wat gebeurd er als A geimplementeerd is en er toch wijzigingen in de structuur plaatsvinden (normaal voor software)

- postman gebruikt voor API koppeling met A. 

- api koppeling werkt goed, echter komen hele meldingen terug; zouden eigenlijk codes moeten krijgen 

- er is geen swagger gemaakt voor de api; 

- kiss

- html href en target blank werkt niet

- output in latex 

- ui model; wettenkern met gedeelde concepten en proces deel; wettenkern is specifiek wet; gedeelde concepten zijn ook onderdeel van de wet maar komen ook elders voor;


