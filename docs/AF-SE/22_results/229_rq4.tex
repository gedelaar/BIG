\begin{comment}
RQ4 - Wat zijn de sterke en zwakke punten (SWOT) bij het gebruik van Ampersand voor registratiesystemen voor een overheidsorganisatie.
\end{comment}

\subsection{Ampersand in government environment}
[RQ4]- What are the strengths and weaknesses (SWOT) in using Ampersand for registry systems for a government organization.
\begin{enumerate}
    \item rq4-1 A kan niet rekenen; Maar aangezien A statisch is, kunnen proces gegevens op andere manier worden bewaakt.
    \newline\textbf{obs}: kan niet rekenen!
    
    \item rq4-2 api koppeling werkt goed, echter komen hele meldingen terug; zouden eigenlijk codes moeten krijgen 
    \newline\textbf{obs}: api returned een tekst/melding -> daar kunnen aanroepende applicaties meestal niet zo goed mee overweg. 
    gebruikelijk is om een code en soms met tekst terug te geven. 
    denk hierbij aan httpresponse codes
    
    \item rq4-3 inbedden in architectuur 
    \newline rq4-6 ui model; 
    wettenkern met gedeelde concepten en proces deel; 
    wettenkern is specifiek wet; 
    gedeelde concepten zijn ook onderdeel van de wet maar komen ook elders voor;
    dit is onderdeel van inbedding in de architectuur
    \newline\textbf{obs}: inbedden in de bestaande architectuur is van belang voor de bruikbaarheid, in de vorm van acceptatie
    
    \item rq4-4 interface levert veel meldingen en deze blijven ook staan
    \newline\textbf{obs}: schermen van het prototype raken vol met meldingen wanneer deze niet opgelost worden
    
    \item rq4-5 postman gebruikt voor API koppeling met A. 
    \newline\textbf{obs}: bij wijze van test is het mogelijk om Postman te gebruiken voor de koppeling. 
    Dus het is niet persé nodig om een applicatie hiervoor te bouwen
    
    \item rq4-7 wat gebeurt er als A geimplementeerd is en er toch wijzigingen in de structuur plaatsvinden (normaal voor software)
    \newline\textbf{obs}: wanneer er een nieuw model wordt gemaakt binnen Ampersand, dan wordt de datastructuur opnieuw geladen.
    Er zijn geen voorzieningen om de data die al ingevoerd is te behouden.
    -> zijn hier oplossingen voor
    
    \item rq4-8:22-11: team achter Ampersand is zeer toegewijdt. 
    \newline\textbf{obs}: calls worden snel opgelost. 
    voorbeeld was de foutmelding op een verkeerde multipliciteit. 
    (melding \url{https://github.com/AmpersandTarski/RAP/issues/128})
    het opnieuw laden van een nieuwe versie is niet evident
    
    
\end{enumerate}