\subsection{Design} \label{Design}
\def\cat{3}
The requirements for the design are limited beforehand.
The basis of the information was the law and it must fit within the architecture.

\begin{table}[H]
    \caption{Category \acrshort{cat\cat}}
    \begin{tabularx}{\linewidth}{|X|X|}
        \hline
        Category        & CAT1-\cat \\\hline
        Category Title  & \acrshort{cat\cat} \\\hline
        Definition      & \acrlong{cat\cat} \\\hline
        Anchor examples &  
        \begin{itemize}
        \setlength{\itemindent}{-2em}
            \item \nameref{obs:rq1-36:3-10}
            \item \nameref{obs:rq2-18:16-11}
        \end{itemize}\\\hline
        Coding rules    & requirements, solution, interface, documentation        \\\hline
    \end{tabularx}
    \label{tab:Design}
\end{table}
\sbbs{1}{Architectual fit}
\POstart{%1
    \PIS{int:I-1.12}     %\iref{int:I-1.12}    x
    \PIS{int:I-1.3}     %\iref{int:I-1.3}    x
    \PIS{int:I-2.1}     %\iref{int:I-2.1}    x
    \PIS{int:I-2.6}     %\iref{int:I-2.6}    x
    \PIS{int:I-2.7}     %\iref{int:I-2.7}    x
    \PIS{int:I-2.8}     %\iref{int:I-2.8}    x
    \POS{obs:rq4-1}     %\oref{obs:rq4-1}    x
    \POS{obs:rq4-3}     %\oref{obs:rq4-3}    x
    \POS{obs:rq1-63:10-11}     %\oref{obs:rq1-63:10-11}    x
    \POS{obs:rq1-89:7-12}     %\oref{obs:rq1-89:7-12}    x
}

Ampersand's design must fit into the architecture of the organization where it is used (\oref{obs:rq1-63:10-11}).
We see that the design using Ampersand overlaps with the existing \acrlong{rk} (\iref{int:I-1.3}).
Mapping is necessary to get Ampersand aligned with \acrlong{rk} (\iref{int:I-2.7}, \oref{obs:rq1-89:7-12}, \iref{int:I-1.12}, \iref{int:I-2.6}).
The angles that \acrlong{rk} and Ampersand use are different.
Ampersand is reactive and \acrlong{rk} is strongly workflow oriented (\iref{int:I-2.8}, \oref{obs:rq4-3}, \oref{obs:rq4-1}).
In particular, the generic part of the design shows coherence with \acrlong{rk} (\iref{int:I-2.1}).
Ampersand cannot calculate out of the box.
\sbbs{2}{Conceptual analysis}
\POstart{%2
    \POS{obs:rq1-1}     %\oref{obs:rq1-1}    x
    \POS{obs:rq1-75:20-11}     %\oref{obs:rq1-75:20-11}    x
    \POS{obs:rq1-78:20-11}     %\oref{obs:rq1-78:20-11}    x
    \POS{obs:rq1-99:6-1}     %\oref{obs:rq1-99:6-1}    x
    \POS{obs:rq2-18:16-11}     %\oref{obs:rq2-18:16-11}    x
}

The design is mainly expressed in the \acrlong{ca} and the prototype.
While building the script, there are options to influence the layout and readability of the analysis (\oref{obs:rq2-18:16-11}).
Adding HTML headers can disrupt the layout of the analysis considerably (\oref{obs:rq1-75:20-11}, \oref{obs:rq1-78:20-11}, \oref{obs:rq1-99:6-1}).
Using includes and patterns structures the \acrlong{ca} (\oref{obs:rq1-1}).
\sbbs{3}{Lifecycle law}
\POstart{%3
    \POS{obs:rq1-39:3-10}       %\oref{obs:rq1-39:3-10}    x
    \POS{obs:rq1-95:29-12}      %\oref{obs:rq1-95:29-12}    x
}

During the analysis of the law, it appears that the law focuses on the origin of the registration (\oref{obs:rq1-95:29-12}).
The law does not clearly consider the life cycle of and registers.
You will not find the steps to dismantle a register in the description of the law (\oref{obs:rq1-39:3-10}).
\sbbs{4}{Register unbundling}
\POstart{%4
    \POS{obs:rq1-92:14-12}     %\oref{obs:rq1-92:14-12}    x
    \POS{obs:rq1-93:19-12}     %\oref{obs:rq1-93:19-12}    x
    \POS{obs:rq3-8:19-9}     %\oref{obs:rq3-8:19-9}    x
}

The law speaks very clearly about multiple registers (\oref{obs:rq1-92:14-12}, \oref{obs:rq3-8:19-9}).
The current implementation of \acrshort{zorro} is set up as one register with differentiation per profession.
When the law is explicitly followed, this has consequences for the design (\oref{obs:rq1-93:19-12}).
This will then be set up with multiple registers.
\sbbs{5}{Test scenario}
\POstart{%5
    \PIS{int:I-4.2}     %\iref{int:I-4.2}    x
    \POS{obs:rq1-36:3-10}     %\oref{obs:rq1-36:3-10}    x
    \POS{obs:rq1-69:14-11}     %\oref{obs:rq1-69:14-11}    x
}

The use of Ampersand for the design means that there is the possibility to develop test scenarios early in the process (\oref{obs:rq1-69:14-11}).
Doing analysis and making the prototype gives the opportunity to create test scenarios (\oref{obs:rq1-36:3-10}, \iref{int:I-4.2}).
\sbbs{6}{Brp}
\POstart{%6
    \PIS{int:I-4.7}             %\iref{int:I-4.7}    x
    \POS{obs:rq2-17:10-11}     %\oref{obs:rq2-17:10-11}    x
    \POS{obs:rq3-10:12-10}     %\oref{obs:rq3-10:12-10}    x
    \POS{obs:rq3-11:12-10}     %\oref{obs:rq3-11:12-10}    x
}

In the context of the design you want to have things made explicit (\iref{int:I-4.7}, \oref{obs:rq2-17:10-11}).
Matters such as address management, country management and address formatting are not included in the law (\oref{obs:rq3-10:12-10}).
But these items are part of the design.
The manner of management must be sought elsewhere than in the law.
For example, address formatting is something that the BRP prescribes.
This is not an Ampersand item, but one that is encountered when analyzing legislation and regulations (\oref{obs:rq3-11:12-10}).
\sbbs{7}{Total design}
\POstart{%7
    \POS{obs:rq1-71:14-11}     %\oref{obs:rq1-71:14-11}    x
}

There is a lot to be said about the use of Ampersand in the design in many detail points.
But bottom-line, the design of the system grows by performing the analysis.
The analysis is performed using the scripts (\oref{obs:rq1-71:14-11}) (see appendix \ref{appendixAdl}).
A consequence of this analysis is the design, the \acrshort{ca}~(see appendix \ref{ConceptualAnalysis} and the prototype (see appendix \ref{appendixPrototype}.
\sbbs{8}{User experience}
\POstart{%8
    \PIS{int:I-1.10}     %\iref{int:I-1.10}    x
    \PIS{int:I-3.4}     %\iref{int:I-3.4}    x
    \PIS{int:I-3.5}     %\iref{int:I-3.5}    x
    \PIS{int:I-4.8}     %\iref{int:I-4.8}    x
}

The user usually works from lists and overviews.
Ampersand also provides lists and overviews via the interface (\iref{int:I-1.10}).
The purpose of an application is to unburden users and to support these users in processing the registry data.
Ampersand is well suited for this because the conceptual analysis, from which the implementation is derived, specifies exactly what is needed for the registry (\iref{int:I-3.4}, \iref{int:I-3.5}).
Ampersand's design is such that a new design is always delivered (\iref{int:I-4.8}).
\sbbs{9}{Excluded}
\POstart{%9
\PIS{int:I-4.4}     %\iref{int:I-4.4}    o
\POS{obs:rq4-2}     %\oref{obs:rq4-2}    o
\POS{obs:rq4-4}     %\oref{obs:rq4-4}    o
\POS{obs:rq4-7}     %\oref{obs:rq4-7}    o
\POS{obs:rq2-14:19-10}     %\oref{obs:rq2-14:19-10}    o
\POS{obs:rq2-15:19-10}     %\oref{obs:rq2-15:19-10}    o
\POS{obs:rq2-9:7-10}     %\oref{obs:rq2-9:7-10}    
}

The following observations have not been explicitly included.
