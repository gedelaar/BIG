\def\rq{rq3}
\cntA{Ampersand}
\cntA{api}
\cntA{classify}      
\cntA{concept}   
\cntA{Conceptual analysis}
\cntA{concepten}     
\cntA{crud}           
\cntA{Docker}   
\cntA{documentation}
\cntA{flexible}     
\cntA{include}
\cntA{interface}     
\cntA{latex}
\cntA{Lifecycle}
\cntA{linkto}        
\cntA{multiplicity}  
\cntA{Obsidian}
\cntA{pattern}
\cntA{php}
\cntA{population}    
\cntA{prototype}
\cntA{RAP}            
\cntA{relation}
\cntA{represent}      
\cntA{rule}
\cntA{VSC}
\cntA{XML}
\cntA{JSON}
\cntA{RTF}
\cntA{PDF}
\cntA{validation}
\cntA{architecture}



\begin{comment}
RQ3 - Hoe zijn de wet- en regelgeving opgezet zodat ze kunnen worden gebruikt in een bruikbare manier voor de Ampersand-methode.
\end{comment}
\subsection{The wet BIG and Ampersand}
[RQ3]- How are the laws and regulations set up so that they can be used in a useful way for the Ampersand method.
\begin{enumerate}
    \item rq3-2 door het lezen van de wet wordt er een opbouw duidelijk -> persoon; inschrijving; registratie->beheer; tucht->maatregelen; 
    \newline\textbf{obs}: de hoofdlijnen van de wet lijken, voor een niet jurist, duidelijk

    \item rq3-3 12-9 er zijn delen van de wet die niet meer geldig zijn, deze worden niet meegenomen
    \newline\textbf{obs}: de wet is best complex opgebouwd, en het mogelijk om terug te gaan in de tijd.
    Keuze die gemaakt is, is om niet terug te gaan in de tijd.
    
    \item rq3-4 12-9 er zijn meer wetten bij betrokken dan enkel de wet BIG
    \newline rq3-6 12-9 naast de web zijn ook besluiten van belang
    \newline\textbf{obs}: op de website van de wet zijn verwijzingen opgenomen naar andere wetten en regelgeving.
    
    \item rq3-5 12-9 jurisprudentie buiten scope -> nagaan of dit wel het geval is. 
    Maar beargumenteren waarom dit niet in een informatie systeem opgenomen zou moeten worden.
    \newline\textbf{obs}: is geen observatie maar een mening !!!

    \item rq3-7 het toevoegen van de juiste beschrijving bij een concept en relatie is nog niet zo eenvoudig; gemakkelijk om af te dwalen en een eigen interpretatie toe te voegen. 
    Ontbreekt een directe toets.
    \newline\textbf{obs}: tijdens het opstellen van concepten en relaties moet er direct een beschrijving bij van de positie waar het element vandaan komt.
    Dit gebeurt niet altijd omdat de scripttaal je zodanig bezig houdt (veel gepruts) zodat er vergeten wordt om de tekst toe te voegen.
    
    \item rq3-8:19-9: stelling in de wet dat er meerdere registers zijn. 
    Er is een register per beroepsgroep. 
    Mogelijk moet de indeling van de scripts ook op die wijze plaatsvinden.
    \newline\textbf{obs}: in de tekst van de wet wordt gesproken over meerdere registers. 
    De huidige implementatie Zorro laat zien dat er maar één register is geimplementeerd, met verschillende workflows voor de afhandeling van de beroepen (de feitelijke registers).
    
    \item rq3-9:19-9: opbouw van de registers is gelijk, registratie wordt ook wel inschrijving genoemd
    \newline\textbf{obs}: geen
    
    \item rq3-10:12-10: opmaak van de naam staat niet in de wet letterlijk genoemd, maar moet conform GBA/BRP normen. waar staat dat dan?
     \newline\textbf{obs}: is het relevant dat dit er niet in staat. 
     Dit zou op een andere plaats van in Ampersand afgedwongen kunnen worden. 
     Input validaties
     
    \item RQ3-11:12-10: zaken als autorisatie besluiten waardoor een informatie systeem BRP gegevens mag ophalen kom je in de wet niet tegen.
     \newline\textbf{obs}: de wet is niet gericht op de vertaling naar ICT
     
    \item rq3-12:12-10: ook kom je zaken als de termijnenwet niet tegen.
     \newline\textbf{obs}: geen
     
    \item rq3-13:17-10: in wetbig is geen lijst specialismen opgenomen, waar dan wel
     \newline\textbf{obs}: in de wet wordt wel verwezen naar specialisme
     
    \item rq3-14:19-10: een persoon en een big-nummer zijn heel verschillende zaken. Persoon is immutable, big nummer niet. Ze hebben wel een relatie met elkaar.
     \newline\textbf{obs}: in het spraak gebruik worden deze nog wel eens door elkaar gebruikt. 
     zodat het lijkt dat een persoon equivalent is aan een bignummer
     
    \item rq3-15:24-10: in de wet wordt de nationaliteit genoemd, ook wordt gerefereerd aan de EU en niet-eu bewoners, maar er wordt niet onderkent dat de nationaliteit definitie per land wordt gedefineerd.
     \newline\textbf{obs}: beperking van de wet
     
    \item rq3-16:24-10: voor nederland geldt dat we een landentabel hebben van de Rvig. Dit zijn nationaal vastgestelde en onderhouden tabellen. Er is hier dan ook geen onderhoudsfunctie op nodig.
     \newline\textbf{obs}: er is meer dan de wet
     
    \item rq2-17:10-11: een nl persoon heeft een adres dat aan de opmaak van de BRP moet/kan voldoen (zou een standaard bouwblok voor moeten zijn!), een buitenlands adres is onduidelijk wat hiermee moet.    
     \newline\textbf{obs}: zie rq-10
\end{enumerate}

\subsubsection{Reading law}
\begin{comment}
Hoort dit niet bij rq3?
plaats hier de afgehandelde items.
\end{comment}
\begin{enumerate}
    \item rq1-23:24-10: \textbf{wet} lezen is een vak 
    \newline\textbf{obs}: zie item
    
    \item rq1-25:12-9: eerst \textbf{overzicht} maken van de alle wetten en regelingen
    \newline\textbf{obs}: scoping belangrijk
    
    \item rq1-26:12-9: ook de wetten en de regelgevingen kunnen nog weer \textbf{verwijzigen} hebben naar andere wetten en regelgevingen. Omdat ze daarop gebaseerd kunnen zijn of deze uitbreiden.
    \newline\textbf{obs}: dat is waar.
    
    \item rq1-27:12-9: er zijn ook wetten en regels die niet opgenomen zijn in deze specifieke wet, maar geldig zijn vanuit een hoger liggende wet (implicite \textbf{verwijzingen}). 
    Denk hierbij aan de grondwet. 
    In geval van big zou dit bv de archiefwet kunnen zijn of de termijnenwet en strafrecht.
    \newline\textbf{obs}: hoe komt je tot een volledig overzicht van wetten en regelgeving
    
    \item rq1-28:12-9: regelgeving over een afgesloten en verleden, wordt niet meer meegenomen. 
    Deze is niet meer geldig. 
    Doel is niet om historie op te bouwen maar de huidige wet te ondersteunen.
    \newline\textbf{obs}: geen observatie maar meer scoping
    
    \item rq1-29:12-9: Niet alle wet- en regelgeving mbv BIG vinden we terug onder de \textbf{zoekterm} big.
    \newline\textbf{obs}: je moet er attend op zijn dat er meer is dan enkel BIG
    
    \item rq1-42:21-10: het is makkelijk om van de wetsteksten af te wijken. 
    Juist omdat deze zo lastig te lezen zijn. 
    En enigszins \textbf{kennis van de wet}cq het proces maakt dat eigen interpretatie snel gemaakt is. 
    Action research maakt ook dat je snel in deze valkuil stapt.
    \newline\textbf{obs}: zie item
    
    \item rq1-61:9-11: er zou een check op geb datum moeten komen(\textbf{rule}), dat iemand minimaal 18 moet zijn. 
    Klinkt logisch, maar is een afgeleide regel. 
    dus niet opnemen. 
    Deze zit al impliciet in de opleidingseis. 
    De duur van de opleiding maakt al dat iemand minimaal 18 jaar is voor dat de opleiding is afgerond.
    \newline\textbf{obs}: zie item
    
    \item rq1-95:29-12: de opmaak van een \textbf{concept} bignummer  is niet de wet opgenomen
    \newline\textbf{obs}: zouden er eisen zijn aan het bignummer nu is het volgens mij 8 cijfers, maar zou dit ook zo moeten zijn. 
    Een guid is misschien vanuit gebruikersvriendleijkheid niet zo handig
\end{enumerate}


\begin{tabular}{ || l | c | c ||}
    \hline
    object in \rq & count & y\\
    \hline\hline
    \tabifA{Ampersand}
    \tabifA{architecture}
    \tabifA{api}
    \tabifA{classify}      
    \tabifA{concept}     
    \tabifA{Conceptual analysis}
    \tabifA{crud}           
    \tabifA{Docker}   
    \tabifA{documentation}
    \tabifA{flexible}    
    \tabifA{include}
    \tabifA{interface}     
    \tabifA{JSON}
    \tabifA{latex}    
    \tabifA{Lifecycle}
    \tabifA{linkto}        
    \tabifA{multiplicity}  
    \tabifA{Obsidian}
    \tabifA{pattern}
    \tabifA{PDF}
    \tabifA{php}
    \tabifA{population}    
    \tabifA{prototype}
    \tabifA{RAP}            
    \tabifA{relation}
    \tabifA{represent}      
    \tabifA{RTF}
    \tabifA{rule}
    \tabifA{validation}
    \tabifA{VSC}
    \tabifA{XML}

\end{tabular}

\newpage
