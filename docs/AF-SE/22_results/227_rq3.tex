\begin{comment}
RQ3 - Hoe zijn de wet- en regelgeving opgezet zodat ze kunnen worden gebruikt in een bruikbare manier voor de Ampersand-methode.
\end{comment}
\subsection{The \acrshort{big} and Ampersand}
[RQ3]- How are the laws and regulations set up so that they can be used in a useful way for the Ampersand method.
\begin{enumerate}
    \item rq3-2 door het lezen van de wet wordt er een opbouw duidelijk -> persoon; inschrijving; registratie->beheer; tucht->maatregelen; 
    \newline\textbf{obs}: de hoofdlijnen van de wet lijken, voor een niet jurist, duidelijk

    \item rq3-3 12-9 er zijn delen van de wet die niet meer geldig zijn, deze worden niet meegenomen
    \newline\textbf{obs}: de wet is best complex opgebouwd, en het mogelijk om terug te gaan in de tijd.
    Keuze die gemaakt is, is om niet terug te gaan in de tijd.
    
    \item rq3-4 12-9 er zijn meer wetten bij betrokken dan enkel de wet BIG
    \newline rq3-6 12-9 naast de web zijn ook besluiten van belang
    \newline\textbf{obs}: op de website van de wet zijn verwijzingen opgenomen naar andere wetten en regelgeving.
    
    \item rq3-5 12-9 jurisprudentie buiten scope -> nagaan of dit wel het geval is. 
    Maar beargumenteren waarom dit niet in een informatie systeem opgenomen zou moeten worden.
    \newline\textbf{obs}: is geen observatie maar een mening !!!

    \item rq3-7 het toevoegen van de juiste beschrijving bij een concept en relatie is nog niet zo eenvoudig; gemakkelijk om af te dwalen en een eigen interpretatie toe te voegen. 
    Ontbreekt een directe toets.
    \newline\textbf{obs}: tijdens het opstellen van concepten en relaties moet er direct een beschrijving bij van de positie waar het element vandaan komt.
    Dit gebeurt niet altijd omdat de scripttaal je zodanig bezig houdt (veel gepruts) zodat er vergeten wordt om de tekst toe te voegen.
    
    \item rq3-8:19-9: stelling in de wet dat er meerdere registers zijn. 
    Er is een register per beroepsgroep. 
    Mogelijk moet de indeling van de scripts ook op die wijze plaatsvinden.
    \newline\textbf{obs}: in de tekst van de wet wordt gesproken over meerdere registers. 
    De huidige implementatie Zorro laat zien dat er maar één register is geimplementeerd, met verschillende workflows voor de afhandeling van de beroepen (de feitelijke registers).
    
    \item rq3-9:19-9: opbouw van de registers is gelijk, registratie wordt ook wel inschrijving genoemd
    \newline\textbf{obs}: 
    
    \item rq3-10:12-10: opmaak van de naam staat niet in de wet letterlijk genoemd, maar moet conform GBA/BRP normen. waar staat dat dan?
    \item RQ3-11:12-10: zaken als autorisatie besluiten waardoor een informatie systeem BRP gegevens mag ophalen kom je in de wet niet tegen.
    \item rq3-12:12-10: ook kom je zaken als de termijnenwet niet tegen. 
    \item rq3-13:17-10: in wetbig is geen lijst specialismen opgenomen, waar dan wel
    \item rq3-14:19-10: een persoon en een big-nummer zijn heel verschillende zaken. Persoon is immutable, big nummer niet. Ze hebben wel een relatie met elkaar.
    \item rq3-15:24-10: in de wet wordt de nationaliteit genoemd, ook wordt gerefereerd aan de EU en niet-eu bewoners, maar er wordt niet onderkent dat de nationaliteit definitie per land wordt gedefineerd.
    \item rq3-16:24-10: voor nederland geldt dat we een landentabel hebben van de Rvig. Dit zijn nationaal vastgestelde en onderhouden tabellen. Er is hier dan ook geen onderhoudsfunctie op nodig.
    \item rq2-17:10-11: een nl persoon heeft een adres dat aan de opmaak van de BRP moet/kan voldoen (zou een standaard bouwblok voor moeten zijn!), een buitenlands adres is onduidelijk wat hiermee moet.    
\end{enumerate}