\subsection{Registry systems} \label{Registry systems}
\def\cat{4}
Few observations about registration systems have been recorded.
Some comments were made in the interviews about information systems in general, but not about registers.

\begin{table}[H]
    \caption{Category \acrshort{cat\cat}}
    \begin{tabularx}{\linewidth}{|X|X|}
        \hline
        Category        & CAT1-\cat \\\hline
        Category Title  & \acrshort{cat\cat} \\\hline
        Definition      & \acrlong{cat\cat} \\\hline
        Anchor examples & 
        \begin{itemize}
            \setlength{\itemindent}{-2em}
                \item \nameref{obs:rq3-9:19-9}
            \end{itemize}\\\hline
        Coding rules    & All about use of information systems \\\hline
    \end{tabularx}
    \label{tab:Registry systems}
\end{table}
\begin{samepage}
    We have broken down ``\acrshort{cat\cat}'' into the following subcategories.
    We distinguish the following subcategories:
    \begin{itemize}[nosep,topsep=-1pt,parsep=1pt]
        \item \nameref{s:5_1_registerkern}
        \item \nameref{s:5_2_demarcation}
    \end{itemize}
\end{samepage}
Per subcategory, the observations (obs.) and the interview fragments (int.) are clustered and provided with a description of the results.
\sbbs{1}{Registerkern}\label{s:5_1_registerkern}
\POstart{%1
    \PIS{int:I-2.10}            %\iref{int:I-2.10}    x
    \PIS{int:I-2.6}             %\iref{int:I-2.6}    x
    \PIS{int:I-2.7}             %\iref{int:I-2.7}    x
    \PIS{int:I-2.9}             %\iref{int:I-2.9}    x
}

Registration systems are aligned with \acrshort{rk} at the \acrshort{cibg}.
The approach advocated by Ampersand is conceptually in line with \acrshort{rk} (\iref{int:I-2.10}).
The terminology used by \acrshort{rk} differs in many areas from what the \acrshort{big} prescribes.
This is not an Ampersand issue, but means that a mapping of terms must take place after the analysis (\iref{int:I-2.6}, \iref{int:I-2.7}).
A number of functions that we model in Ampersand also appear as generic functions in \acrshort{rk} (\iref{int:I-2.9}).
Since Ampersand is a reactive system, process support is controlled from \acrshort{rk} (\iref{int:I-2.9}).
\sbbs{2}{Demarcation}\label{s:5_2_demarcation}
\POstart{%2
    \POS{obs:rq1-85:30-11}      %\oref{obs:rq1-85:30-11}    x
    \POS{obs:rq1-92:14-12}      %\oref{obs:rq1-92:14-12}    x
    \POS{obs:rq3-8:19-9}        %\oref{obs:rq3-8:19-9}    x
    \POS{obs:rq3-9:19-9}        %\oref{obs:rq3-9:19-9}    x
}

Across registers, independent of \acrshort{rk}, there are common values between the laws.
Here also the generic elements must be marked, so that reuse can take place (\oref{obs:rq1-85:30-11}).
When reusing and separating elements, the multicontext problem is encountered (\oref{obs:rq1-92:14-12}).
The format of a register is subjective (\oref{obs:rq3-8:19-9}, \oref{obs:rq3-9:19-9}).

