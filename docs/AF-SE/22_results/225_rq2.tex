\subsection{Ampersand Conceptual Analysis}
[RQ2]- What are the Concepts, Relationships and Rules in the \acrshort{big}.
\begin{enumerate}
    \item rq2-1 inhoudelijk -> wetbig omvat ook tuchtrecht, dat is een andere tak van sport
    \newline\textbf{obs}: tuchtrecht laat zich niet makkelijk vangen in concepten en relaties. 
    Tuchtrecht bestaat uit andere zaken dan big??
     
    \item rq2-2 wordt enkel gebruik gemaakt van UNI, TOT, INJ en SUR
    \newline\textbf{obs}: ondanks er meer vormen van multipliciteit zijn, wordt in de praktijk (ook in de voorbeelden) voornamelijk UNI en TOT gebruikt. In mindere mate INJ en SUR. 
     
    \item rq2/rq3-3 in de wetbig wordt specialismes genoemd, maar geen lijst oid
    \newline\textbf{obs}: zie rq2
     
    \item rq2-4:30-9: Welke afspraken moeten gemaakt worden aangaande de structuur van de beschrijvingen. 
    Moeten we hier afspraken over maken of ongestructureerd laten.
    \newline\textbf{obs}: om te voorkomen dat het beschrijven een rommeltje wordt, moeten er afspraken (impliciet of expliciet) worden gemaakt over de manier van beschrijven. 
    de referenties benoemen.
    
    \item rq2-6:2-10/13-11: rules zijn niet simpel te realiseren. Er zijn wel trucjes om deze te realiseren.
    \newline rq1-17 om een rule toe te passen is veel geduld en oefening nodig; vrij steile leercurve; 
    \newline\textbf{obs}: rules vergen kennis van ampersand, maar ook veel voorbeelden. En die zijn niet heel erg voorhanden.
    
    \item rq2-7:4-10: persoon is niet gelijk aan bignummer; maar bignummer is attribuut van de inschrijving; een persoon kan meerdere bignummers hebben
    \newline\textbf{obs}: het lijkt in de tekst alsof een bignummer gelijkgesteld wordt aan een persoon. Verder doet dit niet heel erg terzake.
    
    \item rq2-8:7-10/10-10: geboortedatum moet opgemaakt worden als datum. Represent lijkt die rol te moeten vervullen.
    represent definieert een type van een concept, maar datetime geeft problemen bij de interface
    \newline\textbf{obs}: type van elementen zijn te sturen, onder voorwaarden
    
    \item rq2-9:7-10: inschrijftijd wordt automatisch toegevoegd. Dit middels rules.
    \newline\textbf{obs}: ondanks het niet kunnen toevoegen van php-functies blijkt het toch mogelijk te zijn om een datum-tijd automatisch toe te voegen. 
    In eerdere pogingen mislukte dat. 
    Pas bij ondersteuning ging dit werken.
    
    \item rq2-10:19-10: de naamgeving van een relatie wordt meestal toegekend aan de TRG attribuut van de set. zoals [Persoon * Voornaam] met als relatienaam "voornaam"
    \newline\textbf{obs}: het maken van afspraken rondom verwerking zijn belangrijk. Wanneer er afspraken zijn, dan zijn zaken ook weer terug te vinden.
    
    \item rq2-12:19-10: TOT heeft de eigenschap dat deze verplicht ingevuld moet worden in de Interface omdat de data anders niet wordt opgeslagen. Variant hierop is een rule met deze eigenschap, hierdoor worden de overige zaken wel opgeslagen in de database, maar blijft er een melding van onvolledigheid verschijnen.
    \newline rq1-21:7-11: TOT wordt meestal ondervangen door een tot-rule -> blijkt dat een TOT tot gevolg heeft dat iets kan worden gesaved wanneer ingevuld, terwijl een tot-rule een save kan plaatsvinden terwijl de melding open blijft staan
    \newline\textbf{obs}: er zijn dus meerdere manieren om met een TOT om te gaan. Dus ook de over de inzet van dit middel moet worden nagedacht.
    
    \item rq2-13:19-10: wat voor TOT geldt, dat geldt ook voor SUR
    \newline\textbf{obs}:  idem als rq2-12
    
    \item rq2-14:19-10: ROLE geeft controle aan de gebruiker. wat heb ik hier nu mee bedoeld?
    \newline\textbf{obs}: in het begin liet ik dit element buiten beschouwing, in de veronderstelling dat er een soort van autorisatie voor gold. 
    Maar dit is noodzakelijk omde RULE werkend te krijgen.
    
    \item rq2-15:19-10: in de Interface kan ook een FOR worden gebruikt. Hiermee worden gebruikersrollen ingevuld.
    \newline\textbf{obs}: Hier kan dus autorisatie geregeld worden. Vraag is even hoe dit dan werkt in bv een combinatie van API met FOR
    
    \item rq2-16:19-10/11-11: bij het bepalen van een periode heeft Ampersand het moeilijk. Rekenen kan A niet out of the box. Hier zijn de PHP functies voor nodig, die ook weer niet makkelijk te alloceren zijn.
    \newline\textbf{obs}: Ampersand kan niet out-of-the-box rekenen. 
    Is dit eigenlijk een probleem of moet je dit soort elementen bij Ampersand oplossen
    
    \item rq2-17:11-11/14-11: is er een verschil tussen inschrijving en registratie. Deze lijken naast elkaar te kunnen bestaan.
    \newline\textbf{obs}: de wet spreekt over inschrijving en later ook over registraties. 
    Vanuit de jurist wordt aangegeven dat dit hetzelfde is.
    Vraag me af of dit ook werkelijk zo is. 
    Er lijkt een timing verschil in dit gebruik te zitten.
    
    \item rq2-18:16-11: Goed beseffen dat de meanings die je opschrijft ook zo in de analyse terechtkomt. Dus kijken naar de wijze van opschrijven dat het een verhaal kan vormen in de analyse
    \newline\textbf{obs}: De meaning moet zodanig verwoord zijn dat al deze meanings een verhaal vormen.
    
    \item rq2-19:16-11: ampersand levert constraints en geen executable
    \newline\textbf{obs}: het is geen uitvoer bestand, maar een verzameling van database constraints die de kern van Ampersand is.
    
\end{enumerate}
