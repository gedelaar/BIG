
In addition to the \acrshort{big}, the investigation into the suitability of Ampersand also includes the list of regulations (see list ~\ref{list:ass-laws-regulations}).
Due to timeboxing it is not possible to analyze and process all associated laws and regulations.
The aim is not to give a fully fleshed out conceptual analysis, but to use and experience Ampersand.
During use, all observations are recorded.

Notable findings relate to the design of Ampersand, the set-up of the documentation, keeping the overview and setting up a team to work on the analysis.

In the following paragraphs we discuss the results obtained.
In paragraph~\ref{Method} it is indicated that we perform the content analysis according to \cite{mayring_qualitative_2000}.
From that perspective, there is a table for each category with the category ID, category title and definition (see~\ref{tab:Category_definitions}).
The table also contains one or more examples of observations or interview items related to the category.
The coding rules are a testing instrument for the observation and interview components.
When an item meets the coding rules, it can be placed in the appropriate category.

A subcategory ``excluded'' has also been added for all categories.
This is where the observations and interview fragments are placed that do not add value to the research.
The observations were made and therefore also named, but were not included in the analysis.
In section~\ref{section:discussion} the interpretations of the results are discussed.






