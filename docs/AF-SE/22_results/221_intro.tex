\begin{wrapfigure}{r}{.5\textwidth} 
    \includegraphics[scale=0.3]
        {Contented_Definitie_Ampersand_Wikipedia-1024x698.png}
    \caption{www.contented.nl/wat-weet-jij-van-het-en-teken-de-ampersand}
    \label{fig:Ampersand definition}
\end{wrapfigure}
The research focuses on the question of the usability of Ampersand(\&).
The ampersand sign~\footnote{\url{https://www.contented.nl/wat-weet-je-van-het-en-teken-de-ampersand}} and the Ampersand method both emphasize the meaning "and self-contained".
On the website of 
Ampersand~\footnote{\url{https://ampersandtarski.gitbook.io/documentation/why-ampersand/business-rules-in-ampersand}} 
we find an interpretation of the statement "standalone (op zichzelfstaand)".

During the graduation project, research was conducted into the suitability of Ampersand for designing registers for the government.
These registers are always based on legislation and regulations.
The research focuses on a specific law, namely the \acrshort{big}.

This law does not stand alone.
The associated laws and regulations are the following:
\begin{enumerate}
    \item Wet op de beroepen in de individuele gezondheidszorg
    \item Algemene wet bestuursrecht
    \item Besluit periodieke registratie Wet BIG
    \item Registratiebesluit BIG
    \item Tuchtrechtbesluit BIG
    \item Besluit gezondheidszorgpsycholoog
    \item Regeling periodieke registratie Wet BIG
    \item Regeling tarieven registratie beroepsbeoefenaren Wet BIG
    \item Algemene wet erkenning EU-beroepskwalificaties
    \item Besluit buitenslands gediplomeerden volksgezondheid
    \item Regeling erkenning EU-beroepskwalificaties beroepen in de individuele gezondheidszorg
\end{enumerate} 

In addition to the law-big, the investigation into the suitability of Ampersand also covers the above list of regulations.
Due to time-boxing it is not possible to analyze and process all associated laws and regulations.
The aim is not to provide a fully elaborated conceptual analysis, but to test Ampersand for usability.

The results of the study will be discussed in the following sections.
The research that focuses on the research question "\acrlong{research question}".
By answering the derived questions, we provide a view on the usability of Ampersand.



