
In addition to the \acrshort{big}, the investigation into the suitability of Ampersand also covers the list of regulations (see list ~\ref{list:ass-laws-regulations}).
Due to time-boxing it is not possible to analyze and process all associated laws and regulations.
The aim is not to provide a fully elaborated conceptual analysis, but to test Ampersand for usability.

Notable findings relate to the setup of Ampersand, the setup of the documentation, maintaining the overview and setting up a team working on the analysis.

In the following paragraphs we discuss the results obtained.
In section~\ref{Method} it is indicated that we perform the content analysis according to \cite{mayring_qualitative_2000}.
From that perspective, there is a table per category that contains the category-id, category title and definition (see~\ref{tab:Category_definitions}.
The table also includes one or more examples of observations or interview parts that relate to the category.
The coding rules are a testing instrument for the observations and interview parts.
When an item meets the coding rules, it can be placed in the relevant category.

In the next section, section~\ref{section:discussion} the interpretations of the results will be discussed.







