\def\rq{RQ2}
\cntA{Ampersand}
\cntA{api}
\cntA{classify}      
\cntA{concept}   
\cntA{Conceptual analysis}
\cntA{concepten}     
\cntA{crud}           
\cntA{Docker}   
\cntA{documentation}
\cntA{flexible}     
\cntA{include}
\cntA{interface}     
\cntA{latex}
\cntA{Lifecycle}
\cntA{linkto}        
\cntA{multiplicity}  
\cntA{Obsidian}
\cntA{pattern}
\cntA{php}
\cntA{population}    
\cntA{prototype}
\cntA{RAP}            
\cntA{relation}
\cntA{represent}      
\cntA{rule}
\cntA{VSC}
\cntA{XML}
\cntA{JSON}
\cntA{RTF}
\cntA{PDF}
\cntA{validation}
\cntA{architecture}

\acrlong{\rq}
\paragraph{Ampersand}
\begin{enumerate}
    \item rq2-19:16-11: \A{Ampersand} returns constraints and no executable
    \newline\textbf{obs}: It is not an executable file, but a collection of database constraints that is the core of Ampersand.

\end{enumerate}

\paragraph{Authorized}
\begin{enumerate}
    \item rq2-15:19-10: In the {interface} a FOR can also be used.
    This populates user roles.
    \newline\textbf{obs}: So be \A{authorized} can be arranged here.
    The question is how this works in, for example, a combination of api with FOR.

    \item rq2-14:19-10: The role gives control to the user.
    A user is \A{authorized} for use.
    It indicates which user is allowed to use the function.
    \newline\textbf{obs}: In the beginning I left this element out of consideration, assuming there was some kind of authorization.
    But this is necessary to get the {rule} working.

\end{enumerate}

\paragraph{Concept}
\begin{enumerate}
    \item rq2-7:4-10: A \A{concept} Person is not equal to BIG-number.
    A big number is an attribute of the registration.
    A person can have multiple BIG-numbers.
    \newline\textbf{obs}: It seems in the text that a BIG-number is equated with a person.

\end{enumerate}

\paragraph{Conceptual analysis}
\begin{enumerate}
    \item rq2-4:30-9: Which agreements must be made regarding the structure of the descriptions for \A{Conceptual analysis}.
    Do agreements have to be made about it or leave it unstructured?    
    \newline\textbf{obs}: To prevent the description from becoming a mess, agreements (implicit or explicit) must be made about the way of escribing.
    And name the references.

    \item rq2-18:16-11: Good to realize that the {meaning} you write down also ends up in the \A{Conceptual analysis}.
    So looking at the way of writing it down can form a story in the analysis.
    \newline\textbf{obs}: The meaning must be worded in such a way that all these meanings form a story.
    
\end{enumerate}

\paragraph{Interface}
\begin{enumerate}
    \item rq2-12:19-10: TOT has the property that this must be entered in the \A{interface} because otherwise the data will not be saved.
    A variant of this is an {rule} with this property
    As a result, the other items are stored in the database, but a notification of incompleteness continues to appear.
    \newline rq1-21:7-11: TOT is usually overcome by a tot-rule, it turns out that a TOT causes something to be saved when entered, while a tot-rule allows a save to occur while the notification remains open to stand.
    \newline\textbf{obs}: so there are several ways to deal with a TOT.
    Therefore, the use of this resource must also be considered.

    \item rq2-15:19-10: In the \A{interface} a FOR can also be used.
    This populates user roles.
    \newline\textbf{obs}: So be {authorized} can be arranged here.
    The question is how this works in, for example, a combination of api with FOR.

\end{enumerate}

\paragraph{Meaning}
\begin{enumerate}
    \item rq2-18:16-11: Good to realize that the \A{meaning} you write down also ends up in the {Conceptual analysis}.
    So looking at the way of writing it down can form a story in the analysis.
    \newline\textbf{obs}: The meaning must be worded in such a way that all these meanings form a story.

\end{enumerate}

\paragraph{Multiplicity}
\begin{enumerate}
    \item rq2-2 Only UNI, TOT, INJ and SUR are used.
    \newline\textbf{obs}: Although there are more forms of \A{multiplicity}, in practice (also in the examples) UNI and TOT are mainly used.
    To a lesser extent INJ and SUR.

    \item rq2-13:19-10: What applies to \A{multiplicity} TOT, also applies to SUR. 
    \newline\textbf{obs}: There are several ways to deal with a SUR.
    Therefore, the use of this resource must also be considered.

\end{enumerate}

\paragraph{Php}
\begin{enumerate}
    \item rq2-9:7-10: Subscription time is added automatically.
    This is done by means of a \A{rule}.
    \newline\textbf{obs}: Despite not being able to add \A{php} functions, it appears to be possible to add a date-time automatically.
    That failed in previous attempts.
    This only worked with support.

    \item rq2-16:19-10/11-11: Ampersand has a hard time determining a period.
    Ampersand cannot calculate out of the box.
    This requires the \A{php} functions, which are also not easy to allocate.
    \newline\textbf{obs}: Ampersand cannot calculate out-of-the-box.
    Is this actually a problem or do you have to solve these types of elements at Ampersand.

\end{enumerate}

\paragraph{Relation}
\begin{enumerate}
    \item rq2-1 Substantively includes \acrshort{big} also includes disciplinary law (tuchtrecht), which is another branch of sport.
    \newline\textbf{obs}: Disciplinary law is not easy to capture in {concept} and \A{relation}s.
    Disciplinary law consists more of processes and procedures.

    \item rq2-10:19-10: The naming of a \A{relation} is usually assigned to the TRG attribute of the set.
    Such as \textit{[Persoon * Voornaam]} with relation name "voornaam".
    \newline\textbf{obs}: Making agreements about processing is important.
    When there are agreements, things can also be found again.

\end{enumerate}

\paragraph{Represent}
\begin{enumerate}
    \item rq2-8:7-10/10-10: Date of birth must be formatted as date.
    The \A{represent} seems to have to fulfill that role.
    Represent defines a type of a concept, but DATETIME causes interface problems.
    \newline\textbf{obs}: Type of elements can be sent, subject to conditions.

\end{enumerate}

\paragraph{Rule}
\begin{enumerate}
    \item rq2-6:2-10/13-11: A \A{rule} is not easy to realize.
    There are tricks to realize this.
    \newline rq1-17 Applying a rule requires a lot of patience and practice.
    \newline\textbf{obs}: Rules require knowledge of Ampersand, but also many examples.
    And they are not very available.

    \item rq2-9:7-10: Subscription time is added automatically.
    This is done by means of a \A{rule}.
    \newline\textbf{obs}: Despite not being able to add {php} functions, it appears to be possible to add a date-time automatically.
    That failed in previous attempts.
    This only worked with support.

    \item rq2-12:19-10: TOT has the property that this must be entered in the {interface} because otherwise the data will not be saved.
    A variant of this is an \A{rule} with this property
    As a result, the other items are stored in the database, but a notification of incompleteness continues to appear.
    \newline rq1-21:7-11: TOT is usually overcome by a tot-rule, it turns out that a TOT causes something to be saved when entered, while a tot-rule allows a save to occur while the notification remains open to stand.
    \newline\textbf{obs}: so there are several ways to deal with a TOT.
    Therefore, the use of this resource must also be considered.

    \item rq2-14:19-10: The role gives control to the user.
    A user is {authorized} for use.
    It indicates which user is allowed to use the function.
    \newline\textbf{obs}: In the beginning I left this element out of consideration, assuming there was some kind of authorization.
    But this is necessary to get the \A{rule} working.

\end{enumerate}


\begin{tabular}{ || l | c | c ||}
    \hline
    object in \rq & count & y\\
    \hline\hline
    \tabifA{Ampersand}
    \tabifA{architecture}
    \tabifA{api}
    \tabifA{classify}      
    \tabifA{concept}     
    \tabifA{Conceptual analysis}
    \tabifA{crud}           
    \tabifA{Docker}   
    \tabifA{documentation}
    \tabifA{flexible}    
    \tabifA{include}
    \tabifA{interface}     
    \tabifA{JSON}
    \tabifA{latex}    
    \tabifA{Lifecycle}
    \tabifA{linkto}        
    \tabifA{multiplicity}  
    \tabifA{Obsidian}
    \tabifA{pattern}
    \tabifA{PDF}
    \tabifA{php}
    \tabifA{population}    
    \tabifA{prototype}
    \tabifA{RAP}            
    \tabifA{relation}
    \tabifA{represent}      
    \tabifA{RTF}
    \tabifA{rule}
    \tabifA{validation}
    \tabifA{VSC}
    \tabifA{XML}

\end{tabular}

\newpage
