
\INVA{I-1.1}{The assumption made by the interviewee that Ampersand is a tool that performs an interpretation on the law itself, is not correct. 
A manual stroke has to be done over the text of the law to recognize the concepts and relationships. 
This is seen as an intensive action.}

\INVA{I-1.2}
{Ampersand can be interesting, because it will be able to clear conflicting matters from the law. 
By performing the analysis, these will show up in the analysis. 
This makes it a resource to use before the law is enacted.}

\INVA{I-1.3}
{The \acrlong{rk}, an architecture model of CIBG, uses shared concepts. 
With this it has similarities with Ampersand. 
There is also an overlap of Ampersand with \acrlong{rk}. 
Within Ampersand are concepts that are also in \acrlong{rk}. 
\acrlong{rk} is a defining part of the architecture. Other parts will have to conform to this architecture.}

\INVA{I-1.4}
{The danger of using legislation and regulations is that there is a possible incomplete picture of the concepts. 
This by adopting the rules one-on-one, without the interpretations.
More laws are also used in an analysis than just the \acrshort{big}. 
The question is how far is the analysis of the various laws going.}

\INVA{I-1.5}
{To be able to use Ampersand it would be useful to avoid having to write code in C\#.}

\INVA{I-1.6}
{In addition, the implementation must be such that the effective dates of the specific amendments to the law are also taken into account. 
For example, at the time of an application, it is decisive whether the processing will take place in accordance with the old situation or the new situation.}

\INVA{I-1.7}
{Ampersand does not support a maintenance cycle. 
There must be a solution for this.}

\INVA{I-1.8}
{Ampersand's possible positioning is to use it as an interpreter of legislation and regulations. 
Then maintain the current analysis and development process and use the prototype to validate the analysis.
The question is whether this approach will not result in additional work compared to the current working method. 
There is a certain skepticism towards Ampersand.}

\INVA{I-1.9}
{Ampersand relies on facts and not on processes. 
While a practitioner is strongly process oriented. For example, the law does indicate that a diploma is required and also which type, but not exactly which diploma. 
So the law tells you what to do, but in most cases not how.}

\INVA{I-1.10}
{In addition, the practitioner's usual working method is that he works from overviews and lists. 
Ampersand will have to be designed for this with user requirements, because these things are not mentioned in the law. 
The law does not support a method and approach. 
This will have to be a so-called co-creation between IT and business.}

\INVA{I-1.11}
{The question is whether the wet-big is very suitable for this approach. 
The original law dates from 1993 and it is based on the legislation of 1865.}

\INVA{I-1.12}
{The terms case, submission and application are strongly represented in the handling of the registers. 
These terms do not appear in the Ampersand analysis. The term “case” is not mentioned at all in the law-big.
Because the process part is missing, this is considered a weakness of Ampersand. 
It is clear that Ampersand state is oriented and reactive and not process oriented.}