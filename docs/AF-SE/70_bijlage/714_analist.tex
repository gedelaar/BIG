
\INVA{I-4.1}
{The \acrshort{big} is big and also old. 
Ampersand could help detect inconsistencies in the law.}

\INVA{I-4.2}
{The prototype shown is not easy for the user to understand. 
The user not only looks at the functionality, but also at the design.
The current design does not comply with the national government web guidelines.
The question is whether the user will be able to see through this. 
It was not part of the research, but it was stated that adjusting the CSS could bring closer to the web guidelines.}

\INVA{I-4.3}
{Ampersand its deployment could be applied to new tasks. 
These have no history and can be built from scratch using the Ampersand method.}

\INVA{I-4.4}
{A use case can also be devised for the use of rebuilding existing systems. 
Through the analysis with the help of Ampersand, a system can be rebuilt in which the waste has been cut away. 
The question is how much this waste would be. Worth a try.}

\INVA{I-4.5}
{For use, the question is how quickly a base is set up. 
It may be difficult to get to a 100\% model. 
It may also be okay if this covers an 80\% charge.
LCSH as new project could be a good candidate.}

\INVA{I-4.6}
{Ampersand method is a way of writing things down. 
That is not necessarily better or worse than any other method. 
So when something is being written down, so analysis is being done, why not with this. 
More is possible with it than with a Word document. 
The output is good to use and the structure too.}

\INVA{I-4.7}
{In the current trend, validations are usually located in the business layer.
Is that also the case with Ampersand? 
The validations are spread over the database and surrounding code.}

\INVA{I-4.8}
{How is the maintenance of the system? 
A new model is always made with the help of Ampersand. 
The data will have to be migrated itself. 
Ampersand does not support that. 
Usually the data structure is taken into account in advance so that as little conversion as possible has to take place. 
This means that a system is getting bigger and less manageable. 
So the strength of Ampersand is that this is prevented because a new core system is always being built and the effort is in the data conversion and the connection of adjacent systems.}

\INVA{I-4.9}
{The learning curve doesn't seem that big. 
Even less technical people can work with this. 
With the adjustment in the styling, a prototype can be quickly made with which a working system can be demonstrated. 
On the other hand, only the conceptual analysis can be used. 
Based on this analysis, test scenarios can be devised and executed.}

\INVA{I-4.10}
{The Ampersand approach is different from most products. 
Most workbenches work from a drawn model and from there generate code from the documentation or possibly. 
Ampersand does this from a script and generates the models and documentation itself.}

\INVA{I-4.11}
{The question is whether the system will only work for simple registers or whether we can also use it to tackle complex registers.
The \acrshort{big} is complex, but not fully analyzed either.}

\INVA{I-4.12}
{A follow-up study could be to make a comparison between a system built traditionally and a system built on the Ampersand method. 
It is expected that due to code generation and being closer to the law, the amount of code will be a lot less and with that also a better SIG qualification.}

\INVA{I-4.13}
{To start with, a team should be set up to deal with this. 
This team of lawyers and analysts should be doing the analysis of a law and have it built.}

\INVA{I-4.14}
{One could also only use the output of the analysis to build a system. 
Multiple scenarios are possible.}