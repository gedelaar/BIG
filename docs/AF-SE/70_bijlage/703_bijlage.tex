\def\rq{RQ3}
\cntA{Ampersand}
\cntA{api}
\cntA{classify}      
\cntA{concept}   
\cntA{Conceptual analysis}
\cntA{concepten}     
\cntA{crud}           
\cntA{Docker}   
\cntA{documentation}
\cntA{flexible}     
\cntA{include}
\cntA{interface}     
\cntA{latex}
\cntA{Lifecycle}
\cntA{linkto}        
\cntA{multiplicity}  
\cntA{Obsidian}
\cntA{pattern}
\cntA{php}
\cntA{population}    
\cntA{prototype}
\cntA{RAP}            
\cntA{relation}
\cntA{represent}      
\cntA{rule}
\cntA{VSC}
\cntA{XML}
\cntA{JSON}
\cntA{RTF}
\cntA{PDF}
\cntA{validation}
\cntA{architecture}

\acrlong{\rq}

\paragraph{Concept}
\begin{enumerate}
    \item rq3-2 By reading the law, a structure becomes clear.
    The \A{concept} \mbox{Person}, \mbox{Registration} and \mbox{Registration} with management and Discipline(Discipline) with measures.
    \newline\textbf{obs}: The main lines of the law seem clear to a non-lawyer.

    \item rq1-95:29-12: The format of a \A{concept} big number is not included in the law
    \newline\textbf{obs}: Should there be requirements for the big number now I think it is 8 digits, but there should be.
    A guid may not be very useful for user-friendliness.

    \item rq2-17:10-11: A dutch person has an \A{concept} address that must conform to the BRP format (should be a standard building block for it!).
    A foreign address is unclear what to do with this.
    \newline\textbf{obs}: It is unclear how to handle addresses.

    \item rq3-14:19-10: A \A{concept} person and a big number are very different things.
    Person is immutable, big number is not.
    They do have a {relation} with each other.
    \newline\textbf{obs}: In speech, these are sometimes used interchangeably, so that it seems that a person is equivalent to a big number.
    
    \item rq2-1 Substantively includes \acrshort{big} also includes disciplinary law (tuchtrecht), which is another branch of sport.
    \newline\textbf{obs}: Disciplinary law is not easy to capture in \A{concept} and {relation}s.
    Disciplinary law consists more of processes and procedures.

\end{enumerate}

\paragraph{Documentation}
\begin{enumerate}
    \item rq3-7 Adding \A{documentation} with the correct description to a concept and relation is not so easy.
    Easy to stray and add your own interpretation.
    \newline\textbf{obs}: While drafting concepts and relationships, a description of the position where the element comes from must be immediately included.
    This doesn't always happen because the scripting language keeps you so busy (a lot of messing around) that you forget to add the text.

\end{enumerate}

\paragraph{Interpretation}
\begin{enumerate}
    \item rq1-42:21-10: It is easy to deviate from the legal texts.
    Because they are so hard to read.
    Some knowledge of the {law} or the process means that your own \A{interpretation} is quickly made.
    Action research also means that you quickly fall into this trap.
    \newline\textbf{obs}: Due to complex texts, there is a danger that knowledge is trusted on your own background.

\end{enumerate}

\paragraph{Law}
\begin{enumerate}
    \item rq3-3 12-9 There are parts of the \A{law} that are no longer valid, they are not included
    \newline\textbf{obs}: The law is quite complex, and it is possible to go back in time.
    The choice that has been made is not to go back in time within this scope.

    \item rq3-4 12-9 There are more \A{law}s involved than just the \acrshort{big}.
    \newline rq3-6 12-9 In addition to the law, decisions are also important.
    \newline\textbf{obs}: The law website contains references to other laws and regulations.

    \item rq1-42:21-10: It is easy to deviate from the legal texts.
    Because they are so hard to read.
    Some knowledge of the \A{law} or the process means that your own {interpretation} is quickly made.
    Action research also means that you quickly fall into this trap.
    \newline\textbf{obs}: Due to complex texts, there is a danger that knowledge is trusted on your own background.

    \item rq1-29:12-9: Not all \A{law}- and regulations using \acrshort{big} can be found under the search term "big".
    \newline\textbf{obs}: There is more than just \acrshort{big}.

    \item rq1-26:12-9: Also the \A{law}s and the regulations can still have {references} to other laws and regulations.
    Because they can be based on these laws or extend it.
    \newline\textbf{obs}: Scoping is important.
    
    \item rq1-27:12-9: There are also laws and regulations that are not included in this particular \A{law}, but are valid from a higher law (implicit references).
    In case of \acrshort{big} this could be eg the Archives Act or the Time Limits Act and Criminal Law.
    \newline\textbf{obs}: To get a complete overview of laws and regulations, the help of a lawyer is needed.
    
    \item rq1-23:24-10: \A{law} Reading is a skill.
    \newline\textbf{obs}: The law consists of jargon and you need a lawyer for that.

    \item rq3-16:24-10: For the Netherlands, we have a country table from the RvIG.
    These are nationally established and maintained tables.
    No maintenance function is therefore required.
    \newline\textbf{obs}: There is more than the \A{law}.

    \item rq3-15:24-10: In the \A{law} the nationality is mentioned, it also refers to the EU and non-eu residents.
    It is not recognized that the nationality definition is defined per country.
    \newline\textbf{obs}: This is a limitation of the law.

    \item rq3-10:12-10: Formatting of the name is not stated literally in the \A{law}, but must conform to BRP standards.
    \newline\textbf{obs}: Is it relevant that this is not in there.
    This could be enforced elsewhere than in Ampersand.
    Input validations at a front-end system.
     
    \item RQ3-11:12-10: Matters such as authorization decisions that allow an information system to retrieve BRP data are not found in the \A{law}.
    \newline\textbf{obs}: The law does not focus on the translation to ICT.
    
\end{enumerate}

\paragraph{Overview}
\begin{enumerate}
    \item rq1-25:12-9: First make \A{overview} of all laws and regulations
    \newline\textbf{obs}: Scoping is important.

\end{enumerate}

\paragraph{Register}
\begin{enumerate}
    \item rq3-9:19-9: The structure of the \A{register}'s is the same, registers are also called registrations
    \newline\textbf{obs}: Commonality emerges here.

    \item rq3-8:19-9: The law states that there are multiple registers.
    There is a \A{register} per profession.
    The scripts may also need to be formatted that way.
    \newline\textbf{obs}: Multiple registers are mentioned in the text of the law.
    The current implementation of Zorro shows that only one register has been implemented, with different workflows for handling the professions (the actual registers).

\end{enumerate}

\paragraph{Relation}
\begin{enumerate}
    \item rq3-14:19-10: A {concept} person and a big number are very different things.
    Person is immutable, big number is not.
    They do have a \A{relation} with each other.
    \newline\textbf{obs}: In speech, these are sometimes used interchangeably, so that it seems that a person is equivalent to a big number.

\end{enumerate}

\paragraph{Rule}
\begin{enumerate}
    \item rq1-61:9-11: There should be a check on the draft date of birth(\A{rule}), so that someone must be at least 18.
    Sounds logical, but is a derived rule.
    This is already implicit in the training requirement.
    The duration of the training means that someone is at least 18 years old before the training is completed.
    \newline\textbf{obs}: You don't have to think of anything yourself.

\end{enumerate}

\paragraph{Specialism}
\begin{enumerate}
    \item rq3-13:17-10: There is no list of specialties in \acrshort{big}, where is it?
    \newline\textbf{obs}: The law does refer to \A{specialism}.

\end{enumerate}



\begin{tabular}{ || l | c | c ||}
    \hline
    object in \rq & count & y\\
    \hline\hline
    \tabifA{Ampersand}
    \tabifA{architecture}
    \tabifA{api}
    \tabifA{classify}      
    \tabifA{concept}     
    \tabifA{Conceptual analysis}
    \tabifA{crud}           
    \tabifA{Docker}   
    \tabifA{documentation}
    \tabifA{flexible}    
    \tabifA{include}
    \tabifA{interface}     
    \tabifA{JSON}
    \tabifA{latex}    
    \tabifA{Lifecycle}
    \tabifA{linkto}        
    \tabifA{multiplicity}  
    \tabifA{Obsidian}
    \tabifA{pattern}
    \tabifA{PDF}
    \tabifA{php}
    \tabifA{population}    
    \tabifA{prototype}
    \tabifA{RAP}            
    \tabifA{relation}
    \tabifA{represent}      
    \tabifA{RTF}
    \tabifA{rule}
    \tabifA{validation}
    \tabifA{VSC}
    \tabifA{XML}

\end{tabular}

\newpage
