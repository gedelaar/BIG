For a lawyer, the IT environment is an unfamiliar environment. 
The lawyer actually wants to form a picture of the system. 
In particular, what it looks like and what it can do. 
This while we want to involve the lawyer at the beginning of the process.
Especially when we don't have the system yet.

A draft of the conceptual analysis is available and this is experienced as trusted by the lawyer. 
In fact, these are recognizable texts because they have been taken directly from the law.

The \acrshort{big} offers a lot of room for interpretation. 
This interpretation possibility means that the law may lend itself less to an Ampersand translation than a recent law would.
The new laws have therefore been drafted more carefully. 
The law provides a framework and the question is how far one should go with recording. 
This law gives the freedom to fill in matters yourself.

The aim should not be to record everything that is stated in the law in an ICT system. 
That makes it very rigid.
Make sure that 80\% of the situations are supported and leave the rest to the employees.
Ampersand is very suitable for this, precisely because it has a reactive approach and therefore does not prescribe how the practitioners should act.

The aim of an ICT system should be to do as little manual work as possible. 
And when a new law is being developed, an ICT representative should be present.

Because the law was drafted some time ago, the definitions are not always unambiguous. 
And because of the aforementioned interpretation possibility, the legislator can interpret the law slightly differently through jurisprudence.

The law consists of the following parts.
\begin{itemize}
    \item Artikel 1 definities 
    \item Artikel 3 welke beroepen ed
    \item Artikel 4 uitbreiding Artikel 3
    \item Artikel 5 grondslagen voor regelgeving
    \item Artikel 6 weigergronden
    \item Artikel 7 doorhalen - 7a hardheidsclausules
    \item Artikel 8 basis voor herregistratie, technische artikelen
    \item Artikel 9 tuchtgebeuren; wat we aantekenen op het register (inschrijving = registratie) maatregelen, doorgehaalde reden bepaald of je zichtbaar bent 
    \item Artikel 10 beschikking
    \item Artikel 11 aanmelden beschikking (staatscourant)
    \item Artikel 12 openbaarmaking big-registratie - staat wat er gemeld mag worden
    \item Artikel 13 privacy + delen van info; grondslagen per doelgroep
    \item Artikel 14 beroepsverenigingen - wordt aangetekend in big-register
    \item Artikel 15.16.17 specialisten registers
    \item H3- Artikel 18 eisen per beroep oa opleiding tm Artikel 33
    \item Artikel 34 geen beroepstitel, maar wel behandeld als a3-beroep; ook opleiding is bepalend. Geen eigen register
    \item Artikel 35 voorbehouden handelingen
    \item Artikel 36a+b tijdelijke registers bv mondhygenistes
    \item h5 tav buitenlandse gediplomeerden; erkenning process (EU, overige buitenland)
    \item Artikel 45 als Artikel 34
    \item h6/7 tucht
    \item etc

\end{itemize}
Going through the law should be a first step for the conceptual analysis.