\INVA{I-2.1}{CIBG's architecture for new registers consists largely of \acrlong{rk}. 
This was introduced not so long ago and is still being expanded.}

\INVA{I-2.2}{Ampersand has APIs and that is interesting to be able to link with. 
Whether that can also be linked with \acrlong{rk} is not clear at the moment.}

\INVA{I-2.3}{Nice that Ampersand is an open source product. 
There is not much to be found. 
Only the github repository can be found.}

\INVA{I-2.4}{Does Ampersand support databases other than just MariaDB? Not at the moment, but it is to be expected that this will be possible.}

\INVA{I-2.5}{When maintenance takes place on the model, how do we get from one model to another. 
So how does the IST go to SOLL situation. 
Ampersand is always creating a new model. 
So when the law is changed and a new model is needed as a result, Ampersand will produce a completely new model. 
As a result, no technical debt will remain in the model.
It is always a new model. 
However, the challenge will be in the data migration from the old to the new model.}

\INVA{I-2.6}
{\acrlong{rk} its terminology includes things and products. 
Every service, read implementation of a law, we call a product. 
There are standard parts that always appear in every register. 
These are pre-modeled within \acrlong{rk}. 
This includes a base for each registry and can be expanded according to the needs of the registry. 
The basis is the minimum common denominator of the registers. 
Extendable to specific elements arising from the law. 
There is certainly overlap in the data obtained from the analysis of the big law and the \acrlong{rk}. 
About 80\% of the \acrlong{rk} is generic and the other 20\% is customised. 
So all new registers have the same basic principles and for the most part run on the same software.}

\INVA{I-2.7}{Another aspect of the terminology is that items with the same definition are named differently within the law and within the \acrlong{rk}. In \acrlong{rk} we are talking about business and products, while the law is big about registrations, applications and professional registers. 
A mapping of the terms used will have to take place.}

\INVA{I-2.8}{Due to the overlap between \acrlong{rk} and the Conceptual analysis of Big, it is difficult to find the demarcation line between the two systems. 
Ampersand is state oriented and the \acrlong{rk} is process oriented. 
The link and cooperation must be sought.}

\INVA{I-2.9}{The usual procedure within a register is the application process for a registration. 
The \acrlong{rk} has a wizard for this, which includes a diploma check, for example. 
This diploma check is also part of the current implementation of the \acrshort{big}.}

\INVA{I-2.10}{Ampersand's approach is in line with the \acrlong{rk}, but not at the implementation level. 
It doesn't seem possible to implement Ampersand directly, but the analysis seems quite useful for extending \acrlong{rk}. 
Where generality is discovered and for the specific parts of the law. 
Then we are talking about a conceptual link and not a technical one.}

\INVA{I-2.11}{An addition of Ampersand is that a prototype is made that can also be tested. 
This allows the entire system to be tested because this combination must comply with validation from the law.}