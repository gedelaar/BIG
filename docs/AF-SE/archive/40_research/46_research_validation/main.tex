\begin{comment}
\item[RQ1]- What knowledge, in the role of software engineer, is needed to use Ampersand.
\item[RQ2]- What are the Concepts, Relationships and Rules in the \acrshort{big}.
\item[RQ3]- How are the laws and regulations set up so that they can be used in a useful way for the Ampersand method.
\item[RQ4]- What are the strengths and weaknesses (SWOT) in using Ampersand for registry systems for a government organization.
\end{comment}



\subsection{Validation}
\begin{comment}
hoe check je de validiteit van de bevindingen
Possibly the paper is using standardized questionnaires, or the references provide leads to other articles that have contained the information, or you have to take the material as having face validity and validate it more thoroughly yourself. Otherwise, the indication is that this is sloppy reporting, so don't bother!

During data gathering: You should validate your information through yes, triangulation method and exercise your objective judgment as to the data informants or participants or respondents have shared. After collating data, go back to them and ask if you got things or ideas right as they had revealed or shared. (esp. for qualitative data).
After everything is done:
Organize a meeting or conference of major stakeholders, key players, key informants or participants of your study. They will form part of your audience as you present and validate the research report that you have drafted. Note of their comments and suggestions needed to finalize the report or final output.
\end{comment}
From a constructive perspective, the validation of the results is quite complex.
The repeatability of the study and obtaining equal results seems impossible according~\cite{sandelowski_rigor_1993}.
Many of the results are based on interpretation of the information, in this case \acrfull{big}.
Triangulation can be used to validate the results.
This has also been used to collect results.
In addition to supplementing, the information can also be used to check the results among themselves.
In addition, it must be made clear what bias is involved.
Deviations from the expected result must also be made transparent.
To prevent my own organization from failing in the process, it is important to properly involve them in the process.

The usefulness of the Ampersand method is determined by two things.
On the one hand, it is the method itself that must be understood and used correctly.
This point is hedged in RQ1 and RQ2.
The validation of these points will emerge during the investigation.

On the other hand, the research relates to the usability of the end product.
The CIBG organization will have to determine whether they can continue with the end product.
The recognizability and the degree of acceptance determine the usability for the organization.
In order to determine this usability for the organization, we will have to regularly discuss the results with the key players during the research.

The identified key players within the CIBG organization are:
\begin{itemize}
    \item Enterprise architect.
    \item Tactical triangle with the product owner, advisor information provision and application manager.
    \item Development team Zorro.
\end{itemize}

Consultation structure with the enterprise architect has been set up and will take place once every two weeks.
This needs to be coordinated with the other groups.