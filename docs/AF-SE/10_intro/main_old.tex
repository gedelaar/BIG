\section{Introduction} \label{Introduction}

Within the government, a lot of data is stored in register systems, also called registers.
The registers mainly contain personal information for the domain for which the register has been drawn up.
These registers can be used both inside and outside the government.

Registers are based on laws and regulations.
These laws and regulations explicitly lay down what is possible within the register and what it is used for.

The translation of legislation and regulations into the register must be as direct as possible.
Ampersand is used for this translation because Ampersand supports a direct translation of laws and regulations into formal language.
This research is used to determine the usefulness of Ampersand in translating legislation and regulations into registry specifications.

Ampersand consists of concepts, relations and rules.
Using these components, it can generate a prototype in which all boundary conditions are monitored.
The core of Ampersand contains the knowledge of relation algebra~\footnote{\url{https://nl.wikipedia.org/wiki/Relational_algebra}} and thus the operations on sets.
In particular, Ampersand supports software development using relation algebra~\citep{joosten_software_2017}.
It also creates a functional design based on the model created with the Ampersand script.

The real-life situation that will be treated as a case is the \acrfull{big}.
When the \acrshort{big} went into effect on November 11, 1993, an executive agency was established to enforce this law.
In 1995 it was \acrfull{cibg} that performed this task.
Laws governing the implementation of \acrshort{big}~\footnote{\url{https://www.bigregister.nl/}}, Donor~\footnote{\url{https://www.donorregister.nl /} } and Uzi~\footnote{\url{https://www.uziregister.nl/}} are housed here.
Later other laws were added to the \acrshort{cibg}.
This led to the conclusion that \acrfull{cibg} could no longer keep this name, because it no longer covered the load.
In the current situation, the organization only uses the name \acrshort{cibg}.

One of the first projects within the \acrshort{cibg} was the Ribiz project.
The current case system~\acrshort{zorro} was developed as a successor to Ribiz in 2007.
The project name \acrshort{zorro} stands for \acrfull{zorro}.
The project name has persisted as an internal name for the system that implemented the \acrshort{big}.
The system has been active since 2007 and has been maintained and updated until now with all the changes \acrshort{big} went through.

\begin{wrapfigure}{r}{0.4\textwidth} 
    \includegraphics[scale=0.13]
        {images/big-register-cibg.png}
    \caption{Big-register}
    \label{fig:Big-register}
\end{wrapfigure}

The \acrshort{zorro} is deprecated.
Based on an \acrfull{alm} recommendation (2019)\citep{de_kok_analyse_2019}, it has been determined that the system is no longer adequate from the perspectives of security, maintenance, finances, functionality and process support.
Within the \acrshort{cibg} initiatives have been started to replace the current system.
New insights in the field of system development mean that the development will mainly use reusable building blocks.
The basis of the new system includes the principle of a register core.
Registry core is the principle where a registry is developed using the mentioned reusable building blocks and using a workflow system.


In order to determine the usefulness of the Ampersand method, we will investigate this case with a real-life situation, namely \acrshort{big}.
Since the BIG system needs to be replaced, this is an authentic problem.
As an application manager, I have been involved with BIG system for a number of years.
The CIBG organization has asked whether it is possible to investigate a more direct link and translation of the legislation towards registers.
That is why we are going to use Action Research~\citep{Easterbrook} as a research approach.








In the \nameref{context}, section \ref{context}, we discuss the related topics.
This is a closer look at Ampersand's basics, namely relation algebra, in subsection \ref{relation_algebra}.
Ampersand is also discussed in subsection \ref{ampersand}.
We give an idea how an Ampersand script looks like.
Unlike the common process approach, Ampersand uses an event-oriented approach.
This is explained in subsection \ref{reactive_approach}.
The current system, which focuses on a workflow that will be replaced, is Zorro and the case \acrshort{big} is also discussed.
In section \ref{problem_analysis} we look at the topic of the action research, namely the \acrshort{big}.
We determine the reason for the choice of action research.
In section \ref{Research} we look at the main question and related issues.
We also discuss the method, approach and validation of the action research.
The last section \ref{planning} focuses on planning.


