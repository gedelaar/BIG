\newpage
\section{Introduction} \label{Introduction}
%vv
Ampersand~\citep{joosten_software_2017} is used to specify registration systems.
We will investigate the usefulness of Ampersand for designing registration systems.
In this study, we want to determine if Ampersand is useful for government organizations that build and manage registration systems.

%vv
The usability of Ampersand is apparently a problem, reflected in its low usage.
The tool and its method of use work in practice. 
This has been shown by the efforts made in other projects, nevertheless the method is not widely used.
The question is why.

%Ampersand
%vv
Ampersand is a rule-based design tool.
It is a way to design information systems that comply with all business rules.
This makes business rules sufficient as a tool to design registration systems that comply with these rules.
A method and tool to create error-free specifications of the registration system to support business processes.
Ampersand supports the way in which the design and \acrlong{ca} are created.
We use the tooling to translate the design into a prototype and \acrshort{ca} that comply with all business rules.

%Vv
Ampersand is based on Relation Algebra~\citep{maddux_bibliography_2006}.
The most commonly used components of Ampersand's relation algebra are the Concept, the Relationship, and the Rules.
A Concept is an abstract representation of immutable items.
A Relationship defines a connection between two or more Concepts.
The Rules deal with validating Relationships between Concepts.

%vv
We use the Ampersand method to translate laws and regulations.
Legislation and regulations are strongly rule-oriented and because Ampersand is a rule-oriented design method, it is therefore ideally suited for legislation and regulations.
We will investigate whether this law is also suitable for the Ampersand analysis method.

%designing
%vv
The design process within Ampersand focuses on the creation of a prototype and the \acrlong{ca}.
We analyze the source texts through which the prototype and design are formed.
The source text in this case is the legislation and regulations.
The analysis of the legislation and regulations results in a script that generates a prototype and \acrlong{ca} using Ampersand.
Ampersand is both the method followed and a tool to obtain the result.
One of the research results is the \acrlong{ca} in which we can see which Concepts, Relations and Rules this law provides us with.

%registration
%vv
Ampersand is going to be deployed at the \acrshort{cibg}, which is an executive agency of the Ministry of Health, Welfare and Sport and therefore a government organization.
These organizations operate on the basis of laws and regulations.
The CIBG is a government organization that designs, builds and manages registration systems.
These registration systems, called registers, are always based on laws and regulations.
The purpose of a registry is to provide reliable and accessible information so that it can be used for individual consultation and as research data as used in ~\cite{schmidt_danish_2015} and ~\cite{bakken_norwegian_2019}.
The reliability of the data in the registry must be ensured by the management organization.

%investigated
%vv
The cause of the low use of Ampersand for designing register systems is unknown.
There is not enough information to determine the causes.
To find out the reason for the low usage, we will conduct an exploratory study.
Therefore, the exploratory approach chosen is ~\acrfull{ar} of ~\cite{Easterbrook}.
The exploratory approach of~\cite{Easterbrook} lends itself to the research using Ampersand to design register systems to derive hypotheses and construct theories.

%vv
The researcher is not completely independent of the case that will be investigated.
We adopt the approach of  \acrshort{ar} because the researcher is closely related to the research case.
Other arguments in support of the \acrshort{ar} approach relate to the \acrshort{cibg}.
The \acrshort{cibg}, the employer of the researcher, has an interest in the research and is particularly interested in the design of the registry.
Designing the successor to the registry system Zorro, the registry system for the \acrfull{big}, is a matter that is readily accessible for research.
Too few reference cases are available to conduct quantitative research.
Objective analysis is therefore not possible here.
Therefore, it makes sense to investigate the usefulness of Ampersand by actively participating in the design process and curiously examining what Ampersand does in that process.

%vv
In practice, we see that Ampersand is not that widely used.
In recent years, experiments have been conducted at TNO, KPN, Bank MeesPierson, ING-Bank, Rabobank and Delta Lloyd.
They have experimentally confirmed the method and provided insight into its practicality.

%vv
Despite the fact that Ampersand works in practice, it is little used.
A possible cause of the current low usage may be the unfamiliarity of Ampersand.
Which produces circular reasoning:
Unknown means it is rarely used, when it is rarely used, it remains unknown.
The popular products in the Open Source market are affiliated with a large organization that can push marketing and knowledge. 
For example: operating systems like Ubuntu which is maintained by Canonical Ltd.
In addition, such an organization can also build a community to support the product.
The usefulness of a product like Ampersand says nothing about its use.
Because it is not always the case that a useful product is widely used.
In the IT world, this happens with the Linux desktop versus a Windows desktop.
It contains all the functionalities needed to work with and has a functional user interface.
But many users still choose a Windows desktop.
Windows apparently has features that are considered indispensable.
Windows was also a forerunner in usability and has a great marketing department.
Despite the user-friendliness of Linux, it is mainly chosen by IT people or users who want a free operating system.


%vv
The research techniques used are limited to interviews with stakeholders and an analysis of the material collected during the study.
By showing stakeholders what Ampersand means to the registry, statements about the usefulness of the method can be elicited here.
To show what Ampersand delivers, we will also build an environment using the Ampersand method.
During the build, we will collect observations about issues we encounter.

%vv
Building the environment as a software engineer provides data on the usability of Ampersand.
By classifying this data, we gain quantitative insight into the points that stand out the most.
The numbers provided do not necessarily say anything about usability, but they do say something about the issues the researcher encountered.

%vv
Software engineers can use Ampersand to design and prototype registration systems.
What knowledge does an \acrlong{se} need to perform this analysis task and design with Ampersand.
They are used to designing and developing in a programming language.
Prototyping is often part of their development task.
Ampersand is both a design tool and a development method.
In addition, Ampersand provides the ability to have it evaluated via a prototype.

%v
The stakeholders are able to make a statement about the usefulness of Ampersand for the use of registry system within a government organization.
Because they have a relationship with the current registry system called Zorro.
From that relationship they make statements about the Concepts and Relationships within the generated \acrlong{ca}.
They look at the usability of Ampersand in designing and developing systems, at the incorporation into the architecture and  at the possibilities of using the \acrlong{ca} and the prototype.
Based on the observations and interviews, we make a swot analysis of Ampersand for the \acrshort{cibg} organization.

%v
There have been previous studies in which Ampersand plays an important role.
This has not encouraged the use of Ampersand to date.
The work of \citeNonPub{baecke_elleke_argument_2018} looked at Argument assistance software in legal reasoning.
There, Ampersand is used to build a prototype to support professionals in legal reasoning and not for the automation of argumentation.

In the dissertation of Pim Bos~\citepNonPub{bos_pim_bedrijfsregels_2013}, this author investigates the feasibility of implementing the business rules of the VOG case using semantic web technologies.
This work compares Ampersand with Semantic Web Rule Language at the level of relation algebra.



