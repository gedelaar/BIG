\newpage
\section{Introduction} \label{Introduction}
\begin{comment}


This research is about the usefulness of Ampersand for designing register systems.

Registry systems are designed and built for a legal task and thus have a specific purpose.
In order to perform this statutory duty, register systems must be accessible to the target groups.
The target groups need relevant information from such a register system.
The statutory duty requires the registry systems to be up-to-date.
The target groups must have confidence that the quality of the data is in order.
That is why register systems that are built on the basis of legislation and regulations are always built and maintained by government agencies.

The case addressed in this research concerns the \acrfull{big}.
The \acrshort{big} is a compilation of legislation and regulations concerning the determination of the qualification of care providers.
It is a register that falls under the responsibility of the Ministry of Health, Welfare and Sport.
Within this ministry, the implementing organization CIBG is responsible for managing the register.
The CIBG monitors all aspects that a register system must comply with.
To be able to do this, she has built a register system to manage the data.
The registry system, an information system called Zorro (\acrlong{zorro})~\footnote{\url{https://www.bigregister.nl/}} was built in 2008 and an \acrfull{alm} has established that it needs to be replaced~\citepNonPub{de_kok_analyse_2019}.
\begin{wrapfigure}{r}{0.4\textwidth} 
    \includegraphics[scale=0.13]
        {04_images/big-register-cibg.png}
    \caption{Big-register}
    \label{fig:Big-register}
\end{wrapfigure}


The CIBG has asked whether it is possible to translate directly from the legislation and regulations into a design and possible implementation.
We are going to use Ampersand for the design.
We put the implementation part out of scope.
Ampersand is a design method that designs and models from the source, i.e. the law and regulations.
As a result, it is not necessary for the design process to rely on user information.
The relevant concepts, relationships and rules are determined on the basis of legislation and regulations.
Ampersand is therefore based on the relation algebra and enforces validation based on the concepts, relations and rules found~\citep{joosten_software_2017}.
To make the design process visible, Ampersand has tooling to generate a functional design from a conceptual model.
In addition, Ampersand offers the possibility to make a prototype.
This prototype, together with the design, is used to validate the model.
The involvement of the CIBG organization is essential for this validation process.

The research concerns an authentic situation, namely the redesign of the registry system of \acrshort{big} is necessary.
The case to be investigated is a real-life situation, also the \acrshort{big}.
The demand and the support comes from the CIBG organization.
The researcher himself knows the register system and, in general, the legislation and regulations from which the system originates, which makes it a participatory form.
As a result, Action Research~\citep{Easterbrook} has been chosen as the approach.

On the basis of action research, we investigate the usefulness of Ampersand for designing register systems.
Usability indicates to what extent this method meets the need.
It's about the ability to use Ampersand for register systems design.
Is Ampersand immediately deployable, or is supporting knowledge required?
It is also questionable whether the legislation and regulations are suitable as a source for Ampersand.
The question for government organizations is what the strengths and weaknesses are of using Ampersand for registration systems.

In the \nameref{context}, section \ref{context}, we discuss the related topics.
This is a closer look at Ampersand's basics, namely relation algebra, in subsection \ref{relation_algebra}.
Ampersand is also discussed in subsection \ref{ampersand}.
We give an idea how an Ampersand script looks like.
Unlike the common process approach, Ampersand uses an event-oriented approach.
This is explained in subsection \ref{reactive_approach}.
The current system, which focuses on a workflow that will be replaced, is Zorro and the case \acrshort{big} is also discussed.
In section \ref{problem_analysis} we look at the topic of the action research, namely the \acrshort{big}.
We determine the reason for the choice of action research.
In section \ref{Research} we look at the main question and related issues.
We also discuss the method, approach and validation of the action research.
The last section \ref{planning} focuses on planning.

commentaar van Stef Joosten (dd 14-1-22) hierop

    Over de introductie: De huidige redeneerlijn in de introductie is:
        eerste alinea: het onderzoek gaat over de bruikbaarheid van Ampersand voor het ontwerpen van registratiesystemen. 
        tweede alinea: toelichting registratiesystemen.
        derde alinea: geeft een toelichting op de casus.
        vierde alinea: noemt de belanghebbende, CIBG, en stelt dat we Ampersand gaan gebruiken.
        vijfde alinea: benoemt de argumenten waarom Action Research als aanpak gekozen is
        zesde alinea: stelt dat dit werk de bruikbaarheid van Ampersand voor het ontwerpen van registersystemen onderzoekt.
        Alinea 7 wil een leeswijzer zijn.

Deze redeneerlijn roept vragen op:

        Waarom moet de bruikbaarheid van Ampersand worden onderzocht?
            Wat is Ampersand?
            Wat is “het gebruik” van Ampersand?
            Waarom is de bruikbaarheid van Ampersand een probleem?
        Waarom specifiek voor registratiesystemen? Wat is dat precies?
        Hoe onderzoek je de bruikbaarheid van Ampersand voor het ontwerpen van registratiesystemen?
        Waarom juist action research?

Deze vragen worden niet in de introductie beantwoord, wat bij mij als lezer een onbevredigend gevoel achterlaat. Maar hoe doe je dat? Welnu: Door deze vragen te beantwoorden in je betoog krijg je een veel betere redeneerlijn. Immers, alle vragen die je oproept heb je beantwoord. Ik heb de verkorte antwoorden achter de vragen gezet, bij wijze van voorbeeld. Overigens, ook die antwoorden roepen vragen op, maar die vragen zitten een niveautje dieper. Ik heb er een aantal bij opgeschreven. Die vragen kun je dus in de daaropvolgende hoofdstukken beantwoorden.

   
\end{comment}
\begin{comment}
    \item Wat is het onderzoek? Antw: we onderzoeken de bruikbaarheid van Ampersand voor het ontwerpen van registratiesystemen.
    \item Waarom moet de bruikbaarheid van Ampersand worden onderzocht? (antw: we zien dat het gebruik van Ampersand uitblijft)
    \begin{enumerate}
        \item Wat is Ampersand? (antw: Ampersand is een formele taal die registratiesystemen specificeert)
        \item Wat is het gebruik van Ampersand? (antw: Software engineers kunnen Ampersand gebruiken om registratiesystemen te ontwerpen en er prototypes van te maken.)
        \item Waarom is de bruikbaarheid van Ampersand een probleem? (antw: Ampersand wordt nauwelijks gebruikt, terwijl het wel werkt.)
        \begin{enumerate}
            \item Zijn er vergelijkbare systemen met vergelijkbare bruikbaarheidsproblemen?
            \item Wat zouden mogelijk oorzaken kunnen zijn?
        \end{enumerate}
    \item Waarom specifiek voor registratiesystemen? Wat zijn dat eigenlijk? (beantwoorden vanuit de literatuur, en goed afbakenen, maar wel zodanig dat het BIG-register er binnen valt.)    
    
    \item Hoe onderzoek je de bruikbaarheid van Ampersand voor het ontwerpen van registratiesystemen? (antw: omdat we niet precies weten wat de oorzaak is, moeten we het exploratief aanpakken)
    \item Waarom juist action research? (antw: De onderzoeker is belanghebbend. Het CIBG, de werkgever van de onderzoeker, heeft er belang bij. Het ontwerpen van de opvolger van Zorro is een enkele casus die goed toegankelijk is voor het onderzoek. Er zijn te weinig referentiecasussen om kwantitatief onderzoek mee te doen. Motiveren vanuit de karakteristieken van action research die je uit de literatuur haalt. Objectieve analyse is hier dus niet mogelijk. Het ligt dus voor de hand om de bruikbaarheid van Ampersand te bestuderen door actief deel te nemen aan het ontwerptraject en nieuwsgierig te bestuderen wat Ampersand doet in dat traject.
        \begin{enumerate}
        \item Welke onderzoekstechnieken zijn hierbij nuttig?
        \begin{enumerate}
            \item gesprekken met belanghebbenden (waarom?)
            \item content analyse op aantekeningen (waarom?)
        \end{enumerate}
    \end{enumerate}

    \end{enumerate} 
\end{comment}
Ampersand is the subject of research.
The usefulness of Ampersand for designing registration systems is being investigated.
The registration systems to be designed have a legal basis.
They are based on laws and regulations.

In practice, we see that Ampersand is not used much.
In the past there have been a few projects in which Ampersand has played a role.
For example, this was the \acrlong{indigo} project.
Within this project, Ampersand was used for the conceptual analysis.

Ampersand is a formal language that is used in this case to specify registration systems.
This language is based on relation algebra.
We are going to convert legislation and regulations to Ampersand.

Software engineers can use Ampersand to design and prototype registration systems.
Ampersand uses scripting that produces a database structure and web interface after compilation.
In the setup provided by the \acrlong{ou}, Ampersand is deployed within a Docker environment.
By deploying the database structure and the web interface within Docker, a container is created that provides access to the generated prototype.

Ampersand its usability is apparently an issue, which is reflected in its low usage.
The language and method of use work in practice. This has become apparent from the efforts made in other projects.
But the language is not used.
The question is why this is not used.
Possible cause of the current low usage may be due to the unfamiliarity of Ampersand.
This produces circular reasoning.
Unknown means that it is rarely used.
When it is rarely used, it remains unknown.
The popular products in the Open Source market are affiliated with a large organization that can push the marketing and knowledge. 
For example operating systems like Ubuntu which is maintained by Canonical Ltd.
In addition, such an organization can also build a community.
The usability of a product like Ampersand says nothing about its use.
It is not always the case that a useful product is widely used.
In the IT world, this happens with the Linux desktop versus a Windows desktop.
The usability of a Linux desktop is good.
It contains all the functionalities that are needed and has a nice user interface.
But many users still opt for a Windows desktop.
Windows apparently has features that are considered indispensable.
Windows was also a forerunner in user-friendliness and has a large marketing department
Despite the usability of Linux, it is mainly chosen by IT people or users who want a free operating system.

A register is a list of data that is collected on a legal basis.
The scope is determined by the legislation on which it is based.
Usually this scope is national or European.
Due to its legal basis, the register is also reliable.
This reliability must be guaranteed by the management organisation.
The registry data is collected centrally and is intended to provide insight into the registrations.
For this, data elements have been designated that must be present.
In addition to reliability, a register is also accessible.
So that it can be used for individual consultation and also as research data as used in ~\cite{schmidt_danish_2015} and \cite{bakken_norwegian_2019}.

Due to the legal basis of the register, it is by definition extensively described in the form of legislation and regulations.
Ampersand is used to convert these laws and regulations into register design.
A register design is therefore very suitable to do with Ampersand.

It is not known what causes the low usage of Ampersand in register system design.
To find out, it has to be approached exploratory.
The chosen exploratory approach is therefore \acrfull{ar}~\citep{Easterbrook}.
The exploratory approach lends itself to the research where Ampersand is used to design register systems to, as \cite{Easterbrook} argues, derive hypotheses and construct theories.
The research therefore focuses on a theory for the low use of Ampersand.

The approach of \acrshort{ar} was chosen because the researcher is part of the research.
The CIBG, the investigator's employer, has an interest in the investigation and is particularly interested in the design of the register.
Designing the successor to the registration system Zorro, the registration system for the \acrshort{big}, is a case that is easily accessible for research.
There are too few reference cases available to conduct quantitative research.
Objective analysis is therefore not possible here.
It therefore makes sense to investigate the usefulness of Ampersand by actively participating in the design process and curiously investigating what Ampersand does in that process.

The research techniques used are limited to discussions with stakeholders and a substantive analysis of the collected material during the research.
In this case, the stakeholders are located in different parts of the organization.
From the point of view of legislation and regulations, these are company lawyers and the company's own business units.
The business unit is responsible for the implementation of the \acrshort{big}.
This is the basis of the conversation with the product owners.
Architects are involved to enable embedding in the architecture.
From the architecture came the first request for a regulatory inquiry into application design.

