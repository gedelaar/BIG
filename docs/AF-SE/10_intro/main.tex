\newpage
\section{Introduction} \label{Introduction}
%vvv
Ampersand~\citep{joosten_software_2017} is used to specify information systems.
We investigate the usefulness of Ampersand for designing a special form of information systems, namely registration systems that serve as registers.
These registration systems are information systems where registrations are managed and these registrations are based on laws and regulations.
In this research we want to investigate whether Ampersand can be used by government organizations that build and manage this form of registration systems.

%Ampersand
%vvv
Ampersand is a rules-based design tool.
It is a way of designing information systems that comply with all business rules.
Business rules are therefore sufficient as a tool to design registration systems that comply with these business rules.
Ampersand is a method and tool to create error-free specifications of the registration system to support business processes and provides support for the way the design and \acrlong{ca} are created.
We use this tooling to translate the design into a prototype and \acrshort{ca} that comply with all relevant company rules.

%Vvv
Ampersand is based on Relation Algebra~\citep{maddux_bibliography_2006}.
The most commonly used components of Ampersand its relation algebra are the Concept, the Relation and the Rules.
A Concept is an abstract representation of immutable items and a Relationship defines a connection between two or more Concepts.
The Rules are about validating relationships between concepts.

%vvv
We use the Ampersand method to translate legislation and regulations into a system that demonstrably complies with the rules of the law.
Legislation and regulations are strongly rules-oriented and because Ampersand is a rules-oriented design method, it is ideally suited for legislation and regulations.
We will investigate whether \acrfull{big} is also suitable for the Ampersand analysis method.

%designing
%vvv
The design process within Ampersand focuses on making a prototype and the \acrlong{ca}.
We analyze the legislation and regulations with the Ampersand method and with the Ampersand tooling we make a script through which the prototype and design are created.
With this script Ampersand generates a prototype and \acrlong{ca}.
Ampersand is both the method followed and a tool to obtain the result.
One of the research results is the \acrlong{ca} in which we can see which Concepts, Relations and Rules \acrshort{big} gives us.

%registration
%vvv
Ampersand is deployed at the \acrshort{cibg}, which is an executive organization of the Ministry of Health, Welfare and Sport and therefore a government organization.
Implementing organizations work on the basis of legislation and regulations.
The \acrshort{cibg} is an implementing organization that designs, builds and manages registration systems.
These registration systems, called registers, are always based on legislation and regulations and aim to provide reliable and accessible information so that it can be used for individual consultation and as research data as used in \cite{schmidt_danish_2015} and \cite{bakken_norwegian_2019}.
The reliability of the data in the register must be guaranteed by the management organisation.

%vvv
The usability of Ampersand is apparently an issue, which is reflected in the low usage, despite the tool and method of use working in practice.
This has been shown by the efforts made in other projects, nevertheless the method is not widely used.
The question is why this occurs.

%investigated
%vvv
The reason for the low usage of Ampersand for designing registry systems is unknown and there is insufficient information to determine the causes.
To find out the reason for the low usage, we conduct an exploratory study.
Therefore it was decided to use the exploratory approach \acrlong{ar} of \cite{Easterbrook}.
\cite{Easterbrook} his exploratory approach lends itself to the research where Ampersand is used to design register systems to derive hypotheses and construct theories.

%vvv
The investigator is not wholly independent of the case under investigation.
We use the \acrshort{ar} approach because the researcher is closely related to the research case.
Other arguments in support of the \acrshort{ar} approach relate to the \acrshort{cibg}.
The \acrshort{cibg}, the researcher his employer, has an interest in the research and is particularly interested in the organization of the register.
Designing the successor of the \acrshort{zorro}, is a case that is easily accessible for research.
There are too few reference cases available to conduct quantitative research.
Due to the interdependence of the researcher, an objective analysis is therefore not possible here.
That is why it makes sense to investigate the usefulness of Ampersand by actively participating in the design process and curiously investigating what Ampersand does in that process.

%vvv
The research techniques used are limited to interviews with stakeholders and an analysis of the material collected during the research.
By showing stakeholders what Ampersand means for the design of the register, statements about the usefulness of the method can be elicited here.
We will also build an environment using the Ampersand method to show what Ampersand can deliver.
During the analysis and construction we will collect observations about things we come across.

%vvv
A possible cause of the current low consumption could be the unfamiliarity of Ampersand.
This produces circular reasoning:
Unknown means it is rarely used, if it is rarely used it remains unknown.
The popular products in the Open Source market are affiliated with a large organization that can push marketing and knowledge.
For example: operating systems such as Ubuntu maintained by Canonical Ltd that as a large organization can drive usage.
In addition, such an organization can also build a community to support the product.
The usefulness of a product like Ampersand says nothing about its use, because it is not always the case that a useful product is widely used.
In the IT world, this happens with the Linux desktop versus a Windows desktop.
It contains all the functionalities needed to work with and has a functional user interface, but many users still opt for a Windows desktop.
Windows apparently has features that are considered indispensable.
Windows was also a forerunner in usability and has a great marketing department.
Despite the user-friendliness of Linux, it is mainly chosen by IT people or users who want a free operating system.

%vvv
In recent years, experiments have been conducted at TNO, KPN, Bank MeesPierson, ING-Bank, Rabobank and Delta Lloyd.
They have experimentally confirmed the method and provided insight into its practicality.

%vvv
Building the environment as a software engineer provides data about the usability of Ampersand.
By classifying this data, we gain quantitative insight into the points that stand out the most.

%vvv
Software engineers can use Ampersand to design and prototype registration systems.
An \acrlong{se} is used to designing and developing in a programming language, but what knowledge does an \acrlong{se} need to perform this analysis task and design and prototype with Ampersand.
Ampersand is both a design tool and a development method for this analysis and construction task.
It is also possible to use a prototype to evaluate the system.

%vVV
The stakeholders make a statement about the usefulness of Ampersand for the use of the registration system within a government organization, because they have a relationship with the current \acrshort{zorro}.
From that relation they can make statements about the Concepts and Relations within the generated \acrlong{ca}.
In addition, they look at the usefulness of Ampersand in the design and development of systems, at the integration into the architecture and at the possibilities of using the \acrlong{ca} and the prototype.
Based on the observations and interviews, we make a SWOT analysis of Ampersand for the \acrshort{cibg} organization.

%v
There have been previous studies in which Ampersand plays an important role.
The work of \citeNonPub{baecke_elleke_argument_2018} looked at Argument Helper software in legal reasoning.
There, Ampersand is used to build a prototype to support professionals in legal reasoning and not for automating argumentation.

In the dissertation of Pim Bos~\citepNonPub{bos_pim_bedrijfsregels_2013}, this author investigates the feasibility of implementing the business rules of the VOG case using semantic web technologies.
This work compares Ampersand with Semantic Web Rule Language at the level of relation algebra.

In section \ref{context} we discuss the related topics.
This is a closer look at Ampersand its basic principles, namely relation algebra, discussed in subsection \ref{relation_algebra}.
Ampersand as subject in subsection \ref{ampersand}.
In contrast to the usual process approach, Ampersand uses an event-oriented approach. This is explained in subsection \ref{reactive_approach}.
The current system, which focuses on a workflow that will be replaced, \acrshort{zorro} and the case \acrshort{big} are discussed in \ref{subsubsection:wet-big}.
In paragraph \ref{Method} \acrshort{ar} is the subject of discussion.
In section \ref{Results} we perform the content analysis, discussing the results in \ref{section:discussion} and the conclusions and recommendations in \ref{conclusions} and \ref{future_work}.

