\newpage
\section{Summary} \label{Summary}

\subsection{Dutch}
In deze thesis gaan we de bruikbaarheid van Ampersand onderzoeken door het ontwerpen van een register systeem bij een overheidsorganisatie.
Ampersand wordt niet breed gebruikt en we vragen ons af waarom dat het geval is.

Het \acrshort{cibg} is een uitvoeringsorganisatie van het Ministerie van Volksgezondheid, Welzijn en Sport. 
Deze organisatie, waar ik werkzaam ben, beheert register systemen.
Een register systeem, ook wel register genoemd, heeft voor ons altijd een wettelijke basis. 
Dus op basis van een wet, creëert het \acrshort{cibg} een register.
Om dit onderzoeken uit voeren hebben we een authentieke case genomen.
Het systeem dat de \acrshort{big} ondersteunt staat op nominatie om vervangen te worden.
De gekozen methode om het onderzoek uit te voeren is dan ook \acrshort{ar}.

Tijdens het ontwerp proces zijn observaties over het verloop van de Ampersand analyse vastgelegd. 
Ook zaken van Ampersand die opvallen zijn meegenomen.
De \acrlong{ca} die opgeleverd is en qua opzet besproken met geïnterviewde personen.

Alle observaties en interviews zijn middels content analyse gerubriceerd en hebben de basis gevormd voor de beantwoording van de hoofd- en subvragen. 
Deze hebben geleid tot het trekken van conclusies op de vraag of Ampersand bruikbaar is voor het ontwerpen van register systemen bij een overheidsorganisatie.
Het is hierbij opgevallen dat register systemen niet anders zijn dan andere informatie systemen.
Het grote verschil is dat een register systeem een wettelijk basis heeft en een informatie niet per definitie.
De bruikbaarheid van Ampersand om een wet te analyseren is goed, echter moet een organisatie bereid zijn om het te gebruiken. 
Natuurlijk zijn er verbeteringen mogelijk en er kan nog onderzoek plaatsvinden naar het gebruik van tooling om ontwikkeling te versnellen. 


\newpage
\subsection{English}
In this thesis we will investigate the usefulness of Ampersand by designing a registry system at a government organization.
Ampersand is not widely used and we wonder why that is the case.

The \acrshort{cibg} is an implementing organization of the Ministry of Health, Welfare and Sport.
This organization, where I work, manages registry systems.
A register system, also known as a register, always has a legal basis for us.
So based on a law, the \acrshort{cibg} creates a register.
To carry out these investigations, we have taken an authentic case.
The system that supports the \acrshort{big} is nominated to be replaced.
The chosen method to carry out the research is \acrshort{ar}.

During the design process, observations about the course of the Ampersand analysis were recorded.
Also items of Ampersand that stand out are included.
The \acrlong{ca} that was delivered and discussed in terms of structure with interviewees.

All observations and interviews have been classified by means of content analysis and have formed the basis for answering the main and sub questions.
These have led to conclusions about whether Ampersand can be used for designing registry systems in a government organization.
It has been noticed that registry systems are no different from other information systems.
The big difference is that a register system has a legal basis and an information not by definition.
Ampersand's utility for analyzing a law is good, but an organization must be willing to use it.
Of course improvements are possible and research can still be done on the use of tooling to accelerate development.