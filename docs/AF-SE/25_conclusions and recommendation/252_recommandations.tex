\subsection{Future work}\label{future_work}
Now that the research has been completed and the results and conclusions have been described, it is worth considering what further research could take place.
The conclusions revealed that there are omissions in the area of maintaining an overview.
In future research, attention could be given to the way in which this overview can be maintained.
This may move in the direction of annotation tools.
The follow-up research could focus on selecting and implementing tools within Ampersand for the purpose of maintaining overview.
An overview is also needed on the Concepts management side.
This seems to be possible by porting Atlas from RAP to the local environment.

In the context of maintaining the overview, it has been suggested to make use of the addition of XML in the source document.
So enriching the source document with the annotation XML.
The big advantage of this would be that it is then possible to generate the Ampersand script.
Especially after changes in the source document, where the existing annotations can be inserted in the new version.
This adjustment would be even more beneficial once there is an existing national base of Concepts and Relations.
The link between annotations and the national base could result in an enormous acceleration in development.
This is worth investigating, but will have to be split into several studies.

Another conclusion that has been drawn is that the \acrshort{big} is not the most suitable law to analyze it via the Ampersand method.
This is not the fault of Ampersand, but the law.
It is interesting to map out which requirements the law must meet in order to fit in well with the method and the follow-up is to examine how we can shape future laws that people would like to be supported by (register) systems so that they can be quickly analyzed by Ampersand.
Early participation by business analyst and Ampersand skilled lawyers in the legislative process could save a lot of time and money.
How much that may be is a topic for a research project.

One of the interviewees feared that the Ampersand approach would take more time in the design phase than the regular approach.
The regular approach includes a more or less agile approach, in which the design is made in outline, after which the system is divided into parts and these are made agile.
One could examine the design of two similar systems or possibly even the same system, with one done the Ampersand way and the other the regular way.
Then it is interesting to see what is faster, more complete and more workable for the follow-up process.

Another comment made during the interviews relates to the size of the system.
The hypothesis was that the system size of an Ampersand project will be smaller than the size of a system from a regular trajectory.
This could be related to the fact that Ampersand is directly on the source and does not want to include all kinds of peripheral matters.

During the research we were regularly confronted with the \acrshort{rk}.
It is worth investigating how exactly this link should be established.
Where are the similarities and where are the differences?
In this context we again come across the issue of the common Concepts.

To ensure a smooth start of a project, it is good to start from a standard set of agreements.
Let us develop best practices, including things like naming (upper and lower case, CamelCase, etc.) and suggestions regarding the use of source texts.
So that these can be used at the kick-off of an Ampersand project.

