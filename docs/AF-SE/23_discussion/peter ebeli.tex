gesprek gevoerd met Peter Ebeli, informatie analist
dd 28-1

wetbig -> is groot, oud
in het verleden bij uwv iets gelijks gezien. weet niet of dit ook gebruikt is.
daar was een klein groepje ermee bezig vs een groot team op "conventionele wijze". Project is daar gestrand

** door conceptuele analyse zou je ook tegenstrijdigheden in de wet kunnen tegenkomen.

prototype -> gebruiker kijkt niet altijd door het getoonde heen. vaak is de styling (de eerste indruk)  mede bepalend voor het oordeel
oplossing is om CSS toe te passen die prototype meer op de huisstijl laat passen.

plaatje van de architectuur lijkt ook voor processen op te gaan

zou ingezet kunnen worden voor nieuwe taken. van start af aan op deze lijn gaan zitten.
aan de andere kant, door bestaand systeem te analyseren, zou de waste eruit gesneden kunnen worden. 
benieuwd hoeveel dit zou schelen.

vraag is even hoe snel je een basis hebt neergezet
lastig om te komen tot een 100\% model. maar misschien is 80 ook goed ????

in de huidige situatie heeft de business al voorwerk gedaan en dan pas komt het bij ict terecht. dus zitten daar de nodige stappen tussen

LCSH zou een goede kandidaat zijn voor een nieuwe acties. is ook een kleine wet

zou je het gaan gebruiken:
- het is een manier om iets op te schrijven, niet beter of slechter dan iets anders
dus als je iets opschrijft, waarom dan niet hierin.
je kan er iets meer mee, dan met een gewoon Word docuement.
de output is te gebruiken, de structuur is te gebruiken. heb je keuze in.

vraag is even of het de businessrules in enkel in de db zitten of dat er meer is. 
er is natuurlijk meer, anders kunnen api's ook niet bestaan.
veel rules leunen tegen de db aan, huidige trend is om validaties in de businesslaag op te nemen

onderhoud is wel een dingetje
hoe gaat dit onderhoud?

via unit tests zou je steeds de code kunnen testen bij verschillen. ik betwijfel dat

onderhoud en traceerbaarheid, hoe komen we achter de verschillen tussen twee versies van Ampersand code.
hoe pakken we dit onderhoud aan

leercurve lijkt niet te groot
sterk: ook minder tech mensen kunnen hiermee werken, je kan ze makkelijk meenemen
styling prototype is aan te passen
snel een werkend systeem
ook op te vatten als conc.analsye
je kan op basis van de analyse al testscenario's maken. dus dit hangt af van de wijze implementatie

er zijn meerdere producten die vanuit een model code kunnen genereren
meestal wordt het model (tekening) gemaakt en van daaruit wordt iets gegenereerd, maar Ampersand werkt andersom
het model ( de tekening) is het resultaat

voor eenvoudige zaken zal dit zeker werken, maar ook voor complexe zaken?
de wetbig is complex

vervolg -> vergelijk tussen bestaande bouw en nieuwbouw in hoeveelheid code
door code generatie is een systeem kleiner

betere sig-kwalificatie

om mee te starten, zou een team opgericht moeten worden die zich hiermee bezig houdt. het moet durven aan te pakken.

het onderhoud is anders. Onderhoud is met A, het genereren van een nieuw systeem en daarna een puzzel hoe de data conversie moet plaatsvindne.
meestal wordt op voorhand rekening gehouden met de datastructuur zodat er zo weinig mogelijk conversie hoeft plaats te vinden.
dit maakt (waarschijnlijk) dat een systeem steeds groter wordt en steeds minder beheersbaar. 
dus de kracht van A is dat dit voorkomen wordt omdat je steeds een nieuw, core systeem, bouwt en de inspanning zit in de data conversie en de aansluiting van aanpalende systemen.

men zou ook enkel de output van de analyse kunnen gebruiken om een systeem te bouwen
meerdere scenario's zijn mogelijk

goed om de huidige informatia analyse groep en business analyse groep bij de resultaten te betrekken.

samenvatting:
als analist is elke correct beschrijvend methode goed werkbaar.
bij de meeste methodes is het model het uitgangspunt voor de beschrijving en later de code.
Bij Ampersand is het anders. 
Daar is de beschrijving leidend en dat leidt tot een model en code.
onderhoud is een issue
