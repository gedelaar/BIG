\subsection{Limitions}\label{sbs:limitions}
The focus of the research was on executing the process to construct at a \acrshort{ca} and a prototype.
The time for this is in principle 26 weeks.
We have moved a little over time and it turns out that it is not possible to fully analyze just a law like \acrshort{big}.
The limitation we encountered is that there is a shortage of time.
This has to do with an optimistic estimate of what work can be done.
The start with Ampersand is more difficult than it seems.
The \acrshort{big} is much larger and more complex than it first appears.
Reading the law is also an art.
The law has many references to other laws.
This resulted in an \acrshort{ca} which is not complete.
The process did provide enough material to make several statements (see section~\ref{conclusions})

\begin{comment}
discussie voer

niet alle observaties zijn gebruikt

Een risico dat gelopen wordt door de moeilijke teksten is dat er niet zorgvuldig genoeg gelezen wordt en er eigen interpretatie plaatsvindt.
Dat risico neemt toe, naar mate het domein vertrouwder is voor de onderzoeker. 
Dus de keerzijde van de \acrshort{ar} aanpak is een bias op de inhoud.

De overall aanpak van de analyse van de wet is om eerst een overzicht te krijgen van de wet.
Het doorlopen van de wet en de highlights van de artikelen helder te krijgen.
Dit sluit aan bij het idee om de indeling op voorhand te maken


Vanuit verschillende interviews werd het statement gemaakt of \acrshort{big} wel de meest geschikte wet is om deze met Ampersand te analyseren.
Reden is dat de wet van orgine heel oud is~\ref{section:big}. 
De wet is verschillende malen bijgewerkt, maar de structuur is niet simpel om te zetten naar een ICT-systeem.
Daarnaast bevat de wet zeer veel impliciete en expliciete verwijzingen naar andere wet- en regelgeving.
En de wet is zelf niet expliciet genoeg.
Er zijn behoorlijk veel interpretatie mogelijkheden.
-> leidt tot sneller aansluiten bij de tot stand koming van de wet
-> vroegtijdige analyse van de haalbaarheid
-> alternatieve aanpak etc.
\end{comment}

