\subsubsection{Threats}\label{subsub:4_treaths}

\paragraph{\textbf{NIH}}\label{swot:t_nih}:
Ampersand is a completely new method for the \acrshort{cibg} organization.
People have never heard of it and unfortunately there is not much to be found about it.
That means that people are not positive about it in advance, and NIH calls it~\footnote{not invented here}\citep{antons_assessing_2017}.
The expectation of the organization is that the Ampersand method will take more time than the current method (see interview~\ref{int:I-1.8}) and people will be wary of something new.
Obviously, the advantages of the method are not yet understood.
Benefits such as working directly at the source, generating a prototype from there (see appendix~\ref{appendixPrototype}) with all validations and full conceptual specifications (see appendix~\ref{ConceptualAnalysis}).
Having a prototype makes it possible to build test scenarios at an early stage and with the \acrlong{ca} you can start building right away.
(Ref. to \nameref{s:4_5_test_scenario}, \nameref{s:2_6_prototype_use}, \nameref{s:1_11_law_effective})

\paragraph{\textbf{Process orientation}}\label{swot:t_process_orientation}:
Many organizations are process oriented.
With \acrshort{cibg}, the work instructions that the employees use are usually process steps that must be performed.
Also the design of systems such as \acrshort{zorro} are designed to support a process-oriented approach.
Ampersand supports a reactive focused approach.
This cannot be translated into process-oriented work instructions.
The challenge for the adoption of Ampersand therefore lies partly in the organization its ability to adapt from process-based to reactive.
Also contributing to this is that the adjacent system, \acrshort{rk}, is a process-oriented system, which supports a number of standard processes that affect (almost) every register.
(Ref. to \nameref{s:1_7_architecture_and_registerkern}, \nameref{s:5_1_registerkern}, \nameref{s:5_2_demarcation})

\paragraph{\textbf{Redundancy}}\label{swot:t_redundancy_script_tool}:
The reuse of Concepts, Relations and Rules is on the one hand a powerful means whereby consistency is enforced.
However, when using Ampersand, it is not always visible that components are already being used.
For example, it is possible to include these components several times in the analysis and to assign them a different definition each time.
While this should be detectable by Ampersand.
Conversely, it is also possible to rename components with the same definition.
Of course, Ampersand cannot control this.
The IDE used should support this, but it does not at the moment.
(Ref. to \nameref{s:5_2_demarcation})

\paragraph{\textbf{IDE-refactoring}}\label{swot:t_ide_refactoring}:
Ampersand as a tool is supported by an IDE.
For \acrshort{vsc} there is a plugin that supports the use of Ampersand.
In addition, the Ampersand tool is built in such a way that the compilation of the script from the \acrshort{vsc} is supported.
An omission in the use of \acrshort{vsc} is a refactoring option.
To be able to refactor, the IDE tool needs to know which components are used and where they are used.
This condition is not met, where refactoring is not possible in advance.
Not being able to refactor is a cause of causing redundancy in the code (see paragraph~\nameref{swot:t_redundancy_script_tool}).
(Ref. to \nameref{s:3_2_common_objects})



