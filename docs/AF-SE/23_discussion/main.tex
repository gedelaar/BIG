\newpage
\section{Discussion} \label{Discussion}

aanpak log verwerking:
- totaliseren van de results
- actie research: wat ben ik aan het doen
- experimenteren in de praktijk
- deel uitmaken van het proces
- laten gebeuren
- valt me op dat .... -> content analyse
- cluster waarneming -> beschouwen: het valt me op dat .. en onderbouwing
- bottom up (logs) en top down (gesprekken)
- bij elk log -> wie wat waar; traceerbaar houden


5-1:
algemene presentatie aan Seniz, Vivianne, Carina
-> wet big is eigenlijk niet geschikt voor dit
- is groot
- is complex
- is verouderd
- veel interpretatie mogelijkheden (moet ook wel, want niet alles is goed beschreven)
- door de betekenis is er veel ruimte in de wet
- gedrocht van een wet, zou eigelijk weggegooid moeten worden en opnieuw gestart
- er moet wel interpretatie ruimte blijven


11-1:
- jurist -> ben je in staat om het goede inhoudelijke gesprek te voeren en in hoeverre heeft Ampersand je daarin geholpen.
- architect -> idem

analyse van de log => content analyse -> hier artikelen over zoeken


11-1: gesprek Carina:
- voorbereiding:
* algemene inleiding waarom deze sessie
* conceptuele analyse meegestuurd.

- uitvoering:
* J wil zich graag een beeld vormen van het ICT systeem
* de concptuele analyse mn de concepten en de beschrijving daarbij komt vertrouwt over. Zo ziet de wet er ook uit.
* de wetbig biedt veel ruimte voor interpretatie
    nieuwere wetten zijn wat zorgvuldiger opgezet
* je moet niet alles in het systeem willen vastleggen. 80 procent en de uitzonderingen met de hand door geschoolde medewerkers
* de wet geeft een kader, vraag is even hoever je moet gaan
* daarbinnen is ruimte om zaken zelf in te vullen
* doel ict systeem is om zo min mogelijk handwerk te doen
* bij het opstellen van een nieuwe wet, zou er een ict-er bij moeten zitten
* definities zijn vaak niet eenduidig
* juriprudentie maakt de wet mogelijk net even anders dan wetgever ooit bedoelde

doorlopen van de wetgeving:
a1-definities; 
a3 welke beroepen ed
a4 uitbreiding a3
a5 grondslagen voor regelgeving
a6 weigergronden
a7 doorhalen - 7a hardheidsclausules
a8 basis voor herregistratie, technische artikelen
a9 tuchtgebeuren; wat we aantekenen op het register
(inschrijving = registratie)
maatregelen
doorgehaalde reden bepaald of je zichtbaar bent 
a10 beschikking
a11 aanmelden beschikking (staatscourant)
a12 openbaarmaking big-registratie - staat wat er gemeld mag worden
a13 privacy + delen van info; grondslagen per doelgroep
a14 beroepsverenigingen - wordt aangetekend in big-register
a15.16.17 specialisten registers
H3-a18 eisen per beroep oa opleiding
tm a33
a34 geen beroepstitel, maar wel behandeld als a3-beroep; ook opleiding is bepalend. Geen eigen register
a35 voorbehouden handelingen
a36a+b tijdelijke registers bv mondhygenistes
h5 tav buitenlandse gediplomeerden; erkenning process (EU, overige buitenland)
a45 als a34
h6/7 tucht
etc

conclusie: 
werkt graag mee. bereid om veel over de wetbig (inhoudelijk) te vertellen.
is een beetje angstig naar ict toe, wil graag weten hoe het er dan uit gaat zien (wil zich een beeld vormen)
Ampersand kan de behoefte invullen. Het is heel tekstueel, waardoor een jurist zich herkent in de conceptuele analyse.
Door een voorbeeld te geven en een stukje uit te werken, is de herkendbaarheid groot en de mee werk bereidheid ook.

12-1: gesprek Stef
- geeft aan dat de omliggende wetgeving conflicterend kan zijn.
- A is interessant
- aanvraag ed, zaak elementen zit er zijn's inzien niet zo in.
- registerkern : technologische manier gelijk hebben
- herkent de gedeelde concepten (althans het idee)
- verschillende wetgevingen naast elkaar, waar zijn de gemeenschappelijke termen (opvallend, normaal is de scope een dienst, nu opeens de gehele wetgeving)
- gevaar van wet- en regelgeving. Niet compleet zijn in de concepten, 1:1 overnemen regels en implementeren zonder verdere interpretaties
- twijfels,meer wetgeving wordt gebruikt. Ook wetten die je nu niet direct ziet. Roll up
- wat is nodig om A wel te kunnen  gebruiken:
* zou handig zijn om te voorkomen dat de code in C\# gemaakt zou moeten worden
* hoe omgaan met de veranderende regelgeving. OVerlappende termijnen bij nieuwe/gewijzigde regelgeving. Ingediende aanvraag gold nog voor oude regeling of juist niet.
- onderhoud is wel een issue
- oplossing : eenmalige implementatie
- of delta bepalen met verschillen analyse op zowel technisch als functioneel niveau
- gebruik: hard hoofd in, 
mogelijkheid technische stap overslaan
posisitioneren als interpretator van de wet- en regelgeving.
daarna huidige analyse en ontwikkelproces in stand houden
prototype als prototype gebruiken
- dient dus al voorwerk voor ontwikkelproces 
- sceptisch of het niet meerwerk gaat opleveren
- sceptisch over de toegevoegde waarde van A
veranderen is lastig

TODO
