12-1: gesprek Stef v duijck
- geeft aan dat de omliggende wetgeving conflicterend kan zijn.
- A is interessant
- aanvraag ed, zaak elementen zit er zijn's inzien niet zo in.
- registerkern : technologische manier gelijk hebben
- herkent de gedeelde concepten (althans het idee)
- verschillende wetgevingen naast elkaar, waar zijn de gemeenschappelijke termen (opvallend, normaal is de scope een dienst, nu opeens de gehele wetgeving)
- gevaar van wet- en regelgeving. Niet compleet zijn in de concepten, 1:1 overnemen regels en implementeren zonder verdere interpretaties
- twijfels,meer wetgeving wordt gebruikt. Ook wetten die je nu niet direct ziet. Roll up
- wat is nodig om A wel te kunnen  gebruiken:
* zou handig zijn om te voorkomen dat de code in C\# gemaakt zou moeten worden
* hoe omgaan met de veranderende regelgeving. OVerlappende termijnen bij nieuwe/gewijzigde regelgeving. Ingediende aanvraag gold nog voor oude regeling of juist niet.
- onderhoud is wel een issue
- oplossing : eenmalige implementatie
- of delta bepalen met verschillen analyse op zowel technisch als functioneel niveau
- gebruik: hard hoofd in, 
mogelijkheid technische stap overslaan
posisitioneren als interpretator van de wet- en regelgeving.
daarna huidige analyse en ontwikkelproces in stand houden
prototype als prototype gebruiken
- dient dus al voorwerk voor ontwikkelproces 
- sceptisch of het niet meerwerk gaat opleveren
- sceptisch over de toegevoegde waarde van A
veranderen is lastig