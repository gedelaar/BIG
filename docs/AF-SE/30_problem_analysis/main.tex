\section{Problem analysis} \label{problem_analysis}
\subsection{Implementation of wet-Big} \label{implementation_wetbig}
The action research has as subject the \acrlong{big}.
This law dates from 1993 and is still valid today.
As mentioned before, the law will eventually be replaced as stated by~\citeNonPub{bussemaker_jet_2019}.

The current system, called \acrshort{zorro}, is built to support the \acrshort{big}.
Research in 2019 showed that the current system is end-of-life.
It has been depreciated financially, but also technically it is lagging behind and can only be maintained at high costs.
Also, not all intended processes can be supported within \acrshort{zorro}.
The problems with Zorro show that a workflow approach involves a high administrative burden, because exceptions to the rule are always necessary.
In order to determine the usefulness of the Ampersand method, we will investigate this case with a real-life situation, namely \acrshort{big}.
Since the BIG system needs to be replaced, this is an authentic problem.  
As an application manager, I have been involved with BIG-system for a number of years.  
The CIBG organization has asked whether it is possible to investigate a more direct link and translation of the legislation towards registers. 
That is why we are going to use Action Research~\citep{Easterbrook} as a research approach.


\begin{comment}
Het informatiesysteem van het BIG-register biedt onvoldoende toegevoegde waarde aan de business- en beheerorganisatie. In de dagelijkse operatie wordt door Behandelaren veelvuldig gebruik gemaakt van workarounds. Met name de uitvoering van de taken Tucht en Erkenning Buitenlandse Diploma’s worden door het systeem niet ondersteund en kan uitvoering alleen gedaan worden met veel handmatig werk buiten het systeem om.
De beheerorganisatie moet veelvuldig ingrijpen met behulp van SQL scripts en heeft daardoor een grote rol in het primaire proces.
Het informatiesysteem voorziet niet in de informatiebehoefte van de business, opdrachtgever en externen (zoals de pers), waardoor deze alleen met behulp van handmatige query’s op de database kunnen worden beantwoord.
Het systeem is enkel na intensief ontwikkelwerk aan te passen aan nieuwe wet- en regelgeving.

Security issues gaan gezien de gebruikte verouderde software toenemen. Componenten voldoen niet aan de huidige en komende security vereisten. Dit kan uiteindelijk leiden tot imagoschade, bestuurlijke onrust en afsluiten van DigiD.
Daarnaast voldoet het systeem niet aan vigerende regelgeving zoals de AVG en de Archiefwet.
De Zorro client, Adminconsole en Advieswijzer gebruiken veelal verouderde techniek. De backend kan worden vereenvoudigd, er ontbreekt een diagnostics service en op de database worden verschillende technieken gebruikt die allen niet voldoen aan de CIBG Standaard. Vanuit Architectuur wordt sterk aanbevolen om te onderzoeken of deze applicatie niet geheel opnieuw ontworpen moet worden.

Uit de uitgevoerde analyse blijkt op alle aspecten nieuwbouw nodig.
Echter, uit de analyse blijkt ook dat op veel aspecten wordt aanbevolen een nadere analyse uit te voeren.

Het advies is dan ook om voor het BIG-register een nadere Business analyse uit te voeren zodat Business- en beheerprocessen nader kunnen worden geanalyseerd, knelpunten kunnen worden gedefinieerd, deze kunnen worden vertaald naar verbeterpunten en een business case kan worden opgesteld.
In de business analyse kan worden onderzocht of de huidige processen geoptimaliseerd kunnen worden of dat implementatie van een nieuw proces efficiënter en voordeliger kan zijn. Hiermee worden de scenario’s moderniseren en nieuwbouw beargumenteerd.

Voor het huidige systeem moet in de tussentijd bekeken worden welke adviezen op basis van de uitgevoerde ALM analyse minimaal noodzakelijk zijn. Deze acties moeten in het beheerplan en de CIBG roadmap worden opgenomen met als doel de werking van het systeem te borgen tot modernisering of nieuwbouw gerealiseerd is.
\end{comment}
The system that supports the \acrfull{big} within the \acrshort{cibg} does not provide sufficient functionality for the organization.
People are insufficiently able to perform daily activities with the aid of the system.
Many activities require workarounds to support the process. 
In particular, the performance of the Disciplinary and Recognition of Foreign Diplomas tasks is not supported by the system and can only be performed with a lot of manual work outside the system.
The management organization often has to intervene using SQL scripts and therefore has a major impact on the primary process. 
The information system does not meet the information needs of the business, client and external parties (such as the press), so these can only be answered with manual searches in the database named by ~\citeNonPub{de_kok_analyse_2019}.
Modifications to the system require intensive development work.
In its current state, it is sensitive to security breaches and does not meet all \acrfull{gdpr}~\footnote{\url{https://gdpr-info.eu/}} and archive law requirements~\footnote{\url{https://www.nationaalarchief.nl/archiveren/kennisbank/wet-en-regelgeving}}.
Security breaches can lead to reputational damage and even to the closure of DigiD.


\subsection{Action research} \label{action_research}
The use of relation algebra contributes to the knowledge about register design with Ampersand and for that we use action research~\citep{Easterbrook}.
Mainly aimed at users who have never done anything with Ampersand.
We use Ampersand to convert the law into relation algebra to see how this works in real-life situations.
This research focuses on this translation and the aspects involved.
An important aspect of the Conceptual analysis is the common language that is defined.
The intentions are even more important than the actual text.

This concerns the prior knowledge needed to apply Ampersand.
Matters that will also be discussed are related to legislative changes.
How can these be included in the method.
How are they visible?
What about the stored data if there are changes in the law?
In a webinar about legal analysis from~\citeNonPub{belastingdienst_webinar_2021}, the relationship between articles of law and decisions taken is explicitly recorded.
The question is whether Ampersand also provides for this.

