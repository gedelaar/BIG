\section{Introduction} \label{Introduction}

This research is about the usefulness of Ampersand for designing register systems.

Registry systems are designed and built for a legal task and thus have a specific purpose.
In order to perform this statutory duty, register systems must be accessible to the target groups.
The target groups need relevant information from such a register system.
The statutory duty requires the registry systems to be up-to-date.
The target groups must have confidence that the quality of the data is in order.
That is why register systems that are built on the basis of legislation and regulations are always built and maintained by government agencies.

The case addressed in this research concerns the \acrfull{big}.
The \acrshort{big} is a compilation of legislation and regulations concerning the determination of the qualification of care providers.
It is a register that falls under the responsibility of the Ministry of Health, Welfare and Sport.
Within this ministry, the implementing organization CIBG is responsible for managing the register.
The CIBG monitors all aspects that a register system must comply with.
To be able to do this, she has built a register system to manage the data.
The registry system, an information system called Zorro (\acrlong{zorro})~\footnote{\url{https://www.bigregister.nl/}} was built in 2008 and an \acrfull{alm} has established that it needs to be replaced~\citepNonPub{de_kok_analyse_2019}.
\begin{wrapfigure}{r}{0.4\textwidth} 
    \includegraphics[scale=0.13]
        {images/big-register-cibg.png}
    \caption{Big-register}
    \label{fig:Big-register}
\end{wrapfigure}


The CIBG has asked whether it is possible to translate directly from the legislation and regulations into a design and possible implementation.
We are going to use Ampersand for the design.
We put the implementation part out of scope.
Ampersand is a design method that designs and models from the source, i.e. the law and regulations.
As a result, it is not necessary for the design process to rely on user information.
The relevant concepts, relationships and rules are determined on the basis of legislation and regulations.
Ampersand is therefore based on the relation algebra and enforces validation based on the concepts, relations and rules found~\citep{joosten_software_2017}.
To make the design process visible, Ampersand has tooling to generate a functional design from a conceptual model.
In addition, Ampersand offers the possibility to make a prototype.
This prototype, together with the design, is used to validate the model.
The involvement of the CIBG organization is essential for this validation process.

The research concerns an authentic situation, namely the redesign of the registry system of \acrshort{big} is necessary.
The case to be investigated is a real-life situation, also the \acrshort{big}.
The demand and the support comes from the CIBG organization.
The researcher himself knows the register system and, in general, the legislation and regulations from which the system originates, which makes it a participatory form.
As a result, Action Research~\citep{Easterbrook} has been chosen as the approach.

On the basis of action research, we investigate the usefulness of Ampersand for designing register systems.
Usability indicates to what extent this method meets the need.
It's about the ability to use Ampersand for register systems design.
Is Ampersand immediately deployable, or is supporting knowledge required?
It is also questionable whether the legislation and regulations are suitable as a source for Ampersand.
The question for government organizations is what the strengths and weaknesses are of using Ampersand for registration systems.

In the \nameref{context}, section \ref{context}, we discuss the related topics.
This is a closer look at Ampersand's basics, namely relation algebra, in subsection \ref{relation_algebra}.
Ampersand is also discussed in subsection \ref{ampersand}.
We give an idea how an Ampersand script looks like.
Unlike the common process approach, Ampersand uses an event-oriented approach.
This is explained in subsection \ref{reactive_approach}.
The current system, which focuses on a workflow that will be replaced, is Zorro and the case \acrshort{big} is also discussed.
In section \ref{problem_analysis} we look at the topic of the action research, namely the \acrshort{big}.
We determine the reason for the choice of action research.
In section \ref{Research} we look at the main question and related issues.
We also discuss the method, approach and validation of the action research.
The last section \ref{planning} focuses on planning.


